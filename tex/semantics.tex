\section{Semantics}

We now define trace models, denoted $\TM$, which are the structures needed to
give a meaning to indices and timestamps. More generally, a trace model
also allows to translate any meta-logic formula $\phi$
(over some meta-logic signature $\Sigma$)
to a base logic $\phi^\TM$ (over some base logic signature~$\Sigma^\TM$).

\medskip

From now on, we assume a protocol $P$ providing some set of action
symbols $\Act$, using function and name symbols from $\F$ and $\N$,
as well as memory cell symbols from $\States$.
We consider meta-logic terms and formulas over $\Sigma = (\F,\N,\Mac,\Act)$
where $\Mac = \{ \minp, \mout, \mframe \} \cup \States$. We require that $\Sigma$ contains the following function symbols, of index arity $0$:
\begin{itemize}
\item a function $\att$ of arity one, which models attacker computations.
% \item a function $\att_n$ of arity $n$ for every $n\in\mathbb{N}$, which models attacker computations. 
\item a binary function symbol $\langle \cdot , \cdot \rangle$ for pairs;
\item binary function symbols
  $\dand$, $\dor$ and
  $\dimpl$ for boolean operations;
\item a ternary function symbol $\mathsf{if}$;
\item nullary function symbols $\true$ and $\false$.
\end{itemize}
We use the usual notations for boolean symbols (infix), $\mathsf{if}$ and for pairs (e.g.\ $\langle t_1,t_2 \rangle$).

\begin{definition}
  Given a finite set of integers $D$, the base logic
  signature $\Sigma^D$ is contains exactly:
  \begin{itemize}
    \item a name symbol $n_{e_1,\ldots,e_k}$ for every $n \in \N_k$
       and $e_1,\ldots,e_k \in D$;
    \item a function symbol $f_{e_1,\ldots,e_k}$ of arity $n$ for every
      $f \in \F_k^n$ and $e_1,\ldots,e_k \in D$;
  \end{itemize}
\end{definition}

For example, if $f \in \F_1$ (resp. $k \in \N_1$) and we have two distinct
integers $n$ and $m$ in $D$, then $f_n$ and $f_m$ (resp. $k_n$ and $k_m$)
are distinct function (resp. name) symbols in $\Sigma^D$.
A computational model may give the same interpretation for $f_n$ and $f_m$
but $k_n$ and $k_m$ will necessarily be mapped to two independent uniform
samplings.

\subsection{Trace models}

We define the structure which will allow us to interpret meta-logic
terms and formulas.

\begin{definition}
  For every protocol $\calp = (\calp_\sfA, \leq)$, a $\calp$-trace model $\TM$ is a tuple
  \(
    (\D_\I, \D_\XT, \leq_\XT, \sigma_\I, \sigma_\XT)
  \) such that:
  \begin{itemize}
    \item $\D_\I \subseteq \mathbb{N}$ is a finite index domain;
    \item $\leq_\XT$ is a a total ordering on
      $\D_\XT := \{ \tinit \} \uplus \{ a(\ov{n}) : a\in\Act_k, \ov{n}\in\D_\I^k
      \}$ which refines the dependency ordering $\leq$ on concrete actions such
      that $\tinit$ is minimal, and such that the sequence of elements of $\D_\XT$ ordered by $\leq$ is an
      abstract trace of $\calp$;
    \item $\sigma_\I : \I \to D_\I$
      and $\sigma_\XT : \XT \to D_\XT$ are mappings that interpret index and
      timestamp variables as elements of their respective domains;
  \end{itemize}
\end{definition}

The total ordering $\leq_\XT$ yields a predecessor function
$\mathsf{pred}_\XT : \D_\XT\to\D_\XT$ which,
by convention, maps $\tinit$ to itself,
and all other elements $v \in \D_\XT$ to the largest $v' \in \D_\XT$ such that
$v' < v$.

When $\TM$ is a trace model and $e$ is an element of its index domain,
$\TM\{i\mapsto e\}$ is the trace model identical to $\TM$ except for
$\sigma_\I$ which is updated to map $i$ to $e$. We similarly define
$\TM\{\tau\mapsto v\}$ when $v$ is an element of the timestamp domain of $\TM$.

\subsection{Interpretation of terms and formulas}

\newcommand{\interp}[3]{(#1)^{#2}_{#3}}

\adrien{We should give an example in the introduction, and explain that we interpret $\exists$ and $\forall$ by, respectively, $\vee$ and $\wedge$}.

We define next the interpretation of terms and formulas. %  when they are
% macro-free, i.e.\ when terms contain no occurrence of $m[\vec{i}]@T$
% and formulas contain no occurrence of $m@T$. This will be used to
% define the interpretation of macros and, finally, of all terms
% and formulas.

\begin{definition}
  Let $\calp$ be a protocol. The interpretation of terms and formulas of the meta-logic in some $\calp$-trace model $\TM$, noted $\interp{t}{\TM}{\calp}$ and $\interp{\phi}{\TM}{\calp}$
  respectively, is given by the clauses of \cref{fig:interpt,fig:interpf,fig:inter-macros}.
\end{definition}

Basically, frames are defined as nested pairs, starting from $0$ and
providing, for each action, the output of the action if it is executable,
and $0$ otherwise. The interpretation of input macros uses function symbols $\att$ to models the arbitrary attacker's computation based on the previous frame, which models its current knowledge. The definition of $\interp{s}{\TM}{\calp}(\vec{f},\sfa[\pvec{e}])$ gives the content of memory cell $s[\vec{f}]$ just after the execution of $\sfa[\pvec{e}]$:
it is obtained by interpreting the update term $u_{\sfa}^{s}$. The semantics of outputs is defined similarly. 


\begin{figure}[t]
  \begin{eqnarray*}
    % Indices
    \interp{i}^\TM &=& \sigma_\I(i)
    \\
    % Timestamps
    \interp{\tau}^\TM &=& \sigma_\XT(\tau) \\
    \interp{\tinit}^\TM &=& \tinit \\
    \interp{\sfa(i_1,\ldots,i_k)}^\TM &=&
      \sfa(\sigma_\I(i_1),\ldots,\sigma_\I(i_k))
      \text{ for all }
      \sfa\in\Act_k \text{ and } i_1,\ldots,i_k\in \I \\
    \interp{\tpred{T}}^\TM &=&
      \mathsf{pred}_\XT(\interp{T}^\TM)
    \\
    % Messages
    (\mathsf{n}[i_1,\ldots,i_k])^\TM &=& \mathsf{n}_{\sigma_\I(i_1),\ldots,\sigma_\I(i_k)}
    \\
    \interp{x}^\TM &=& x
    \\
    \interp{f[i_1,\ldots,i_k](t_1,\ldots,t_n)}^\TM &=&
    f_{\sigma_\I(i_1),\ldots,\sigma_\I(i_k)}\bigl(
      (t_1)^\TM,\ldots,(t_n)^\TM
    \bigr)
    \\
    \interp{m(i_1,\dots,i_k)@T}^\TM &=&
    m^\TM(\interp{i_1}^\TM,\dots,\interp{i_k}^\TM,\interp{T}^\TM)
    \\
    \interp{\myif \phi \mythen t \myelse t'}^\TM &=&
    \myif \interp{\phi}^\TM \mythen \interp{t}^\TM \myelse \interp{t'}^\TM
    \\
    \interp{\myfind \vec{i} \mysuchthat \phi \myin t \myelse t'}^\TM &=&
    \myif \interp{\phi}^{\TM\{\vec{i}\mapsto\vec{e}_1\}}
    \mythen \interp{t}^{\TM\{\vec{i}\mapsto\vec{e}_1\}}
    \myelse \\ & &
    \myif \interp{\phi}^{\TM\{\vec{i}\mapsto\vec{e}_2\}}
    \mythen \interp{t}^{\TM\{\vec{i}\mapsto\vec{e}_2\}}
    \myelse \\ & & \ldots \\ & &
    \myif \interp{\phi}^{\TM\{\vec{i}\mapsto\vec{e}_{n}\}}
    \mythen \interp{t}^{\TM\{\vec{i}\mapsto\vec{e}_{n}\}}
    \myelse \interp{t'}^\TM
%     \\
%     (m[i_1,\ldots,i_k](t_1,\ldots,t_n)@T)^\TM &=&
%     \hat{m}(\sigma_\I(i_1),\ldots,\sigma_\I(i_k),
%       (t_1)^\TM,\ldots,(t_n)^\TM,
%       (T)^\TM)
  \end{eqnarray*}
  where $\vec{e}_1, \ldots, \vec{e}_{n}$ is a complete enumeration
  of $\D_\I^{|\vec{i}|}$ (in lexicographic order).
  \caption{Interpretation of % macro-free
    terms of the meta-logic}
  \label{fig:interpt}
\end{figure}

\begin{figure}[t]
  \begin{eqnarray*}
    (\phi\wedge\phi')^\TM &=& (\phi)^\TM \stackrel{.}{\wedge} (\phi')^\TM
    \quad \text{and similarly for other boolean connectives} \\
    (\forall i:\idx.~ \phi)^\TM &=&
    \stackrel{.}{\wedge}_{e\in D_\I} (\phi)^{\TM[i\mapsto e]} \\
    (\exists i:\idx.~ \phi)^\TM &=&
    \stackrel{.}{\vee}_{e\in D_\I} (\phi)^{\TM[i\mapsto e]} \\
    (\forall \tau:\timestamp.~ \phi)^\TM &=&
    \stackrel{.}{\wedge}_{v\in D_\XT} (\phi)^{\TM[\tau\mapsto v]} \\
    (\exists \tau:\timestamp.~ \phi)^\TM &=&
    \stackrel{.}{\vee}_{v\in D_\XT} (\phi)^{\TM[\tau\mapsto v]} \\
    (i=i')^\TM &=&
    \left\{\begin{array}{ll}
      \mathsf{true} & \text{if } \sigma_\I(i)=\sigma_\I(i') \\
      \mathsf{false} & \text{otherwise}
    \end{array}\right. \\
    (T = T')^\TM &=&
    \left\{\begin{array}{ll}
      \mathsf{true} & \text{if } (T)^\TM = (T')^\TM \\
      \mathsf{false} & \text{otherwise}
    \end{array}\right.
    \quad\text{ and similarly for $\leq$} \\
    (t=t')^\TM &=& (t)^\TM \stackrel{.}{=} (t')^\TM
  \end{eqnarray*}
  \caption{Interpretation of % macro-free
    formulas of the meta-logic}
  \label{fig:interpf}
\end{figure}

\begin{figure}[t]
  % For every action of the protocol:
  % \[
  %   \sfa(\vec{i}).(\phi, \cin, o, \cout,
  %   \{s[\vec{j}] \leftarrow u^{s} \mid s \in \States\})
  %   \in \calp_{\sfA}
  % \]
  % we define the interpretation of the macros, for any $\sfa[\pvec{e}]$ in the index domain of $\TM$:
  \begin{alignat*}{2}
    &\mcond^\TM(\tinit) &\;\;=\;\;& \true
    \\
    &\mcond^\TM(\sfa[\pvec{e}]) &\;\;=\;\;&
    \interp{\phi_{\sfa}
    }^{\TM\{\pvec{i} \mapsto \pvec{e}\}}
    \\
    &\mexec^\TM(\tinit) &\;\;=\;\;& \true
    \\
    &\mexec^\TM(\sfa[\pvec{e}]) &\;\;=\;\;&
    \mcond^\TM(\sfa[\pvec{e}])
    \stackrel{.}{\wedge}
    \mexec^\TM(\tpred{\sfa[\pvec{e}]})
    \\
    &\minp^\TM(\tinit) &\;\;=\;\;& \mframe^\TM(\tinit) = s^\TM(\pvec{f},\tinit)
    = \mout^\TM(\tinit) = 0
    \tag{for every $\vec{f} \in D_\I^p$}
    \\
    &\mframe^\TM(\sfa[\pvec{e}]) &\;\;=\;\;&
    \langle \mframe^\TM(\tpred{\sfa[\pvec{e}]}),
    \myif \mexec^\TM(\sfa[\pvec{e}]) \mythen \mout^\TM(\sfa[\pvec{e}]) \myelse 0
    \rangle
    \\
    &\minp^\TM(\sfa[\pvec{e}]) &\;\;=\;\;&
    \att(\mframe^\TM(\tpred{\sfa[\pvec{e}]}))
    \\
    &\mout^\TM(\sfa[\pvec{e}]) &\;\;=\;\;&
    \interp{
      o_{\sfa}
    }^{\TM\{\pvec{i}\mapsto\pvec{e}\}}
    \\
    &s^\TM(\pvec{f},\sfa[\pvec{e}]) &\;\;=\;\;&
    \interp{
      u_{\sfa}^{s}
    }^{
      \TM\{\pvec{i}\mapsto\pvec{e},\pvec{j}\mapsto\pvec{f}\}}
    \tag{for every $\vec{f} \in D_\I^p$ and $s[\vec{j}] \leftarrow u^{s} \in \sfA$}
  \end{alignat*}
  \textbf{Where:} $\TM = (\D_\I, \D_\XT, \leq_\XT, \sigma_\I, \sigma_\XT)$, $\sfa[\pvec{e}] \in \D_\XT$, and
  \(
    \sfA = \sfa(\vec{i}).(\phi, \cin, o, \cout,
    \{s[\vec{j}] \leftarrow u^{s} \mid s\})
    \in \calp_{\sfA}
  \)
  \caption{Interpretation of macros.}
  \label{fig:inter-macros}
\end{figure}


%%% Local Variables:
%%% mode: latex
%%% TeX-master: "main"
%%% End:


\begin{example}
  The formula
  $\exists i:\idx.~ \tau = \sfa(i)$
  is translated to the disjunction of base logic terms
  $\interp{\tau = \sfa(i)}{\TM\{i\mapsto e\}}{\calp}$ for all $e \in \D_\I$.
  Each disjunct is either $\mathsf{true}$ or $\mathsf{false}$ depending on
  whether $\sigma_\XT(\tau)$ and $\sfa(e)$ are equal.
\end{example}

\begin{proposition}
  For any protocol $\calp$ and $\calp$-trace model $\TM$, the interpretation $\interp{\cdot}{\TM}{\calp}$ of terms and formulas of the meta-logic is well-defined.
\end{proposition}

\begin{proof}
  We only need to prove that the inductive definition of $\interp{\cdot}{\TM}{\calp}$ given in \cref{fig:interpt,fig:interpf,fig:inter-macros} terminates. All cases are trivial, as $\interp{t}{\TM}{\calp}$ (resp.\ $\interp{\phi}{\TM}{\calp}$) are defined using $\interp{\cdot}{\TM}{\calp}$ on strict subterms of $t$ (resp.\ $\phi$), except for the macros case, where termination follows from the well-founded conditions on actions of a protocol.
\end{proof}


%%% Local Variables:
%%% mode: latex
%%% TeX-master: "main"
%%% End:
