\section{Semantics}

We now define trace models, denoted $\TM$, which are the structures needed to
give a meaning to indices and timestamps. More generally, a trace model
also allows to translate any meta-logic formula $\phi$
(over some meta-logic signature $\Sigma$)
to a base logic $\phi^\TM$ (over some base logic signature $\Sigma^\TM$).

\medskip

From now on, we assume a protocol $P$ providing some set of action
symbols $\Act$, using function and name symbols from $\F$ and $\N$,
as well as memory cell symbols from $\States$.
We consider meta-logic terms and formulas over $\Sigma = (\F,\N,\Mac,\Act)$
where $\Mac = \{ \minp, \mout, \mframe \} \cup \States$.

\begin{definition}
  Given a finite set of integers $D$ we define the base logic
  signature $\Sigma^D$ containing exactly:
  \begin{itemize}
    \item a name symbol $n_{e_1,\ldots,e_k}$ for every $n \in \N_k$
       and $e_1,\ldots,e_k \in D$;
    \item a function symbol $f_{e_1,\ldots,e_k}$ of arity $n$ for every
      $f \in \F_k^n$ and $e_1,\ldots,e_k \in D$;
    \item a function $\att_n$ of arity $n$ for every $n\in\mathbb{N}$;
    \item a binary function symbol $\langle \cdot , \cdot \rangle$;
    \item binary function symbols
      $\stackrel{.}{\wedge}$, $\stackrel{.}{\vee}$ and
      $\stackrel{.}{\Rightarrow}$;
    \item a ternary function symbol $\mathsf{if}$;
    \item nullary function symbols $\mathsf{true}$ and $\mathsf{false}$.
  \end{itemize}
  \adrien{Could we remove pairs, and, etc ... from here, and assume that they belong to the inial signature? This would be more compact. }
  Function symbol $\att_n$ will be used to model attacker computations.
  We will use the usual notations for boolean symbols and pairs,
  and we will use the notation $\myif c \mythen t \myelse e$
  for $\mathsf{if}(c,t,e)$ as for meta-logic terms.
\end{definition}

For example, if $f \in \F_1$ (resp. $k \in \N_1$) and we have two distinct
integers $n$ and $m$ in $D$, then $f_n$ and $f_m$ (resp. $k_n$ and $k_m$)
are distinct function (resp. name) symbols in $\Sigma^D$.
A computational model may give the same interpretation for $f_n$ and $f_m$
but $k_n$ and $k_m$ will necessarily be mapped to two independent uniform
samplings.

\subsection{Trace models}

We define the structure which will allow us to interpret meta-logic
terms and formulas.

\begin{definition}
  A trace model is given by:
  \begin{itemize}
    \item a finite index domain $\D_\I \subseteq \mathbb{N}$;
    \item a total ordering $\leq_\XT$ on
      $\D_\XT := \{ \tinit \} \uplus \{ a(\ov{n}) : a\in\Act_k, \ov{n}\in\D_\I^k
      \}$ which refines the dependency ordering $\leq$ on concrete actions such
      that $\tinit$ is minimal, and such that its ordered sequence yields an
      abstract trace of $P$;
    \item mappings $\sigma_\I : \I \to D_\I$
      and $\sigma_\XT : \XT \to D_\XT$ that interpret index and
      timestamp variables as elements of their respective domains;
  \end{itemize}
\end{definition}

The total ordering $\leq_\XT$ yields a predecessor function
$\mathsf{pred}_\XT : \D_\XT\to\D_\XT$ which,
by convention, maps $\tinit$ to itself,
and all other elements $v \in \D_\XT$ to the largest $v' \in \D_\XT$ such that
$v' < v$.

When $\TM$ is a trace model and $e$ is an element of its index domain,
$\TM\{i\mapsto e\}$ is the trace model identical to $\TM$ except for
$\sigma_\I$ which is updated to map $i$ to $e$. We similarly define
$\TM\{\tau\mapsto v\}$ when $v$ is an element of the timestamp domain of $\TM$.

\subsection{Interpretation of terms and formulas}

\newcommand{\interp}[1]{(#1)}

\adrien{We should give an example in the introduction, and explain that we interpret $\exists$ and $\forall$ by, respectively, $\vee$ and $\wedge$}.

We define next the interpretation of terms and formulas when they are
macro-free, i.e.\ when terms contain no occurrence of $m[\vec{i}]@T$
and formulas contain no occurrence of $m@T$. This will be used to
define the interpretation of macros and, finally, of all terms
and formulas.

\begin{definition}
  The interpretation of macro-free terms and formulas of the meta-logic in
  some trace model $\TM$, noted $\interp{t}^\TM$ and $\interp{\phi}^\TM$
  respectively, is given by the clauses of \cref{fig:interpt,fig:interpf}.
\end{definition}

\begin{figure}[t]
  \begin{eqnarray*}
    % Indices
    \interp{i}^\TM &=& \sigma_\I(i)
    \\
    % Timestamps
    \interp{\tau}^\TM &=& \sigma_\XT(\tau) \\
    \interp{\tinit}^\TM &=& \tinit \\
    \interp{\sfa(i_1,\ldots,i_k)}^\TM &=&
      \sfa(\sigma_\I(i_1),\ldots,\sigma_\I(i_k))
      \text{ for all }
      \sfa\in\Act_k \text{ and } i_1,\ldots,i_k\in \I \\
    \interp{\tpred{T}}^\TM &=&
      \mathsf{pred}_\XT(\interp{T}^\TM)
    \\
    % Messages
    (\mathsf{n}[i_1,\ldots,i_k])^\TM &=& \mathsf{n}_{\sigma_\I(i_1),\ldots,\sigma_\I(i_k)}
    \\
    (x)^\TM &=& x
    \\
    (f[i_1,\ldots,i_k](t_1,\ldots,t_n))^\TM &=&
    f_{\sigma_\I(i_1),\ldots,\sigma_\I(i_k)}\bigl(
      (t_1)^\TM,\ldots,(t_n)^\TM
    \bigr)
    \\
    \interp{\myif \phi \mythen t \myelse t'}^\TM &=&
    \myif \interp{\phi}^\TM \mythen \interp{t}^\TM \myelse \interp{t'}^\TM
    \\
    \interp{\myfind \vec{i} \mysuchthat \phi \myin t \myelse t'}^\TM &=&
    \myif \interp{\phi}^{\TM\{\vec{i}\mapsto\vec{e}_1\}}
    \mythen \interp{t}^{\TM\{\vec{i}\mapsto\vec{e}_1\}}
    \myelse \\ & &
    \myif \interp{\phi}^{\TM\{\vec{i}\mapsto\vec{e}_2\}}
    \mythen \interp{t}^{\TM\{\vec{i}\mapsto\vec{e}_2\}}
    \myelse \\ & & \ldots \\ & &
    \myif \interp{\phi}^{\TM\{\vec{i}\mapsto\vec{e}_{n}\}}
    \mythen \interp{t}^{\TM\{\vec{i}\mapsto\vec{e}_{n}\}}
    \myelse \interp{t'}^\TM
%     \\
%     (m[i_1,\ldots,i_k](t_1,\ldots,t_n)@T)^\TM &=&
%     \hat{m}(\sigma_\I(i_1),\ldots,\sigma_\I(i_k),
%       (t_1)^\TM,\ldots,(t_n)^\TM,
%       (T)^\TM)
  \end{eqnarray*}
  where $\vec{e}_1, \ldots, \vec{e}_{n}$ is a complete enumeration
  of $\D_\I^{|\vec{i}|}$ (in lexicographic order).
  \caption{Interpretation of macro-free terms of the meta-logic}
  \label{fig:interpt}
\end{figure}

\begin{figure}
  \begin{eqnarray*}
    (\phi\wedge\phi')^\TM &=& (\phi)^\TM \stackrel{.}{\wedge} (\phi')^\TM
    \quad \text{and similarly for other boolean connectives} \\
    (\forall i:\idx.~ \phi)^\TM &=&
    \stackrel{.}{\wedge}_{e\in D_\I} (\phi)^{\TM[i\mapsto e]} \\
    (\exists i:\idx.~ \phi)^\TM &=&
    \stackrel{.}{\vee}_{e\in D_\I} (\phi)^{\TM[i\mapsto e]} \\
    (\forall \tau:\timestamp.~ \phi)^\TM &=&
    \stackrel{.}{\wedge}_{v\in D_\XT} (\phi)^{\TM[\tau\mapsto v]} \\
    (\exists \tau:\timestamp.~ \phi)^\TM &=&
    \stackrel{.}{\vee}_{v\in D_\XT} (\phi)^{\TM[\tau\mapsto v]} \\
    (i=i')^\TM &=&
    \left\{\begin{array}{ll}
      \mathsf{true} & \text{if } \sigma_\I(i)=\sigma_\I(i') \\
      \mathsf{false} & \text{otherwise}
    \end{array}\right. \\
    (T = T')^\TM &=&
    \left\{\begin{array}{ll}
      \mathsf{true} & \text{if } (T)^\TM = (T')^\TM \\
      \mathsf{false} & \text{otherwise}
    \end{array}\right.
    \quad\text{ and similarly for $\leq$} \\
    (t=t')^\TM &=& (t)^\TM \stackrel{.}{=} (t')^\TM
  \end{eqnarray*}
  \caption{Interpretation of macro-free formulas of the meta-logic}
  \label{fig:interpf}
\end{figure}

\begin{example}
  The formula
  $\exists i:\idx.~ \tau = \sfa(i)$
  is translated to the disjunction of base logic terms
  $\interp{\tau = \sfa(i)}^{\TM\{i\mapsto e\}}$ for all $e \in \D_\I$.
  Each disjunct is either $\mathsf{true}$ or $\mathsf{false}$ depending on
  whether $\sigma_\XT(\tau)$ and $\sfa(e)$ are equal.
\end{example}

\subsection{Macros}
Let $(\alpha'_i)_{0 \leq i < |\D_\XT|}$ be the enumeration of the elements
of $D_\XT$ w.r.t.\ $\leq_\XT$ (in particular, $\alpha'_0 = \tinit$).
For each $k$, let $\alpha_k$ be the representative symbolic action
such that $\alpha'_k = \alpha_k\{\vec{i}_k\mapsto\vec{e}_k\}$.

We define the interpretation $\mcond^\TM$ from $\D_\XT$
to base logic formulas, as follows:
\begin{eqnarray*}
  \mcond^\TM(\tinit) &=& \true \\
  \mcond^\TM(\alpha'_{k+1}) &=&
    \interp{\phi_{\alpha_{k+1}}
    \{x_\beta\mapsto\mout@\beta\}_{\beta\leq \alpha_{k+1}}
    \{x_{\alpha_{k+1}}^{s[\vec{j}]}\mapsto
      s[\vec{j}]@\alpha_{k+1}\}_s
    }^{I\{\vec{i}_{k+1}\mapsto\vec{e}_{k+1}\}}
\end{eqnarray*}
In other words, $\mcond^\TM(\alpha'_{k+1})$ is obtained by taking the
(macro-free) condition $\phi_{\alpha_{k+1}}$ of the symbolic action, and
interpreting it using $I$, mapping indices $\vec{i}_{k+1}$ to their
desired concretization.
The semantics $\mexec^\TM$ of the executability macro is then simply defined as
the conjunction of consecutive conditions:
\begin{eqnarray*}
  \mexec^\TM(\tinit) &=& \true \\
  \mexec^\TM(\alpha'_{k+1}) &=&
    \mcond^\TM(\alpha'_{k+1})
    \stackrel{.}{\wedge}
    \mexec^\TM(\alpha'_k)
\end{eqnarray*}

We then define, for each macro symbol $m\in\Mac_k$,
its interpretation $m^\TM : D_\I^k \times \D_\XT \to T_\Sigma^{D_I}$.
More precisely, we define $m^\TM(\ldots,v)$ by induction on its timestamp
argument $v$ as follows, for all $s \in \States_p$ and $\vec{f} \in D_\I^p$:
\[
    \minp^\TM(\tinit) = \mframe^\TM(\tinit) = s^\TM(\vec{f},\tinit)
    = \mout^\TM(\tinit) = 0\]


\[\begin{array}{rll}
    \mframe^\TM(\alpha'_{k+1}) &=&
    \langle \mframe^\TM(\alpha'_k),
    \myif \mexec^\TM(\alpha'_{k+1}) \mythen \mout^\TM(\alpha'_{k+1}) \myelse 0
    \rangle
    \\
    \minp^\TM(\alpha'_{k+1}) &=&
    \att_{k}(\mframe^\TM(\alpha'_k))
  \\
    s^\TM(\vec{f},\alpha'_{k+1}) &=&
    \interp{
      u_{\alpha_{k+1}}^{s[\vec{j}]}
      \{x_\beta\mapsto\mout@\beta\}_{\beta\leq \alpha_{k+1}}
      \{x^{s[\vec{j}]}\mapsto
        s[\vec{j}]@\alpha_{k+1}\}_s
    }^{
      I\{\vec{i}_k\mapsto\vec{e}_k,\vec{j}\mapsto\vec{f}\}}
  \\
    \mout^\TM(\alpha'_{k+1}) &=&
      \interp{
        o_{\alpha_{k+1}}
        \{x_\beta\mapsto\mout@\beta\}_{\beta\leq \alpha_{k+1}}
        \{x_{\alpha_{k+1}}^{s[\vec{j}]}\mapsto
          s[\vec{j}]@\alpha_{k+1}\}_s
        }^{I\{\vec{i}_{k+1}\mapsto\vec{e}_{k+1}\}}
\end{array}\]

Frames are defined as nested pairs, starting from $0$ and
providing, for each action, the output of the action if it is executable,
and $0$ otherwise. The interpretation of input macros uses function symbols $\att_i$ to model the arbitrary attacker's
computation based on the previous frame, which models its current knowledge.

The definition of $s^\TM(\vec{f},\alpha'_{k+1})$ gives the content of memory
cell $s[\vec{f}]$ just after the execution of $\alpha'_{k+1}$:
it is obtained by interpreting the update term $u_{\alpha_{k+1}}^{s[\vec{j}]}$
of the associated symbolic action $\alpha_{k+1}$ mapping $\vec{i}$
and $\vec{j}$ to their expected concretizations.

The semantics of outputs is defined similarly.

Since meta-logic terms and formulas $o_{\alpha}$, $u_{\alpha}$ and
$\phi_{\alpha}$ may include occurences of message variables
$x_\beta$ and $x^{s[\vec{j}]}$,
we use a term-level substitution for these variables.



    For the other macros, the interpretation at some concrete
    action $\alpha'_i$ is obtained from the symbolic semantics of
    $\alpha_i$ by applying the relevant substitutions (directly
    on terms of the meta-logic when possible\footnote{
      \david{What I wrote implicitly assumes that there is no dependency
      such as e.g.\ $b(i,j) \leq a(i)$ where the dependency introduces
      a new index.}
    }, otherwise by updating the trace model).

Given for all macros $m \in \Mac_k$ the corresponding interpretation function $m^\TM$, we lift the interpretation of macro-free terms and formulas to terms and formulas with:
    \[ \interp{m(i_1,\dots,i_k)@T}^\TM = m^\TM(\interp{i_1}^\TM,\dots,\interp{i_k}^\TM,\interp{T}^\TM) \]




%%% Local Variables:
%%% mode: latex
%%% TeX-master: "main"
%%% End:
