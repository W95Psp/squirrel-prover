\section{Semantics}

We now define meta-interpretations, which are the structures needed to
give a meaning to indices and timestamps. More generally, a trace model
also allows to translate any meta-logic formula $\phi$
(over some meta-logic signature $\Sigma$)
to a base logic $\phi^I$ (over some base logic signature $\Sigma^I$).

\david{I like Adrien's suggestion to rename meta-interpretations to trace
model, but I'm lazy to rename all meta-interpretations $I$ to $T$. Also,
both $T$ and $\T$ have a meaning at this point. Shall we move to trace models
$\mathbb{T}$ and computational models $\mathbb{M}$ (which would avoid the
confusion with the set $\M$ of macro symbols ?}

\medskip

From now on, we assume a protocol $P$ providing some set of action
symbols $\A$, using function and name symbols from $\F$ and $\N$,
as well as memory cell symbols from $\States$.
We consider meta-logic terms and formulas over $\Sigma = (\F,\N,\M,\A)$
where $\M = \{ \minp, \mout, \mframe \} \cup \States$.

\begin{definition}
  Given a finite set of integers $D$ we define the base logic
  signature $\Sigma^D$ containing exactly:
  \begin{itemize}
    \item a name symbol $n_{e_1,\ldots,e_k}$ for every $n \in \N_k$
       and $e_1,\ldots,e_k \in D$;
    \item a function symbol $f_{e_1,\ldots,e_k}$ of arity $n$ for every
      $f \in \F_k^n$ and $e_1,\ldots,e_k \in D$;
    \item a function $g_n$ of arity $n$ for every $n\in\mathbb{N}$;
    \item a binary function symbol $\mathsf{pair}$;
    \item binary function symbols
      $\stackrel{.}{\wedge}$, $\stackrel{.}{\vee}$ and
      $\stackrel{.}{\Rightarrow}$;
    \item a ternary function symbol $\mathsf{if}$;
    \item nullary function symbols $\mathsf{true}$ and $\mathsf{false}$.
  \end{itemize}
  Function symbol $g_n$ will be used, as usual in the BC 
  approach, to model attacker computations.
  We will use the usual notations for boolean symbols and pairs,
  and we will use the notation $\myif c \mythen t \myelse e$
  for $\mathsf{if}(c,t,e)$ as for meta-logic terms.
\end{definition}

For example, if $f \in \F_1$ (resp. $k \in \N_1$) and we have two distinct
integers $n$ and $m$ in $D$, then $f_n$ and $f_m$ (resp. $k_n$ and $k_m$)
are distinct function (resp. name) symbols in $\Sigma^D$.
A computational model may give the same interpretation for $f_n$ and $f_m$
but $k_n$ and $k_m$ will necessarily be mapped to two independent uniform
samplings.

\subsection{Meta-interpretations}

We define the structure which will allow us to interpret meta-logic
terms and formulas.

\begin{definition}
  A meta-interpretation is given by:
  \begin{itemize}
    \item a finite index domain $\D_\I \subseteq \mathbb{N}$;
    \item a total ordering $\leq_\XT$ on
      $\D_\XT := \{ \tinit \} \uplus \{ a(\ov{n}) : a\in\Actions_k,
      \ov{n}\in\D_\I^k \}$ which refines the dependency ordering
      $\leq$ on concrete actions and such that $\tinit$ is minimal;
    \item mappings $\sigma_\I : \I \to D_\I$
      and $\sigma_\XT : \XT \to D_\XT$ that interpret index and
      timestamp variables as elements of their respective domains;
  \end{itemize}
\end{definition}

The total ordering $\leq_\XT$ yields a predecessor function
$\mathsf{pred}_\XT : \D_\XT\to\D_\XT$ which,
by convention, maps $\tinit$ to itself,
and all other elements $v \in \D_\XT$ to the largest $v' \in \D_\XT$ such that
$v' < v$.

When $I$ is a meta-interpretation and $e$ is an element of its index domain,
$I\{i\mapsto e\}$ is the meta-interpretation identical to $I$ except for
$\sigma_\I$ which is updated to map $i$ to $e$. We similarly define
$I\{\tau\mapsto v\}$ when $v$ is an element of the timestamp domain of $I$.

\subsection{Interpretation of terms and formulas}

\newcommand{\interp}[1]{(#1)}

We define next the interpretation of terms and formulas when they are
macro-free, i.e.\ when terms contain no occurrence of $m[\vec{i}]@T$
and formulas contain no occurrence of $m@T$. This will be used to
define the interpretation of macros and, finally, of all terms
and formulas.

\begin{definition}
  The interpretation of macro-free terms and formulas of the meta-logic in 
  some meta-interpretation $I$, noted $\interp{t}^I$ and $\interp{\phi}^I$ 
  respectively, is given by the clauses of \cref{fig:interpt,fig:interpf}.
  In the interpretation of a term built using the $\mathsf{find}$
  construct, the $\vec{e}_1, \ldots, \vec{e}_{k^{|\D_\I|}}$ are a complete
  enumeration of all vectors of $\D_\I^k$, in lexicographic order.
\end{definition}

\begin{figure}[t]
  \begin{eqnarray*}
    % Indices
    \interp{i}^I &=& \sigma_\I(i)
    \\
    % Timestamps
    \interp{\tau}^I &=& \sigma_\XT(\tau) \\
    \interp{\tinit}^I &=& \tinit \\
    \interp{\sfa(i_1,\ldots,i_k)}^I &=&
      \sfa(\sigma_\I(i_1),\ldots,\sigma_\I(i_n))
      \text{ for all }
      \sfa\in\Actions_k \text{ and } e_1,\ldots,e_k\in D_\I \\
    \interp{\tpred{T}}^I &=&
      \mathsf{pred}_\XT(\interp{T}^I)
    \\
    % Messages
    (\mathsf{n}[i_1,\ldots,i_k])^I &=& \mathsf{n}_{\sigma_\I(i_1),\ldots,\sigma_\I(i_k)}
    \\
    (x)^{I} &=& x
    \\
    (f[i_1,\ldots,i_k](t_1,\ldots,t_n))^{I} &=&
    f_{\sigma_\I(i_1),\ldots,\sigma_\I(i_k)}\bigl(
      (t_1)^{I},\ldots,(t_n)^{I}
    \bigr)
    \\
    \interp{\myif \phi \mythen t \myelse t'}^I &=&
    \myif \interp{\phi}^I \mythen \interp{t}^I \myelse \interp{t'}^I
    \\
    \interp{\myfind \vec{i} \mysuchthat \phi \myin t \myelse t'}^I &=&
    \myif \interp{\phi}^{I\{\vec{i}\mapsto\vec{e}_1\}}
    \mythen \interp{t}^{I\{\vec{i}\mapsto\vec{e}_1\}}
    \myelse \\ & &
    \myif \interp{\phi}^{I\{\vec{i}\mapsto\vec{e}_2\}}
    \mythen \interp{t}^{I\{\vec{i}\mapsto\vec{e}_2\}}
    \myelse \\ & & \ldots \\ & &
    \myif \interp{\phi}^{I\{\vec{i}\mapsto\vec{e}_{k^{|\D_\I|}}\}}
    \mythen \interp{t}^{I\{\vec{i}\mapsto\vec{e}_{k^{|\D_\I|}}\}}
    \myelse \interp{t'}^I
%     \\
%     (m[i_1,\ldots,i_k](t_1,\ldots,t_n)@T)^{I} &=&
%     \hat{m}(\sigma_\I(i_1),\ldots,\sigma_\I(i_k),
%       (t_1)^{I},\ldots,(t_n)^{I},
%       (T)^{I})
  \end{eqnarray*}
  \caption{Interpretation of macro-free terms of the meta-logic}
  \label{fig:interpt}
\end{figure}

\begin{figure}
  \begin{eqnarray*}
    (\phi\wedge\phi')^{I} &=& (\phi)^{I} \stackrel{.}{\wedge} (\phi')^{I}
    \quad \text{and similarly for other boolean connectives} \\
    (\forall i:\idx.~ \phi)^{I} &=&
    \stackrel{.}{\wedge}_{e\in D_\I} (\phi)^{I[i\mapsto e]} \\
    (\exists i:\idx.~ \phi)^{I} &=&
    \stackrel{.}{\vee}_{e\in D_\I} (\phi)^{I[i\mapsto e]} \\
    (\forall \tau:\timestamp.~ \phi)^{I} &=&
    \stackrel{.}{\wedge}_{v\in D_\XT} (\phi)^{I[\tau\mapsto v]} \\
    (\exists \tau:\timestamp.~ \phi)^{I} &=&
    \stackrel{.}{\vee}_{v\in D_\XT} (\phi)^{I[\tau\mapsto v]} \\
    (i=i')^{I} &=&
    \left\{\begin{array}{ll}
      \mathsf{true} & \text{if } \sigma_\I(i)=\sigma_\I(i') \\
      \mathsf{false} & \text{otherwise}
    \end{array}\right. \\
    (T = T')^{I} &=&
    \left\{\begin{array}{ll}
      \mathsf{true} & \text{if } (T)^{I} = (T')^{I} \\
      \mathsf{false} & \text{otherwise}
    \end{array}\right.
    \quad\text{ and similarly for $\leq$} \\
    (t=t')^{I} &=& (t)^{I} \stackrel{.}{=} (t')^{I}
  \end{eqnarray*}
  \caption{Interpretation of macro-free formulas of the meta-logic}
  \label{fig:interpf}
\end{figure}

\begin{example}
  The formula
  $\exists i:\idx.~ \tau = \sfa(i)$
  is translated to the disjunction of base logic terms
  $\interp{\tau = \sfa(i)}^{I\{i\mapsto e\}}$ for all $e \in \D_\I$.
  Each disjunct is either $\mathsf{true}$ or $\mathsf{false}$ depending on
  whether $\sigma_\XT(\tau)$ and $\sfa(e)$ are equal.
\end{example}

\subsection{Macros}

Let $(\alpha'_i)_{0 \leq i < |\D_\XT|}$ be the enumeration of the elements
of $D_\XT$ wrt $\leq_\XT$ (in particular, $\alpha'_0 = \tinit$).
For each $k$, let $\alpha_k$ be the representative symbolic action
such that $\alpha'_k = \alpha_k\{\vec{i}_k\mapsto\vec{e}_k\}$.

We define a function $\mcond^I$ from $\D_\XT$
to base logic formulas, as follows:
\begin{eqnarray*}
  \mcond^I(\tinit) &=& \true \\
  \mcond^I(\alpha'_{k+1}) &=&
    \interp{\phi_{\alpha_{k+1}}}^{I\{\vec{i}_{k+1}\mapsto\vec{e}_{k+1}\}}
\end{eqnarray*}
In other words, $\mcond^I(\alpha'_{k+1})$ is obtained by taking the
(macro-free) condition $\phi_{\alpha_{k+1}}$ of the symbolic action, and 
interpreting it using $I$, mapping indices $\vec{i}_{k+1}$ to their
desired concretization.
The semantics $\mexec^I$ of the executability macro is then simply defined as 
the conjunction of consecutive conditions:
\begin{eqnarray*}
  \mexec^I(\tinit) &=& \true \\
  \mexec^I(\alpha'_{k+1}) &=&
    \mcond^I(\alpha'_{k+1})
    \stackrel{.}{\wedge}
    \mexec^I(\alpha'_k)
\end{eqnarray*}

We then define, for each macro symbol $m\in\M_k$,
its interpretation $m^I : D_\I^k \times \D_\XT \to T_\Sigma^{D_I}$.
More precisely, we define $m^I(\ldots,v)$ by induction on its timestamp 
argument $v$ as follows, for all $s \in \States_p$ and $\vec{f} \in D_\I^p$:
$$
    \minp^I(\tinit) = \mframe^I(\tinit) = s^I(\vec{f},\tinit)
    = \mout^I(\tinit) = 0
$$
$$\begin{array}{rll}
    \minp^I(\alpha'_{k+1}) &=&
    g_{k}(\mout^I(\alpha'_1),\ldots,\mout^I(\alpha'_k))
  \\
    \mframe^I(\alpha'_{k+1}) &=&
    \langle \mframe^I(\alpha'_k),
    \myif \mexec^I(\alpha'_{k+1}) \mythen \mout^I(\alpha'_{k+1}) \myelse 0
    \rangle
  \\
    s^I(\vec{f},\alpha'_{k+1}) &=&
    \interp{
      u_{\alpha_{k}}^{s[\vec{j}]}
    }^{
      I\{\vec{i}_k\mapsto\vec{e}_k,\vec{j}\mapsto\vec{f}\}}
  \\
    \mout^I(\alpha'_{k+1}) &=&
      \interp{
        o_{\alpha_i}
        \{x_\beta\mapsto\mout@\beta\}_{\beta\leq \alpha_k}
        \{x_{\alpha_{k+1}}^{s[\vec{j}]}\mapsto 
          s[\vec{j}]@\alpha_{k+1}\}_s
        }^{I\{\vec{i}\mapsto\vec{e}\}}
\end{array}$$

The interpretation of input macros is the simplest:
it uses function symbols $g_i$ to model the arbitrary attacker's
computation based on previous outputs.
Frames are defined as nested pairs, starting from $0$ and
providing, for each action, the output of the action if it is executable,
and $0$ otherwise.

The definition of $s^I(\vec{f},\alpha'_{k+1})$ gives the content of memory
cell $s[\vec{f}]$ just before the execution of $\alpha'_{k+1}$:
it is obtained by interpreting the update term $u_{\alpha_k}^{s[\vec{j}]}$
of the associated symbolic action $\alpha_{k+1}$ mapping $\vec{i}$
and $\vec{j}$ to their expected concretizations.
The semantics of outputs is defined similarly, using a term-level
substitution for the message variables $x_\beta$ and 
$x_{\alpha_k}^{s[\vec{j}]}$.



    For the other macros, the interpretation at some concrete
    action $\alpha'_i$ is obtained from the symbolic semantics of
    $\alpha_i$ by applying the relevant substitutions (directly
    on terms of the meta-logic when possible\footnote{
      \david{What I wrote implicitly assumes that there is no dependency
      such as e.g.\ $b(i,j) \leq a(i)$ where the dependency introduces
      a new index.}
    }, otherwise by updating the meta-interpretation).

\david{
  I have defined $s@T$ to be the content of memory cell $s$
  just before the execution of $T$, and not after it.
  This will have to be fixed one way or another.
}

\subsection{Validity and logical consequence}

We have defined a translation from any meta-logic formula $\phi$
to a base logic boolean term $\interp{\phi}^I$. Intuitively, the
meta-logic formula is true when the associated term is true with
overwhelming probability. This is made formal in the following
definitions.

\begin{definition}
  A formula $\phi$ of the meta-logic is said to be valid when,
  for any meta-interpretation $I$, the base logic formula
  $(\phi)^{I} \sim \true$ is valid.

  More generally, a meta-logic formula $\phi$ is a
  logical consequence of a set $S$ of base logic formulas
  (which might themselves be of the form $\interp{\psi}^I \sim \true$)
  when, for all $I$, $\M$ and $\sigma$ such that
  $\M,\sigma \models \psi$ for all $\psi\in S$,
  we have
  $\M,\sigma \models(\phi)^{I}\sim\mathsf{true}$.
\end{definition}

\begin{remark}
  In the tool, we only allow universal quantification over messages, and
  only allow it at toplevel. The validity of such formulas is the same as
  when the variables are left free.
\end{remark}

In practice, we want to verify that a security property expressed as a
meta-logic formula $\phi$ is satisfied in a class of meta-interpretations
$I$ and computational models $\M$: we will typically restrict computational
models so that cryptographic primitives satisfy some security assumption,
but we may also restrict meta-interpretations e.g.\ to force a condition on
abstract traces. In order to verify security in such a class of models,
we will identify axioms that hold in these models, and verify that our
security property is a logical consequence of these axioms.
