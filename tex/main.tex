\documentclass[a4paper]{article}

\usepackage{lipsum}
\usepackage[normalem]{ulem}
\usepackage{mathtools}
\usepackage{mathpartir}
\usepackage{stmaryrd}
\usepackage{amsthm}
\usepackage{float}
\usepackage{wrapfig}
\usepackage{framed}
\usepackage{xcolor}
\usepackage{textcomp}
\usepackage{xspace}
\usepackage{amsmath,amssymb,amsfonts}
\usepackage[T1]{fontenc}
\usepackage[utf8]{inputenc}
\usepackage{cite}
\usepackage{algorithmic}
\usepackage{graphicx}
\usepackage{placeins}
\usepackage[margin=3cm]{geometry}
\usepackage{multicol}

\usepackage{url}
\usepackage{tikz}
\usepackage{proof}
\usepackage{hyperref}
\usepackage[capitalize,nameinlink,noabbrev]{cleveref}
\usepackage{enumerate}

\setlength{\ULdepth}{0.15em}

\usetikzlibrary{backgrounds}
\usetikzlibrary{intersections}
\usetikzlibrary{scopes}
\usetikzlibrary{calc}
\usetikzlibrary{decorations.pathreplacing}
\usetikzlibrary{decorations.pathmorphing,shapes}
\usetikzlibrary{backgrounds}
\usetikzlibrary{shapes.misc}

\newtheorem{assumption}{Assumption}
\newtheorem*{theorem*}{Theorem}
\newtheorem*{lemma*}{Lemma}
\newtheorem*{proposition*}{Proposition}



\newtheorem{definition}{Definition}
\newtheorem{example}{Example}
\newtheorem{proposition}{Proposition}
\newtheorem{theorem}{Theorem}
\newtheorem{lemma}{Lemma}
\newtheorem{corollary}{Corollary}

\theoremstyle{remark}
\newtheorem{remark}{Remark}

\newcommand{\todo}[1]{\textcolor{red}{(\textbf{TD:} #1)}}
\newcommand{\toadd}[1]{\textcolor{blue}{(\textbf{TA:} #1)}}

\newcommand{\charlie}[1]{\textcolor{cyan}{(\textbf{Charlie:} #1)}}

\newcommand{\true}{\textsf{true}}
\newcommand{\false}{\textsf{false}}

\newcommand{\sem}[1]{[\![#1]\!]}
\newcommand{\pvec}[1]{\vec{#1}\mkern2mu\vphantom{#1}}

\newcommand{\tsuc}[1]{\textsf{suc}(#1)}
\newcommand{\tpred}[1]{\textsf{pred}(#1)}

\newcommand{\tauo}{{\tau_0}}
\newcommand{\taut}{{\tau_1}}
\newcommand{\tautt}{{\tau_2}}
\newcommand{\taui}{{\tau_i}}

\newcommand{\sfr}{\textsf{r}}
\newcommand{\sfx}{\textsf{x}}
\newcommand{\sfy}{\textsf{y}}
\newcommand{\sfz}{\textsf{z}}
\newcommand{\sfb}{\textsf{b}}

\newcommand{\bc}{\textsf{BC}}
\newcommand{\ra}{\rightarrow}
\newcommand{\mdp}{\mathbin{\|}}

\newcommand{\mdef}{\;\;\mathbin{:=}\;\;}

\newcommand{\cexists}{\mathsf{P}}
\newcommand{\cforall}{\mathsf{G}}

\newcommand{\ejudge}[3]{#1 \mdp #2 \vdash #3}
\newcommand{\judge}[2]{#1 \vdash #2}
\newcommand{\subst}[3]{#1\{#2 \mapsto #3\}}
\newcommand{\substs}[2]{#1\{#2\}}

\newcommand{\thetap}{{\theta'}}
\newcommand{\thetapp}{{\theta''}}
\newcommand{\thetao}{{\theta_0}}
\newcommand{\thetat}{{\theta_1}}
\newcommand{\psip}{{\psi'}}
\newcommand{\deltap}{{\delta'}}
\newcommand{\deltai}{{\delta_i}}
\newcommand{\gammap}{{\gamma'}}
\newcommand{\gammao}{{\gamma_0}}
\newcommand{\gammai}{{\gamma_i}}

\newcommand{\valpha}{\pvec{\alpha}}
\newcommand{\valphap}{\pvec{\alpha}'}
\newcommand{\valphao}{{\pvec{\alpha}_0}}
\newcommand{\valphat}{{\pvec{\alpha}_1}}
\newcommand{\valphatt}{{\pvec{\alpha}_2}}
\newcommand{\vbeta}{\pvec{\beta}}
\newcommand{\vbetai}{\pvec{\beta}_i}
\newcommand{\vbetap}{\pvec{\beta}'}
\newcommand{\vgamma}{\pvec{\gamma}}
\newcommand{\vgammai}{\pvec{\gamma}_i}
\newcommand{\vgammap}{\pvec{\gamma}'}
\newcommand{\vdelta}{\pvec{\delta}}
\newcommand{\vdeltai}{\pvec{\delta}_i}
\newcommand{\vdeltap}{\pvec{\delta}'}

\newcommand{\wf}{\textsf{well-formed}}

\newcommand{\env}{\textsf{E}}
\newcommand{\envp}{\textsf{E'}}

\newcommand{\pcnstr}{\mathcal{P}}
\newcommand{\cnstr}{\textsf{C}}

\newcommand{\act}{{\textsf{a}}}
\newcommand{\aset}{\mathcal{S}}

\newcommand{\decl}[3]{\left(#1 , #2 : #3\right)}
\newcommand{\decls}{\mathcal{D}}

\newcommand{\pvars}{\textsf{p-vars}}
\newcommand{\pvtype}[2]{#1 : #2}
\newcommand{\facts}[3]{\pvtype{#1}{#2} \mid #3}

\newcommand{\centail}[2]{#1 \vdash_{\textsf{c}} #2}

% Rules
\newcommand{\defunroll}[1]{\textsf{unroll}($#1$)}
\newcommand{\rraintro}{$\ra$-\textsf{r-intro}}
\newcommand{\grintro}{$\cforall$\textsf{-r-intro}}

\newcommand{\clweaken}{\textsf{c-l-weaken}}
\newcommand{\crstrengthen}{\textsf{c-r-strengthen}}
\newcommand{\cempty}{\textsf{c-empty}}
\newcommand{\cdisj}{\textsf{c-disj}}

\newcommand{\printro}{$\cexists$\textsf{-r-intro}}

\newcommand{\andlintro}{$\wedge$\textsf{-l-intro}}
\newcommand{\orlintro}{$\vee$\textsf{-l-intro}}

\newcommand{\andrintro}{$\wedge$\textsf{-r-intro}}
\newcommand{\orrintro}{$\vee$\textsf{-r-intro}}

\newcommand{\toprintro}{$\top$\textsf{-r-intro}}

\newcommand{\apply}{\textsf{apply}}
\newcommand{\requ}{\textsf{r-equ}}

\newcommand{\induc}{\textsf{induction}}
\newcommand{\induct}{\textsf{induction-}$2$}

%%% Local Variables:
%%% mode: latex
%%% TeX-master: "main"
%%% End:


\begin{document}
\title{Meta-Logic for BC}

\author{David Baelde, Stéphanie Delaune,
  Charlie Jacomme, Adrien Koutsos, Solène Moreau}

\maketitle

\vfill

\tableofcontents

\vfill

\newpage

Bana and Comon have introduced an approach, which they call
\emph{computationally complete symbolic attacker} (CCSA),
to formulate security properties in the computational model as first-order
logic formulas. Roughly, they translate a security property (for a finite set
of traces) as a first-order formula, then they seek to show that this formula
is valid in all \emph{computational models} where attacker computations
are unspecified and primitives satisfy some cryptographic assumptions. The
way this is done is by showing that the formula is a logical consequence of
some axioms, which are sound wrt.\ the considered class of computational
models.

We introduce here a \emph{meta-logic} whose formulas correspond to schemas
of formulas in the \emph{base logic}
--- which we might call the CCSA or BC logic.
We present inference rules which are sound in the same way as before,
i.e.\ any instance of the schema is a formula that is valid in all
computational models subject to the relevant cryptographic assumptions.

We define next the meta-logic which abstractly allows to reason about
executions of a protocol. The precise definition of protocols and their
semantics is not needed for that first step, and is defined only in a
second part.

\section{Logic}

Bana and Comon logic, see Adrien's thesis

\subsection{Syntax}


% We define, for every $n \in \mathbb{N}$, a predicate symbol $\sim_n$ of arity $2n$ representing the equivalence between two vectors of terms of length $n$.
%
% \begin{figure}[h]
%   \[
%     \begin{array}{rcll}
%       t & := & \sfn(i_1 \dots i_n) \mid x \mid \sff(t_1,\dots,t_n) \mid \sfg(t_1,\dots,t_n) & \text{term}
%       \\
%       \atom & := & t_1,\dots,t_n \sim_n t'_1,\dots,t'_n & \text{atomic formula}
%       \\
%       \phi & := & \atom \mid \top \mid \bot \mid \phi \wedge \phi' \mid \phi \vee \phi' \mid \phi \Rightarrow \phi'\mid \neg \phi & \text{formula}
%     \end{array}
%   \]
%   \caption{Syntax for formulas}
%   \label{fig:syntax-formula}
% \end{figure}
%
% \begin{example}
%   Assuming $\F$ contains $\eq, \ok, \error, \funif$ we can build the following formulas.
%   \begin{itemize}
%     \item $\phi_1 := \myif g() \mythen n_0(i) \myelse n_1(i) \sim_1 n(i)$
%     \item $\phi_2 := \myif \eq(g(),n(i)) \mythen \ok \myelse \error \sim_1 \myif \eq(g(),m(i)) \mythen \ok \myelse \error$
%     \item $\phi_3 := \phi_1 \land \phi_2$
%   \end{itemize}
% \end{example}

\subsection{Semantics}

% See Adrien's thesis, section 2.3:
% \begin{itemize}
%   \item formula interpertation
%   \item computational model
%   \item we note $\M \models \phi$ if and only if for every valuation $\sigma$, $\llbracket \phi \rrbracket_\M^\sigma = \True$
% \end{itemize}


\section{Protocols}

% Without entering into the details of the protocol semantics, we can already
% restrict the set of meta-interpretations of interest, to justify some of the
% rules that we use. We would alternatively impose these conditions from the
% beginning in the definition of meta-interpretations.

There exists a natural translation from applied pi-calculus to this notion of
actions, reminescent of the translation inside Horn Clauses performed by
Proverif, or the translation inside Multi Set Rewritting rules performed by
Sapic (a Tamarin extension).

\bigskip

We  model a protocol as a set of possible actions available to the
attacker. An action models a basic step of the protocol, where
the attacker provides some input, some condition is checked, some
state updates are performed, and finally an output is emitted.

In this section we assume a fixed set of action symbols $\Actions$,
as well as a fixed set of channel names $\C$.
We assume that macro symbols include $\minp$ and $\mout$,
both of index and term arities zero.
We also assume that all other macro symbols are of term arities zero.
We call them \emph{state macros} as they will be used to model memory cells.

\begin{definition}
  \label{def:action}
A \emph{symbolic action} $\alpha = \sfa(i_1,\dots,i_n)$, where $\sfa \in \Actions$ and $i_1,\dots,i_n \in \I$, is defined by:
\begin{itemize}
  \item a meta-logic formula $\phi_{\alpha}$ (intuitively, the condition of this action)
  \item a meta-logic term $o_{\alpha}$ (intuitively, the output of this action)
  \item for each state macro symbol $s$, a meta-logic term $u_{\alpha}^{s}$ (intuitively, the update of the memory cell $s$ for this action)
  \item two channel names $\cin_\alpha$ and $\cout_\alpha$ (intuitively, the channels used for the input and the output of this action)
\end{itemize}
where $\phi_{\alpha}$, $u_{\alpha}^{s}$ and $o_{\alpha}$ must not contain any
macro term and the only free variables that can occur in them are
$x_\alpha$ (intuitively, the input of this action)
and $x^s_\alpha$ where $s$ is a state macro symbol (intuitively, the value
of the memory cell at the time of this action).
% ???
% Names appearing in $\phi_{\alpha}$ and $o_{\alpha}$ are of the form $\sfn(i_1,\dots,i_k)$ with $k \leq n$.
\end{definition}

\begin{definition}
A \emph{concrete action} is an instanciation of a \emph{symbolic action} by giving integer values to index variables.
\end{definition}

\begin{definition}
  \label{def:proto}
  A \emph{protocol} is defined by:
  \begin{itemize}
    \item a finite set of \emph{symbolic actions} with distinct action symbols,
    \item a transitive \emph{sequential dependency} relation $\before$ on \emph{symbolic actions},
    \item a \emph{conflict} relation $\conflict$ on \emph{symbolic actions}.
  \end{itemize}
  Intuitively, $\sfa_1(i)\before\sfa_2(i,j)$ means that, for all $n_i,n_j \in \mathbb{N}$, $\sfa_1(n_i)$ must happen before $\sfa_2(n_i,n_j)$ in any execution
  and $\sfa_1(i,j)\conflict\sfa_2(i,j)$ means that, for all $n_i,n_j \in \mathbb{N}$, $\sfa_1(n_i,n_j)$ and $\sfa_2(n_i,n_j)$ cannot both occur in any execution.
\end{definition}

In our tool, actions are compiled from an applied pi-calculus process.
Sequential dependencies are induced by the syntactic sequences in that
process, and conflicts are induced by conditionals.

\begin{definition}
  \label{def:trace}
  Given a protocol $P$ according to \Cref{def:proto}, an \emph{abstract trace}\footnote{Abstract in the sense that we do not know if the trace is executable.} is a sequence $\sfa_1(\vect {i_1}) \dots \sfa_n(\vect {i_n})$ made of concrete actions such that:
  \begin{itemize}
    \item for all $1 \leq i,j \leq n$, if $\sfa_i = \sfa_j$ then $\vect {i_i} \neq \vect {i_j}$;
    \item for all $1 \leq i < j \leq n$, it is not the case that $\sfa_i(\vect {i_i}) \conflict \sfa_j(\vect {i_j})$;
    \item for all $1 \leq i \leq n$, for all concrete action $\beta$ such that $\beta \before \sfa_i(\vect {i_i})$, there exists $1 \leq j < i$ such that $\beta = \sfa_j(\vect {i_j})$.
  \end{itemize}
\end{definition}

\paragraph{Bi-processes}

When reasoning on indistinguishability of a protocol, we may look at equivalences
between two processes given as a bi-process, i.e. a system with a single set of
symbolic actions, each symbolic action describing an execution step of both the
left and right processes, using $\diff{\_}{\_}$ terms as usual.

In that case, the definition of a symbolic action differs from \Cref{def:action}
in the following way:
\begin{itemize}
  \item $\phi_{\alpha}$ is a meta-logic \emph{bi-formula}
  \item $o_{\alpha}$ and $u_{\alpha}^{\sfs}$ are meta-logic \emph{bi-terms}
\end{itemize}

We define in \Cref{fig:bi-terms} how we extend the syntax in \Cref{fig:terms}
to define meta-logic bi-terms.
Bi-formulas are obtained from the syntax in \Cref{fig:formulas} by allowing
atomic propositions over bi-terms.

\begin{figure}[h]
  \[
  \begin{array}{rcll}
    tt & := & t &\text{term} \\
    & \mid & \diff{t_1}{t_2} &\text{bi-term}\\
    & \mid & F(tt_1,\dots,tt_n) &\text{function application over bi-terms}\\
    & \mid & M(tt_1,\ldots,tt_n)@T &\text{macro application over bi-terms}\\
  \end{array}
  \]
  \caption{Syntax of meta-logic bi-terms}\label{fig:bi-terms}
\end{figure}

We use the notation $(\_)^L$ and $(\_)^R$ to denote the left and right projections
(defined as usual), for example: $(\phi_{\alpha})^L$ and $(\phi_{\alpha})^R$,
$(o_{\alpha})^L$ and $(o_{\alpha})^R$.


\begin{example}[Basic Hash]
  \label{ex:basic-hash-bi-process}
  We introduce the bi-process $P_{BH}$ defined by:
  \begin{itemize}
    \item the set of symbolic actions $\{\T(i,j),\R_1(k,i,j),\R_2(k,i,j)\}$ where,
    \begin{itemize}
      \item action $\alpha_1 := \T(i,j)$
        \begin{itemize}
          \item $\phi_{\alpha_1} := \true$
          \item $o_{\alpha_1} := \langle n(i,j), \h(\diff{k(i)}{k'(i,j)},n(i,j) \rangle$
        \end{itemize}
      \item action $\alpha_2 := \R_1(k,i,j)$
        \begin{itemize}
          \item $\phi_{\alpha_2} := \exists i,j, \snd(\g(x_{\alpha_2})) =  \h(\diff{k(i)}{k'(i,j)},\fst(\g(x_{\alpha_2}))$
          \item $o_{\alpha_2} := \ok$
        \end{itemize}
      \item action $\alpha_3 := \R_2(k,i,j)$
        \begin{itemize}
          \item $\phi_{\alpha_3} := \neg (\exists i,j, \snd(\g(x_{\alpha_2})) =  \h(\diff{k(i)}{k'(i,j)},\fst(\g(x_{\alpha_2})))$
          \item $o_{\alpha_3} := \error$
        \end{itemize}
    \end{itemize}
    \item a conflict relation $\conflict$ such that $\R_1(k,i,j)\conflict\R_2(k,i,j)$.
  \end{itemize}

\noindent
Some traces for this bi-process are:
  \begin{itemize}
    \item $\T(1,1),\R_1(1,1,1)$
    \item $\T(1,1).\R_1(1,1,1).\T(1,2).\R_1(2,1,2)$
    \item $\T(1,1).\R_1(1,1,1).\T(2,1).\R_1(2,2,1)$
  \end{itemize}
\end{example}

\bigskip
\noindent
\begin{remark}
  We could get rid of the conflict relationship and still be able to
  model conditionals: we would have more symbolic traces but, when conditions
  are taken into account, the incompatible conditions of actions corresponding
  to different branches of a conditional would allow to discard
  the unwanted branches. For that reason, we should not need a tactic
  relying on the conflict relationship in the prover.
\end{remark}

\begin{remark}
  The structure that we have assumed on actions is restrictive and
  does not allow to model e.g. phases. It does not matter because we can
  impose other conditions on traces later on as axioms (similar to the
  restrictions of Tamarin).
\end{remark}


\section{Concrete meta-logic}

\emph{We are not defining a new logic, but identifying axioms (or inference
rules, which would be closer to the tool) that hold in all
meta-interpretations corresponding to protocol executions.}

Now that the semantics of formulas is defined given a specific protocol, we
can define a logic allowing to prove such formulas. We will reason using
classical sequent calculs, extended with some axioms obtained form the
constraints based on the protocol, or computational axioms made about the
primitives. We will use implicitely the axioms about equality.

Given a protocol $P$, we define a set of path constraints axioms $C_P$ with the formulas:
\begin{itemize}
\item for any action $a(id_1,\dots,id_n) \in \Actions(P)$, when $a'(id_1,\dots,id_k) = \pre (a(i_1,\dots,i_n))$,
  \[\forall i_1, \dots, i_n:\idx, \tau:\timestamp, a(i_1,\dots,i_n)@\tau \Rightarrow \exists \tau', \tau'< \tau \wedge a'(i_1,\dots,i_k)@\tau' \qquad (i) \]
\item for any action $a(id_1,\dots,id_n) \in \Actions(P)$ such that $\pre (a(id_1,\dots,id_n)) =
  \oplus$, for any $a'(id_1,\dots,id_k,id'_1,\dots, id'_l) \in \suc^*( \bro (a(id_1,\dots,id_n)))$,
  \[\forall i_1, \dots, i_n, i'_1,\dots, i'_l:\idx, \tau,\tau':\timestamp, a(i_1,\dots,i_n)@\tau \wedge a'(id_1,\dots,id_k,id'_1,\dots, id'_l)@\tau' \Rightarrow \false   \qquad (ii) \]
\item for any action $a(id_1,\dots,id_n) \in \Actions(P)$,
  \[\forall i_1, \dots, i_n, :\idx, \tau:\timestamp, a(i_1,\dots,i_n)@\tau \wedge a(id_1,\dots,id_n)@\tau' \Rightarrow \tau=\tau'  \qquad (iii)\]
\item for any action $a(id_1,\dots,id_n) \in \Actions(P)$ with output term $o(id_1,\dots,id_n)$ and condition $\phi(id_1,\dots,id_n)$,
  \[\forall i_1, \dots, i_n, :\idx, \tau,\tau':\timestamp, a(i_1,\dots,i_n)@\tau \Rightarrow \mout@\tau = o(i_1,\dots,i_n) \wedge \phi(id_1,\dots,id_n)   \qquad (iv)\]
\end{itemize}

We have that $ P\models C_P$ by definitions of the traces of $P$.Intuitively $(i)$ captures the fact that an action can only happen if the previous actions in the protocol have happened before, $(ii)$ captures the fact that the protocol sometimes enforce a choice between two actions and both cannot then be in a trace, $(iii)$ captures the uniqueness of an action, and $(iv)$ simply enforces the equality given by an action.

\charlie{need to define the axiom for the states, but I'm lazy right now}


We also consider a set of computational sounds axioms $A_P$, which is based on classical axioms from the \BC model, which may contain EQ-INDEP, EUF-CMA, PRF, \dots.

Finally, given a protocol $P$, we obtain a set of axioms $C_P$, and we may choose a set of computationaly sound axioms $A_P$.
Then, we can prove statements of the form $ P \models \phi$, by reasoning using the rules of the sequence calculs, and proving that $A_P,C_P \vdash \phi$.

\newpage

\section{Old stuff}



\paragraph{Facts}
Facts are schemas of $\bc$ formulas indexed by path constraints:
\[
  \phi \mdef
  \cforall_\theta\, \valpha.\;
  \psi
  \ra
  \bigvee_{i}
  \cexists_\gammai \vbetai. \psi_i
\]

\paragraph{Environments}
\[
  \env \mdef
  \aset \mid
  \pcnstr \mid
  \decls
  \qquad
  \text{ where }
  \qquad
  \decls \mdef
  \left\{
    \decl{t_\sfx}
    {t_\epsilon,(t_\sfy^\act)_{\act \in \aset}}
  \right\}
\]
where:
\begin{itemize}
\item $\aset$ is the set of symbolic actions.
\item $\pcnstr$ is a set of path constraints.
\item $\decls$ is a set of inductive term declarations.
\end{itemize}
\paragraph{Judgments}
\[
  \ejudge{\env}{\facts{\valpha}{\theta}{\Gamma}}{\phi}
\]
where:
\begin{itemize}
\item $\env$ is the environment.
\item $\pvtype{\valpha}{\theta}$ declares a set of constrained path variables.
\item $\Gamma$ is a set of facts.
\item $\phi$ is the goal.
\end{itemize}

\begin{remark}
  $\env$ will be invariant during derivations. We omit it when there is no ambiguity.
\end{remark}

\paragraph{Standard Introduction Rules}
\begin{mathpar}
  \inferrule[\rraintro]{
    \judge{\facts{\valpha}{\theta}{
        \Gamma \cup \{\psi_0\}}}
    {
      \cexists_\gammap \vdelta. \psi
    }
  }{
    \judge{\facts{\valpha}{\theta}{\Gamma}}
    {
      \psi_0
      \ra
      \cexists_\gammap \vdelta. \psi
    }
  }

  \inferrule[\andlintro]{
    \judge{\facts{\valpha}{\theta}{\Gamma \cup \{ \psi \} \cup \{\psi'\}}}
    {\phi}
  }{
    \judge{\facts{\valpha}{\theta}{\Gamma \cup \{ \psi \wedge \psi'\}}}
    {\phi}
  }

  \inferrule[\orlintro]{
    \judge{\facts{\valpha}{\theta}{\Gamma \cup \{ \psi \}}}
    {\phi}\\
    \judge{\facts{\valpha}{\theta}{\Gamma \cup \{\psi'\}}}
    {\phi}
  }{
    \judge{\facts{\valpha}{\theta}{\Gamma \cup \{ \psi \vee \psi'\}}}
    {\phi}
  }

  \inferrule[\andrintro]{
    \judge{\facts{\valpha}{\theta}{\Gamma}}{\psi}\\
    \judge{\facts{\valpha}{\theta}{\Gamma}}{\psi'}
  }{
    \judge{\facts{\valpha}{\theta}{\Gamma}}{\psi \wedge \psi'}
  }

  \inferrule[\orrintro-1]{
    \judge{\facts{\valpha}{\theta}{\Gamma}}{\psi}
  }{
    \judge{\facts{\valpha}{\theta}{\Gamma}}{\psi \vee \psi'}
  }

  \inferrule[\orrintro-2]{
    \judge{\facts{\valpha}{\theta}{\Gamma}}{\psi'}
  }{
    \judge{\facts{\valpha}{\theta}{\Gamma}}{\psi \vee \psi'}
  }

  \inferrule[\toprintro]{
  }{
    \judge{\facts{\valpha}{\theta}{\Gamma}}{\top}
  }
\end{mathpar}

\paragraph{Other Rules}
\begin{mathpar}
  \inferrule[\absurd]{
    \bot \in \Gamma
  }{
    \judge{\facts{\valpha}{\theta}{\Gamma}}{\phi}
  }

  \inferrule[\axiom]{
    \psi \in \Gamma
  }{
    \judge{\facts{\valpha}{\theta}{\Gamma}}{\psi}
  }

\end{mathpar}

\paragraph{Modal Introduction Rules}
\begin{mathpar}
  \inferrule[\grintro]{
    \judge{\facts{\valpha,\vbeta}{\theta\wedge\gamma}{\Gamma}}
    { \phi }\\
    \valpha \cap \vbeta = \emptyset
  }{
    \judge{\facts{\valpha}{\theta}{\Gamma}}
    {
      \cforall_\gamma\, \vbeta.\;
      \phi
    }
  }

  \inferrule[\printro]{
    \nu : \vbeta \mapsto \valpha\\
    \judge{\facts{\valpha}{\theta}{\Gamma}}{\psi\nu}\\
    \centail{\theta}{\gamma\nu}
  }{
    \judge{\facts{\valpha}{\theta}{\Gamma}}{\cexists_\gamma \vbeta. \psi}
  }
\end{mathpar}

\paragraph{Path Constraints Rules}
\begin{mathpar}
  \inferrule[\clweaken]{
    \judge
    {\facts{\valpha}{\gamma}{\Gamma}}{\phi}\\
    \centail{\theta}{\gamma}
  }{
    \judge{\facts{\valpha}{\theta}{\Gamma}}{\phi}
  }

  \inferrule[\crstrengthen]{
    \judge{\facts{\valpha}{\theta}{\Gamma}}{\cexists_\delta \vbeta. \phi}\\
    \centail{\delta}{\gamma}
  }{
    \judge{\facts{\valpha}{\theta}{\Gamma}}{\cexists_\gamma \vbeta. \phi}
  }

  \inferrule[\cempty]{
    \centail{\theta}{\bot}
  }{
    \judge{\facts{\valpha}{\theta}{\Gamma}}{
      \phi
    }
  }

  \inferrule[\cdisj]{
    \judge{\facts{\valpha}{\theta}{\Gamma}}{\phi}\\
    \judge{\facts{\valpha}{\gamma}{\Gamma}}{\phi}
  }{
    \judge{\facts{\valpha}{\theta \vee \gamma}{\Gamma}}{\phi}
  }
\end{mathpar}

\paragraph{Rules}
\begin{mathpar}
  \inferrule[\defunroll{t_\tau}]{
    \decl{t_\sfx}
    {t_\epsilon,(t_\sfy^\act)_{\act \in \aset}} \in \decls\\
    \tau \in \valpha\\
    \tauo \not \in \valpha\\\\
    \judge
    {\facts{\valpha}
      {\theta\wedge\tau = \epsilon}
      {\Gamma \wedge t_\tau = t_\epsilon}}
    {\phi}\\
    \left(
      \judge
      {\facts{\valpha,\tauo}
        {\theta\wedge\tauo = \tpred{\tau}\wedge \tauo = \act}
        {\Gamma \wedge t_\tau = t_\tauo^\act}}
      {\phi}
    \right)_{\act \in \aset}
  }{
    \judge{\facts{\valpha}{\theta}{\Gamma}}{\phi}
  }

  \inferrule[\requ]{
    \judge{\facts{\valpha}{\theta}{\Gamma}}{\psi[s]}\\
    \judge{\facts{\valpha}
      {\theta}{\Gamma}}
    {s = t}
  }{
    \judge{\facts{\valpha}{\theta}{\Gamma}}{
      \psi[t]
    }
  }

  \inferrule[\papply]{
    \cexists_\gamma \vbeta. \psi
    \in \Gamma\\
    \valpha \cap \vbeta = \emptyset\\
    \judge{\facts{\valpha, \vbeta}{\theta \wedge \gamma}
      {\Gamma \cup \{\psi\}}}{\phi}
  }{
    \judge{\facts{\valpha}{\theta}{\Gamma}}{\phi}
  }

  \inferrule[\apply]{
    (\cforall_\gamma\, \vbeta.\;
    \psi
    \ra
    \phi_0)
    \in \Gamma\\
    \nu : \vbeta \mapsto \valpha\\
    \centail{\theta}{\gamma\nu}\\
    \judge{\facts{\valpha}{\theta}{\Gamma}}{\psi\nu}\\
    \judge{\facts{\valpha}{\theta}
      {\Gamma \cup \{\phi_0\nu\}}}{\phi}
  }{\judge{\facts{\valpha}{\theta}{\Gamma}}{\phi}}

  \inferrule[\induc]{
    \phi' \equiv
    \cforall_{\gammao} \tauo.\phi\\
    \gammao \equiv
    \subst{\gamma}{\tau}{\tauo}\wedge\tauo < \tau\\
    \tau,\tauo \not \in \valpha\\
    \judge
    {\facts{\valpha,\tau}{\theta\wedge\gamma}{\Gamma\cup\{\phi'\}}}
    {\phi}
  }{
    \judge
    {\facts{\valpha}{\theta}{\Gamma}}
    {\cforall_\gamma \tau.\phi}
  }
\end{mathpar}

\paragraph{Superpositions of $\clweaken$ or $\crstrengthen$ with other rules}
\begin{itemize}
\item $\mrule{\clweaken}$ with $\mrule{\printro}$:
  \[
    \inferrule[\printrogen]{
      \nu : \vbeta \mapsto \valpha\\
      \judge{\facts{\valpha}{\theta \wedge (\gamma\nu)}{\Gamma}}{\psi\nu}\\
      \judge{\facts{\valpha}{\theta \wedge \neg(\gamma\nu)}{\Gamma}}{\bot}
    }{
      \judge{\facts{\valpha}{\theta}{\Gamma}}{\cexists_\gamma \vbeta. \psi}
    }
  \]

\item $\mrule{\crstrengthen}$ with $\mrule{\apply}$ and $\mrule{\absurd}$:
  \[
    \inferrule[\botapply]{
      (
      \cforall_\gamma\, \vbeta.\;
      \psi
      \ra
      \bot
      ) \in \Gamma\\
      \nu : \vbeta \mapsto \valpha\\
      \judge{\facts{\valpha}{\theta \wedge (\gamma\nu)}{\Gamma}}{\psi\nu}\\
      \judge{\facts{\valpha}{\theta \wedge \neg(\gamma\nu)}{\Gamma}}{\phi}
    }{\judge{\facts{\valpha}{\theta}{\Gamma}}{\phi}}
  \]

\item $\mrule{\crstrengthen}$ with $\mrule{\apply}$ (general case):
  \[
    \inferrule[\botapply]{
      (
      \cforall_\gamma\, \vbeta.\;
      \psi
      \ra
      \phio
      ) \in \Gamma\\
      \nu : \vbeta \mapsto \valpha\\
      \judge{\facts{\valpha}{\theta \wedge (\gamma\nu)}{\Gamma}}{\psi\nu}\\
      \judge{\facts{\valpha}{\theta \wedge (\gamma\nu)}
        {\Gamma \cup \{ \phio\nu\}}}
      {\phi}\\
      \judge{\facts{\valpha}{\theta \wedge \neg(\gamma\nu)}{\Gamma}}{\phi}
    }{\judge{\facts{\valpha}{\theta}{\Gamma}}{\phi}}
  \]

\end{itemize}


\end{document}

%%% Local Variables:
%%% mode: latex
%%% TeX-master: t
%%% End:
