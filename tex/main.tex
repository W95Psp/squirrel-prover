\documentclass[a4paper]{article}

\usepackage{lipsum}
\usepackage[normalem]{ulem}
\usepackage{mathtools}
\usepackage{mathpartir}
\usepackage{stmaryrd}
\usepackage{amsthm}
\usepackage{float}
\usepackage{wrapfig}
\usepackage{framed}
\usepackage{xcolor}
\usepackage{textcomp}
\usepackage{xspace}
\usepackage{amsmath,amssymb,amsfonts}
\usepackage[T1]{fontenc}
\usepackage[utf8]{inputenc}
\usepackage{cite}
\usepackage{algorithmic}
\usepackage{graphicx}
\usepackage{placeins}
\usepackage[margin=3cm]{geometry}
\usepackage{multicol}

\usepackage{url}
\usepackage{tikz}
\usepackage{proof}
\usepackage{hyperref}
\usepackage[capitalize,nameinlink,noabbrev]{cleveref}
\usepackage{enumerate}

\usepackage{msc5}

\setlength{\ULdepth}{0.15em}

\usetikzlibrary{backgrounds}
\usetikzlibrary{intersections}
\usetikzlibrary{scopes}
\usetikzlibrary{calc}
\usetikzlibrary{decorations.pathreplacing}
\usetikzlibrary{decorations.pathmorphing,shapes}
\usetikzlibrary{backgrounds}
\usetikzlibrary{shapes.misc}

\newtheorem{assumption}{Assumption}
\newtheorem*{theorem*}{Theorem}
\newtheorem*{lemma*}{Lemma}
\newtheorem*{proposition*}{Proposition}



\newtheorem{definition}{Definition}
\newtheorem{example}{Example}
\newtheorem{proposition}{Proposition}
\newtheorem{theorem}{Theorem}
\newtheorem{lemma}{Lemma}
\newtheorem{corollary}{Corollary}

\theoremstyle{remark}
\newtheorem{remark}{Remark}

\newcommand{\adrien}[1]{\textcolor{blue}{(\textbf{Adrien:} #1)}}
\newcommand{\charlie}[1]{\textcolor{cyan}{(\textbf{Charlie:} #1)}}

\newcommand{\true}{\textsf{true}}
\newcommand{\false}{\textsf{false}}

\newcommand{\sem}[1]{[\![#1]\!]}
\newcommand{\pvec}[1]{\vec{#1}\mkern2mu\vphantom{#1}}

\newcommand{\tsuc}[1]{\textsf{suc}(#1)}
\newcommand{\tpred}[1]{\textsf{pred}(#1)}

\newcommand{\tauo}{{\tau_0}}
\newcommand{\taut}{{\tau_1}}
\newcommand{\tautt}{{\tau_2}}
\newcommand{\taui}{{\tau_i}}

\newcommand{\sfr}{\textsf{r}}
\newcommand{\sfx}{\textsf{x}}
\newcommand{\sfy}{\textsf{y}}
\newcommand{\sfz}{\textsf{z}}
\newcommand{\sfb}{\textsf{b}}

\newcommand{\bc}{\textsf{BC}}
\newcommand{\ra}{\rightarrow}
\newcommand{\mdp}{\mathbin{\|}}

\newcommand{\mdef}{\;\;\mathbin{:=}\;\;}

\newcommand{\cexists}{\mathsf{P}}
\newcommand{\cforall}{\mathsf{G}}

\newcommand{\ejudge}[3]{#1 \mdp #2 \vdash #3}
\newcommand{\judge}[2]{#1 \vdash #2}
\newcommand{\subst}[3]{#1\{#2 \mapsto #3\}}
\newcommand{\substs}[2]{#1\{#2\}}

\newcommand{\thetap}{{\theta'}}
\newcommand{\thetapp}{{\theta''}}
\newcommand{\thetao}{{\theta_0}}
\newcommand{\thetat}{{\theta_1}}
\newcommand{\psip}{{\psi'}}
\newcommand{\deltap}{{\delta'}}
\newcommand{\deltai}{{\delta_i}}
\newcommand{\gammap}{{\gamma'}}
\newcommand{\gammao}{{\gamma_0}}
\newcommand{\gammai}{{\gamma_i}}

\newcommand{\valpha}{\pvec{\alpha}}
\newcommand{\valphap}{\pvec{\alpha}'}
\newcommand{\valphao}{{\pvec{\alpha}_0}}
\newcommand{\valphat}{{\pvec{\alpha}_1}}
\newcommand{\valphatt}{{\pvec{\alpha}_2}}
\newcommand{\vbeta}{\pvec{\beta}}
\newcommand{\vbetai}{\pvec{\beta}_i}
\newcommand{\vbetap}{\pvec{\beta}'}
\newcommand{\vgamma}{\pvec{\gamma}}
\newcommand{\vgammai}{\pvec{\gamma}_i}
\newcommand{\vgammap}{\pvec{\gamma}'}
\newcommand{\vdelta}{\pvec{\delta}}
\newcommand{\vdeltai}{\pvec{\delta}_i}
\newcommand{\vdeltap}{\pvec{\delta}'}

\newcommand{\wf}{\textsf{well-formed}}

\newcommand{\env}{\textsf{E}}
\newcommand{\envp}{\textsf{E'}}

\newcommand{\pcnstr}{\mathcal{P}}
\newcommand{\cnstr}{\textsf{C}}

\newcommand{\act}{{\textsf{a}}}
\newcommand{\aset}{\mathcal{S}}

\newcommand{\decl}[3]{\left(#1 , #2 : #3\right)}
\newcommand{\decls}{\mathcal{D}}

\newcommand{\pvars}{\textsf{p-vars}}
\newcommand{\pvtype}[2]{#1 : #2}
\newcommand{\facts}[3]{\pvtype{#1}{#2} \mid #3}

\newcommand{\centail}[2]{#1 \vdash_{\textsf{c}} #2}

% Rules
\newcommand{\defunroll}[1]{\textsf{unroll}($#1$)}
\newcommand{\rraintro}{$\ra$-\textsf{r-intro}}
\newcommand{\grintro}{$\cforall$\textsf{-r-intro}}

\newcommand{\clweaken}{\textsf{c-l-weaken}}
\newcommand{\crstrengthen}{\textsf{c-r-strengthen}}
\newcommand{\cempty}{\textsf{c-empty}}
\newcommand{\cdisj}{\textsf{c-disj}}

\newcommand{\printro}{$\cexists$\textsf{-r-intro}}

\newcommand{\andlintro}{$\wedge$\textsf{-l-intro}}
\newcommand{\orlintro}{$\vee$\textsf{-l-intro}}

\newcommand{\andrintro}{$\wedge$\textsf{-r-intro}}
\newcommand{\orrintro}{$\vee$\textsf{-r-intro}}

\newcommand{\toprintro}{$\top$\textsf{-r-intro}}

\newcommand{\apply}{\textsf{apply}}
\newcommand{\requ}{\textsf{r-equ}}

\newcommand{\induc}{\textsf{induction}}
\newcommand{\induct}{\textsf{induction-}$2$}

%%% Local Variables:
%%% mode: latex
%%% TeX-master: "main"
%%% End:


\begin{document}
\title{Meta-Logic for BC}

\author{David Baelde, Stéphanie Delaune,
  Charlie Jacomme, Adrien Koutsos, Solène Moreau}

\maketitle

\vfill

\tableofcontents

\vfill

\newpage

Bana and Comon have introduced an approach, which they call
\emph{computationally complete symbolic attacker} (CCSA),
to formulate security properties in the computational model as first-order
logic formulas. Roughly, they translate a security property (for a finite set
of traces) as a first-order formula, then they seek to show that this formula
is valid in all \emph{computational models} where attacker computations
are unspecified and primitives satisfy some cryptographic assumptions. The
way this is done is by showing that the formula is a logical consequence of
some axioms, which are sound w.r.t.\ the considered class of computational
models.

We introduce here a \emph{meta-logic} whose formulas correspond to schemas
of formulas in the \emph{base logic}
--- which we might call the CCSA or BC logic.
We present inference rules which are sound in the same way as before,
i.e.\ any instance of the schema is a formula that is valid in all
computational models subject to the relevant cryptographic assumptions.

We define next the meta-logic which abstractly allows to reason about
executions of a protocol. The precise definition of protocols and their
semantics is not needed for that first step, and is defined only in a
second part.

\section{Logic}

Bana and Comon logic, see Adrien's thesis

\subsection{Syntax}


% We define, for every $n \in \mathbb{N}$, a predicate symbol $\sim_n$ of arity $2n$ representing the equivalence between two vectors of terms of length $n$.
%
% \begin{figure}[h]
%   \[
%     \begin{array}{rcll}
%       t & := & \sfn(i_1 \dots i_n) \mid x \mid \sff(t_1,\dots,t_n) \mid \sfg(t_1,\dots,t_n) & \text{term}
%       \\
%       \atom & := & t_1,\dots,t_n \sim_n t'_1,\dots,t'_n & \text{atomic formula}
%       \\
%       \phi & := & \atom \mid \top \mid \bot \mid \phi \wedge \phi' \mid \phi \vee \phi' \mid \phi \Rightarrow \phi'\mid \neg \phi & \text{formula}
%     \end{array}
%   \]
%   \caption{Syntax for formulas}
%   \label{fig:syntax-formula}
% \end{figure}
%
% \begin{example}
%   Assuming $\F$ contains $\eq, \ok, \error, \funif$ we can build the following formulas.
%   \begin{itemize}
%     \item $\phi_1 := \myif g() \mythen n_0(i) \myelse n_1(i) \sim_1 n(i)$
%     \item $\phi_2 := \myif \eq(g(),n(i)) \mythen \ok \myelse \error \sim_1 \myif \eq(g(),m(i)) \mythen \ok \myelse \error$
%     \item $\phi_3 := \phi_1 \land \phi_2$
%   \end{itemize}
% \end{example}

\subsection{Semantics}

% See Adrien's thesis, section 2.3:
% \begin{itemize}
%   \item formula interpertation
%   \item computational model
%   \item we note $\M \models \phi$ if and only if for every valuation $\sigma$, $\llbracket \phi \rrbracket_\M^\sigma = \True$
% \end{itemize}


\section{Protocols}

% Without entering into the details of the protocol semantics, we can already
% restrict the set of meta-interpretations of interest, to justify some of the
% rules that we use. We would alternatively impose these conditions from the
% beginning in the definition of meta-interpretations.

There exists a natural translation from applied pi-calculus to this notion of
actions, reminescent of the translation inside Horn Clauses performed by
Proverif, or the translation inside Multi Set Rewritting rules performed by
Sapic (a Tamarin extension).

\bigskip

We  model a protocol as a set of possible actions available to the
attacker. An action models a basic step of the protocol, where
the attacker provides some input, some condition is checked, some
state updates are performed, and finally an output is emitted.

In this section we assume a fixed set of action symbols $\Actions$,
as well as a fixed set of channel names $\C$.
We assume that macro symbols include $\minp$ and $\mout$,
both of index and term arities zero.
We also assume that all other macro symbols are of term arities zero.
We call them \emph{state macros} as they will be used to model memory cells.

\begin{definition}
  \label{def:action}
A \emph{symbolic action} $\alpha = \sfa(i_1,\dots,i_n)$, where $\sfa \in \Actions$ and $i_1,\dots,i_n \in \I$, is defined by:
\begin{itemize}
  \item a meta-logic formula $\phi_{\alpha}$ (intuitively, the condition of this action)
  \item a meta-logic term $o_{\alpha}$ (intuitively, the output of this action)
  \item for each state macro symbol $s$, a meta-logic term $u_{\alpha}^{s}$ (intuitively, the update of the memory cell $s$ for this action)
  \item two channel names $\cin_\alpha$ and $\cout_\alpha$ (intuitively, the channels used for the input and the output of this action)
\end{itemize}
where $\phi_{\alpha}$, $u_{\alpha}^{s}$ and $o_{\alpha}$ must not contain any
macro term and the only free variables that can occur in them are
$x_\alpha$ (intuitively, the input of this action)
and $x^s_\alpha$ where $s$ is a state macro symbol (intuitively, the value
of the memory cell at the time of this action).
% ???
% Names appearing in $\phi_{\alpha}$ and $o_{\alpha}$ are of the form $\sfn(i_1,\dots,i_k)$ with $k \leq n$.
\end{definition}

\begin{definition}
A \emph{concrete action} is an instanciation of a \emph{symbolic action} by giving integer values to index variables.
\end{definition}

\begin{definition}
  \label{def:proto}
  A \emph{protocol} is defined by:
  \begin{itemize}
    \item a finite set of \emph{symbolic actions} with distinct action symbols,
    \item a transitive \emph{sequential dependency} relation $\before$ on \emph{symbolic actions},
    \item a \emph{conflict} relation $\conflict$ on \emph{symbolic actions}.
  \end{itemize}
  Intuitively, $\sfa_1(i)\before\sfa_2(i,j)$ means that, for all $n_i,n_j \in \mathbb{N}$, $\sfa_1(n_i)$ must happen before $\sfa_2(n_i,n_j)$ in any execution
  and $\sfa_1(i,j)\conflict\sfa_2(i,j)$ means that, for all $n_i,n_j \in \mathbb{N}$, $\sfa_1(n_i,n_j)$ and $\sfa_2(n_i,n_j)$ cannot both occur in any execution.
\end{definition}

In our tool, actions are compiled from an applied pi-calculus process.
Sequential dependencies are induced by the syntactic sequences in that
process, and conflicts are induced by conditionals.

\begin{definition}
  \label{def:trace}
  Given a protocol $P$ according to \Cref{def:proto}, an \emph{abstract trace}\footnote{Abstract in the sense that we do not know if the trace is executable.} is a sequence $\sfa_1(\vect {i_1}) \dots \sfa_n(\vect {i_n})$ made of concrete actions such that:
  \begin{itemize}
    \item for all $1 \leq i,j \leq n$, if $\sfa_i = \sfa_j$ then $\vect {i_i} \neq \vect {i_j}$;
    \item for all $1 \leq i < j \leq n$, it is not the case that $\sfa_i(\vect {i_i}) \conflict \sfa_j(\vect {i_j})$;
    \item for all $1 \leq i \leq n$, for all concrete action $\beta$ such that $\beta \before \sfa_i(\vect {i_i})$, there exists $1 \leq j < i$ such that $\beta = \sfa_j(\vect {i_j})$.
  \end{itemize}
\end{definition}

\paragraph{Bi-processes}

When reasoning on indistinguishability of a protocol, we may look at equivalences
between two processes given as a bi-process, i.e. a system with a single set of
symbolic actions, each symbolic action describing an execution step of both the
left and right processes, using $\diff{\_}{\_}$ terms as usual.

In that case, the definition of a symbolic action differs from \Cref{def:action}
in the following way:
\begin{itemize}
  \item $\phi_{\alpha}$ is a meta-logic \emph{bi-formula}
  \item $o_{\alpha}$ and $u_{\alpha}^{\sfs}$ are meta-logic \emph{bi-terms}
\end{itemize}

We define in \Cref{fig:bi-terms} how we extend the syntax in \Cref{fig:terms}
to define meta-logic bi-terms.
Bi-formulas are obtained from the syntax in \Cref{fig:formulas} by allowing
atomic propositions over bi-terms.

\begin{figure}[h]
  \[
  \begin{array}{rcll}
    tt & := & t &\text{term} \\
    & \mid & \diff{t_1}{t_2} &\text{bi-term}\\
    & \mid & F(tt_1,\dots,tt_n) &\text{function application over bi-terms}\\
    & \mid & M(tt_1,\ldots,tt_n)@T &\text{macro application over bi-terms}\\
  \end{array}
  \]
  \caption{Syntax of meta-logic bi-terms}\label{fig:bi-terms}
\end{figure}

We use the notation $(\_)^L$ and $(\_)^R$ to denote the left and right projections
(defined as usual), for example: $(\phi_{\alpha})^L$ and $(\phi_{\alpha})^R$,
$(o_{\alpha})^L$ and $(o_{\alpha})^R$.


\begin{example}[Basic Hash]
  \label{ex:basic-hash-bi-process}
  We introduce the bi-process $P_{BH}$ defined by:
  \begin{itemize}
    \item the set of symbolic actions $\{\T(i,j),\R_1(k,i,j),\R_2(k,i,j)\}$ where,
    \begin{itemize}
      \item action $\alpha_1 := \T(i,j)$
        \begin{itemize}
          \item $\phi_{\alpha_1} := \true$
          \item $o_{\alpha_1} := \langle n(i,j), \h(\diff{k(i)}{k'(i,j)},n(i,j) \rangle$
        \end{itemize}
      \item action $\alpha_2 := \R_1(k,i,j)$
        \begin{itemize}
          \item $\phi_{\alpha_2} := \exists i,j, \snd(\g(x_{\alpha_2})) =  \h(\diff{k(i)}{k'(i,j)},\fst(\g(x_{\alpha_2}))$
          \item $o_{\alpha_2} := \ok$
        \end{itemize}
      \item action $\alpha_3 := \R_2(k,i,j)$
        \begin{itemize}
          \item $\phi_{\alpha_3} := \neg (\exists i,j, \snd(\g(x_{\alpha_2})) =  \h(\diff{k(i)}{k'(i,j)},\fst(\g(x_{\alpha_2})))$
          \item $o_{\alpha_3} := \error$
        \end{itemize}
    \end{itemize}
    \item a conflict relation $\conflict$ such that $\R_1(k,i,j)\conflict\R_2(k,i,j)$.
  \end{itemize}

\noindent
Some traces for this bi-process are:
  \begin{itemize}
    \item $\T(1,1),\R_1(1,1,1)$
    \item $\T(1,1).\R_1(1,1,1).\T(1,2).\R_1(2,1,2)$
    \item $\T(1,1).\R_1(1,1,1).\T(2,1).\R_1(2,2,1)$
  \end{itemize}
\end{example}

\bigskip
\noindent
\begin{remark}
  We could get rid of the conflict relationship and still be able to
  model conditionals: we would have more symbolic traces but, when conditions
  are taken into account, the incompatible conditions of actions corresponding
  to different branches of a conditional would allow to discard
  the unwanted branches. For that reason, we should not need a tactic
  relying on the conflict relationship in the prover.
\end{remark}

\begin{remark}
  The structure that we have assumed on actions is restrictive and
  does not allow to model e.g. phases. It does not matter because we can
  impose other conditions on traces later on as axioms (similar to the
  restrictions of Tamarin).
\end{remark}


% Move to macros.tex
\newcommand{\lreach}[1]{\textsf{l-reach}(#1)}
\newcommand{\rreach}[1]{\textsf{r-reach}(#1)}
\newcommand{\phifresh}[3]{\textsf{n-fresh}_{#1}(#2; #3)}
\newcommand{\phihfresh}[4]{\textsf{h-fresh}_{#1}^{#2}(#3; #4)}

\section{Indistinguishability}

\newcommand{\pair}[1]{\langle #1 \rangle}

% We now define indistinguishability of processes, and how we could verify it.
% We make a distinction between symbolic and observable traces:
% symbolic ones are sequences of action names as in \cref{def:trace},
% while observable traces are sequences of pairs $\pair{c_i,c_o}$.
% Intuitively, such a pair describes the input and output channels of an
% action; several actions might have the same input and output channels.
% We write $A : \pair{c_i,c_o}$ when action $A$ inputs on channel $c_i$ and
% outputs on channel $c_o$.

% For convenience, we talk of processes, though in reality the definitions
% deal with abstract systems described by sets of actions.

\newcommand{\fold}{\mathsf{fold}}

% \begin{definition}
%   Given an observable trace $t$ and a process $P$,
%   we define $\fold(P,t)$ as the frame describing all possible
%   executions of $P$ along $t$:
%   \begin{eqnarray*}
%     \fold(P,\epsilon) &=& \epsilon \\
%     \fold(P,(c_i,c_o).t') &=&
%     \mathsf{if}_{A:(c_i,c_o)}
%     \ldots
%   \end{eqnarray*}
% \end{definition}

% \begin{definition}
%   Two processes $P$ and $Q$ are indistinguishable when,
%   for all observable traces $t$,
%   for all computational model $\Mo$,
%   we have
%   $\Mo \models \fold(P,t) \sim \fold(Q,t)$.
%   This itself means that no attacker can distinguish,
%   with non-negligible probability, between the left- and right-hand sides.
% \end{definition}

% This notion of equivalence should coincide with the usual computational
% indistinguishability for bounded processes. In the unbounded case, it
% is restrictive to quantify on traces in this way
% and ask for indistinguishability only for each trace.

\begin{definition}
  A meta-logic equivalence formula is an element of the form $\pvec{u} \sim \pvec{v}$, where $\pvec{u}$ and $\pvec{v}$ are two vectors of terms of the meta-logic of the same length.
\end{definition}

\begin{definition}
  Two protocoles $\calp_1$ and $\calp_2$, defined over the same signature, are compatible if \adrien{todo}.
\end{definition}

\begin{lemma}
  If $\calp_1$ and $\calp_2$ are compatible then they have the same trace models.
\end{lemma}

\begin{definition}
  Let $\calp_1$ and $\calp_2$ be two compatoble protocoles, and $\pvec{u} \sim \pvec{v}$ be a meta-formula equivalence formula. Then, for every $\calp_1$-trace model $\TM$, computational model $\Mo$ and interpretation~$\sigma$:
  \begin{align*}
    \TM, \Mo, \sigma \models_{\calp_1,\calp_2}
    \pvec{u} \sim \pvec{v}
    &&\text{ iff }&&
    \Mo, \sigma \models
    \interp{\pvec{u}}{\TM}{\calp_1} \sim \interp{\pvec{v}}{\TM}{\calp_2}
  \end{align*}
  Moreover, we say that $\pvec{u} \sim \pvec{v}$ is $(\calp_1,\calp_2)$-valid if $\TM, \Mo, \sigma \models_{\calp_1,\calp_2} \pvec{u} \sim \pvec{v}$ for every $\TM$, $\Mo$, $\sigma$.

  Finally, for any set of meta-formulas $S$, $S \models \pvec{u} \sim \pvec{v}$ if for every $\TM$, $\Mo$, $\sigma$ such that $\TM, \Mo, \sigma \models_{\calp_1,\calp_2} S$, we have $\TM, \Mo, \sigma \models_{\calp_1,\calp_2} \pvec{u} \sim \pvec{v}$.
\end{definition}

\adrien{explain how we use bi-protocols as syntactic suger for pairs of compatible protocols?}

\subsection{Straightforward diff-equivalence}
\adrien{I am not sure exactly what we want to say in this subsection. I left it mostly as is.}

We define the diff-equivalence of a process, by asking the equivalence of the projected frames for all possible traces.

\begin{definition}
  \label{def:process-equiv}
  A bi-protocol $\calp$ is diff-equivalent when,
  for any $\calp$-trace model $\TM$, the formula

  \[
    \dand_{v\in D_\XT}
    \interp{\mframe@\tau^L}{\TM[\tau\mapsto v]}{\calp}
    \sim
    \interp{\mframe@\tau^R}{\TM[\tau \mapsto v]}{\calp}
  \]
  is valid.
  \adrien{Why not just take the last action?}
\end{definition}
\charlie{Notice that we could also extend $\interp{\phi}{\TM}{\calp}$ to formulas containing the $\sim$ symbol, and consider the formula $\forall \tau. \mframe@\tau^L \sim \mframe@\tau^R$.}
Note that this equivalence cannot hold if there exists a trace
whose probability of execution significantly differs between the two
projections of the bi-protocol.
Hence this imposes a form of synchronization on the execution of
conditionals on the two sides of bi-protocoles.
The interest of imposing this artificial constraint is that we
can stop considering executions where one process goes to $\mythen$
branch while the other goes to its $\myelse$ branch. \adrien{The constraint is not artificial. Indeed, since the actions are visible, the adversary sees whether the process is going right or left. There is no loss of generality there.}
\begin{lemma}
  A bi-protocol $\calp$ is diff-equivalent if,
  for any $\calp$-trace model $\TM$, the formula

  \[
    \dand_{v\in D_\XT}
    \interp{\pair{\mframe@\tau^L,\mexec@\tau^L}}{\TM[\tau\mapsto v]}{\calp}
    \sim
    \interp{\pair{\mframe@\tau^R, \mexec@\tau^R}}{\TM[\tau \mapsto v]}{\calp}
  \]
  is valid.
  \adrien{This just complicated to show than what Definition~\ref{def:process-equiv} requires. Maybe there is a typo?}
\end{lemma}

\newcommand{\In}{\mathsf{in}}
\newcommand{\Out}{\mathsf{out}}

\begin{example}
  Consider the bi-protocol
  $\In(c,x).\myif x=\diff{0}{1} \mythen \Out(c,n) \myelse \Out(c,m)$
  where $n$ and $m$ are arbitrary, possibly equal names.
  Its two projections are indistinguishable, but the
  bi-protocol is not diff-equivalent.
  \adrien{I do not agree. If you write the process using actions (as we do), then it is equivalent. When you are translating from the pi-process to an representation using actions, you cannot change the visible actions (or this is not a sound translation).}
  Indeed we have
  $\myif g()=0 \mythen \Out(c,n) \not\sim
  \myif g()=1 \mythen \Out(c,m)$: the attacker can simply choose
  $g()=0$ to distinguish the two sides.
  Our bi-protocol can however easily be
  rewritten into a diff-equivalent process, e.g. by pushing the conditional
  inside the output.

  If we modify our bi-protocol into
  $\In(c,x).\myif x=\diff{0}{1} \mythen \Out(c,n) \myelse \Out(c,0)$
  then the two projections become distinguishable.
  The attack is obtained with an execution
  where one process outputs a name while the other outputs $0$. Such
  desynchronized executions are not taken into account with diff-equivalence,
  but diff-equivalence still fails due to the desynchronized condition,
  as before.
\end{example}

In the next examples, we omit the $\myelse$ branch when it consists of a null
process. In these examples, there is a coincidence between diff-equivalence
and indistinguishability, because observable actions coincide with symbolic
actions.

\begin{example} \label{ex:negl}
  Consider the bi-protocol
  $\In(c,x).\myif x=\diff{n}{m} \mythen \Out(c,\ok)$.
  It is diff-equivalent, and its projections are
  indistinguishable as expected.
  In the bi-protocol, the condition $x=\diff{n}{m}$ does not pass
  with the same inputs on the left and right, but it passes with
  the same negligible probability.
\end{example}

\begin{example} \label{ex:sync}
  Consider
  $\In(c,x).\myif x=(\diff{n}{m})_0 \mythen \Out(c,n)$
  where $(t)_0$ denotes the first bit of $t$.
  This bi-protocol is not diff-equivalent because
  $\myif x=(n)_0 \mythen n \not\sim \myif x=(m)_0 \mythen n$, and
  the two projections are distinguishable for the same
  reason: the attacker sends $0$;
  on the left he receives with probability $1\over 2$ a bitstring whose
  first bit is $0$;
  on the right process he receives with probably only $1\over 4$
  a bitstring whose last bit is $0$.
  If we change $\Out(c,n)$ into $\Out(c,\ok)$,
  we have indistinguishable processes and diff-equivalence holds.
\end{example}

\begin{example} \label{ex:problem}
  Consider $\Out(c,\diff{n}{m}).
  \In(c,x).
  \myif x=\diff{n}{m} \mythen \Out(c,\ok)$.
  The projections are observationally equivalent and diff-equivalence
  holds -- in fact they are $\alpha$-equivalent.
\end{example}


\subsection{Reasoning about equivalences}

We extend our sequent calculus with rules to prove the equivalence formulas of the meta-logic.

\begin{definition}
  Let $\calp_1,\calp_2$ be compatible protocols. An equivalence sequent $\Gamma \vdash_{\calp_1,\calp_2} \pvec{u} \sim \pvec{v}$ comprises a set of hypotheses $\Gamma$ and a goal $\pvec{u} \sim \pvec{v}$, where $\Gamma$ and $\pvec{u} \sim \pvec{v}$ are all equivalence formulas of the meta-logic.

  The sequent  $\Gamma \vdash_{\calp_1,\calp_2} \pvec{u} \sim \pvec{v}$ is valid whenever $\Gamma \models_{\calp_1,\calp_2} \pvec{u} \sim \pvec{v}$ is $(\calp_1,\calp_2)$-valid.
\end{definition}

\begin{lemma}
  A bi-protocol $\calp$ is diff-equivalent if and only if
  \[
    \mframe^L@\pre(\tau) \sim \mframe^R@\pre(\tau)
    \vdash_{\calp_1,\calp_2}
    \mframe^L@\tau \sim \mframe^R@\tau
  \]
\end{lemma}
\begin{proof}
  This is by induction on the length of the traces.
\end{proof}

% \begin{lemma}
%   A bi-protocol is diff-equivalent if and only if,
%   \[ \emptyset \vdash  \mframe^L@\tau \sim \mframe^R@\tau\]
%   Equivalently, is it diff-equivalent if and only if:
%   \[\mframe^L@\pre(\tau) \sim \mframe^R@\pre(\tau) \vdash \mframe^L@\tau \sim \mframe^R@\tau\]
% \end{lemma}
% \begin{proof}
%   We prove the first equivalence, which is essentially an unfolding of definitions.
%   \[
%     \begin{array}{l@{~}l}
%       $P$\text{ is diff-equivalent} & \Leftrightarrow \text{for all } \TM,\ \stackrel{.}{\wedge}_{v\in D_\XT} (\mframe@\tau^L)^{\TM[\tau\mapsto v]} \sim (\mframe@\tau^R)^{\TM[\tau \mapsto v]}\text{ is valid}\\
%       & \Leftrightarrow \text{for all } \TM\text{ and } v\in D_\XT,\ (\mframe@\tau^L)^{\TM[\tau\mapsto v]} \sim (\mframe@\tau^R)^{\TM[\tau \mapsto v]}\text{ is valid}\\
%       & \Leftrightarrow \text{for all } \TM\text{ and } v\in D_\XT,\ \true \stackrel{.}{\Rightarrow} (\mframe@\tau^L)^{\TM[\tau\mapsto v]} \sim (\mframe@\tau^R)^{\TM[\tau \mapsto v]}\text{ is valid}\\
%       & \Leftrightarrow  \emptyset \vdash  \mframe^L@\tau \sim \mframe@\tau^R\text{ is valid}\\

%     \end{array}
%   \]

%   The second equivalence is a direct induction on the length of the traces.
% \end{proof}

Adding the execution condition to the sequent is also a valid proof technique.
\begin{lemma}
  A bi-protocol $\calp$ is diff-equivalent if,
  \[ \emptyset \vdash_{\calp_1,\calp_2}  \pair{\mframe^L@\tau,\mexec@\tau^L} \sim \pair{\mframe@\tau^R,\mexec@\tau^R}\]
  \adrien{It is harder to prove the formula above than to prove $\emptyset \vdash_{\calp_1,\calp_2}  \mframe^L@\tau \sim \mframe^R@\tau$. Maybe there is a typo?}
\end{lemma}


We provide in \cref{fig:lk-ind} a set of rules for equivalence sequent. These rules are independent of the protocoles $\calp_1,\calp_2$. Note that:
\begin{itemize}
\item the rule $\textsc{Sym}$ swaps the two protocols.
\item the $\textsc{Refl}$ rule checks that the term $t$ is macro-free, which ensures that $\interp{\pvec{t}}{\TM}{\calp_1}$ and $\interp{\pvec{t}}{\TM}{\calp_2}$ are \emph{syntactically} the same (base logic) terms in any $\calp_1$-trace model $\TM$.
\item the rule $\textsc{Trans}$ introduces an intermediate protocol $\calp_3$ which must be compatible with $\calp_1$ and $\calp_2$.
\end{itemize}

\adrien{explain that, in the tool, we fix the protocols. Hence $\textsc{Sym}$ and $\textsc{Trans}$ are not implemented.}


\begin{figure}
  \begin{mathpar}
    \inferrule[Expand]{
      \Gamma \vdash_{\calp_1,\calp_2} \pvec{u} \sim \pvec{v}
    }{\Gamma \cup \phi \vdash_{\calp_1,\calp_2} \pvec{u} \sim \pvec{v}}

    \inferrule[F-Cut]{
      \Gamma \vdash_{\calp_1,\calp_2} \phi\\      
      \Gamma \cup \phi \vdash_{\calp_1,\calp_2} \pvec{u} \sim \pvec{v}
    }{
      \Gamma \vdash_{\calp_1,\calp_2} \pvec{u} \sim \pvec{v}
    }

    \inferrule[Sym]{
      \Gamma \vdash_{\calp_2,\calp_1} \pvec{u} \sim \pvec{v}
    }{\Gamma \vdash_{\calp_1,\calp_2} \pvec{v} \sim \pvec{u}}

    \inferrule[Refl]{~}
    { \Gamma \vdash_{\calp_1,\calp_2} \pvec{t} \sim \pvec{t} }
    \quad\text{$\pvec{t}$ are macro-free}

    \inferrule[Trans]{
      \Gamma \vdash_{\calp_1,\calp_2} \pvec{u} \sim \pvec{t}\\
      \Gamma \vdash_{\calp_1,\calp_3} \pvec{t} \sim \pvec{v}
    }{\Gamma \vdash_{\calp_1,\calp_3} \pvec{u} \sim \pvec{v}}
    \quad\text{$\calp_1,\calp_2,\calp_3$ are compatible}
    
    \inferrule[Subst]{
      \Gamma \cup \{t_1 \deq t_2\sim \true \}  \vdash_{\calp_1,\calp_2} \pvec{u} \sim \pvec{v}
    }{
      \Gamma \cup \{t_1 \deq t_2\sim \true \} \vdash_{\calp_1,\calp_2}
      \pvec{u}[ t_1 / t_2] \sim  \pvec{v}[ t_1 / t_2]
    }

    \inferrule[if-reach]{
      \phi \vdash_{\calp_1,\calp_2} \false
    }{
      \Gamma  \vdash_{\calp_1,\calp_2} \myif \phi \mythen {u} \sim  \myif \phi \mythen {v}
    }

    \inferrule[if-equiv]{
      \phi \vdash_{\calp_1,\calp_2} \psi \Leftrightarrow \psi'
    }{
      \Gamma \vdash_{\calp_1,\calp_2}
      \myif  \phi \wedge \psi \mythen t \sim \myif \phi \wedge \psi' \mythen t
    }

    \inferrule[${\lnot}$-R]{
      \Gamma \vdash_{\calp_1,\calp_2} \false \sim \true
    }{
      \Gamma \vdash_{\calp_1,\calp_2} \pvec{u} \sim \pvec{v}
    }

    \inferrule[fresh]{
      n,m \not \in \Gamma
    }{
      \Gamma \vdash_{\calp_1,\calp_2} n \sim m
    }

    \inferrule[if-weak]{
      \Gamma \vdash_{\calp_1,\calp_2} \phi, \pvec{u} \sim \phi, \pvec{v}
    }{
      \Gamma \vdash_{\calp_1,\calp_2} \myif  \phi \mythen \pvec{u} \sim \myif \phi \mythen \pvec{v}
    }

    \inferrule[Dup]{
      \Gamma \vdash_{\calp_1,\calp_2} \pvec{u},s \sim \pvec{v},t
    }{
      \Gamma \vdash_{\calp_1,\calp_2} \pvec{u},s,s \sim \pvec{v},t,t
    }
  \end{mathpar}
  \caption{Generic inference rules for indistinguishability.}
  \label{fig:lk-ind}
\end{figure}

\begin{lemma}
  The rules presented in Figure~\ref{fig:lk-ind} are sound.
\end{lemma}

\paragraph{Other rules.}

\Cref{fig:fresh,fig:prf,fig:xor} presents the rules for Fresh, PRF and XOR tactics.
We use the following notations:
\begin{itemize}
\item $\Gamma \vdash_{\calp_1,\calp_2} u$ stands for $\Gamma^L \sim \Gamma^R \vdash_{\calp_1,\calp_2} u^L \sim u^R$
\item $A \in S$ stands for every action in the system (or protocol)
\item $A(\pvec i)^L$ represents the left projection of meta-logic bi-terms and bi-formulas describing the action $A(\pvec k)$ (outputs, updates and conditions)
\item $k(\_) \sqsubseteq_{\h(\_,\cdot)} u$ means that the indexed key $k$ appears only in key positions in $u$
\item indices $\pvec i$ in $A(\pvec i)$ are chosen fresh with relation to the appropriate environment (i.e. indices appearing in $u, C, t, k, \pvec {j_0}$)
\end{itemize}

\begin{figure}[h]
  \textbf{Fresh rule:}
  {\small\[\arraycolsep=10pt
      \begin{array}{|c|c|}
        \hline
        \text{Base logic Rule} &
        \text{Meta-logic Rule}\\
        \hline
        %%%%%%%%%%%%%%%%%%%%%%%%%%%%%%%%%%%%%%%%%%%%%%%%%%%%%%%%%%%%%%%% 
        % Fresh
        %%%%%%%%%%%%%%%%%%%%%%%%%%%%%%%%%%%%%%%%%%%%%%%%%%%%%%%%%%%%%%%% 
        \inferrule{
          \Gamma \vdash \pvec{u}, C[0] \sim \pvec{v}, C[0]
        }{
          \Gamma \vdash \pvec{u}, C[\mathsf{n}] \sim \pvec{v}, C[\mathsf{n}']
        }
        &        
        \inferrule{
          \Gamma \vdash_{\calp_1,\calp_2}
          {\begin{alignedat}[t]{2}
              &
              \pvec{u},
              C\big[
              \myif \phifresh{\calp_1}{\mathsf{n}[\pvec{i}]}{\pvec{u},C}
              \mythen 0 \myelse \mathsf{n}[\pvec{i}]
              \big]\\
              \sim\;\;&
              \pvec{v},
              C\big[
              \myif \phifresh{\calp_2}{\mathsf{n}'[\pvec{i}']}{\pvec{v},C}
              \mythen 0 \myelse \mathsf{n}'[\pvec{i}']
              \big]
            \end{alignedat}}
        }{
          \Gamma \vdash_{\calp_1,\calp_2}
          \pvec{u}, C[\mathsf{n}[\pvec{i}]] \sim \pvec{v}, C[\mathsf{n}'[\pvec{i}']]
        }\\[2em]
        \text{where }
        \mathsf{n} \not \in \st(\pvec{u},C)
        \text{ and }
        \mathsf{n}' \not \in \st(\pvec{v},C)
        &
        \text{where }
        \begin{alignedat}[t]{2}
          &\phifresh{\calp}{\mathsf{n}[\pvec{i}]}{\pvec{u}}
          &\;\;\overset{def}{=}\;\;&
          \quad
          \bigwedge_{\mathclap{(\mathsf{n}[\pvec{j}],\pvec{j},c) \in \ost_{\calp}(\pvec{u})}}
          \quad
          \forall \pvec{j}, c \rightarrow \pvec{i} \ne \pvec{j}
        \end{alignedat}
        \\\hline
      \end{array}
    \]}

  \textbf{PRF Rule:}
  {\small\[\arraycolsep=10pt
      \begin{array}{|c|c|}
        \hline
        \text{Base logic Rule} &
        \text{Meta-logic Rule}\\
        \hline
        %%%%%%%%%%%%%%%%%%%%%%%%%%%%%%%%%%%%%%%%%%%%%%%%%%%%%%%%%%%%%%%% 
        % PRF
        %%%%%%%%%%%%%%%%%%%%%%%%%%%%%%%%%%%%%%%%%%%%%%%%%%%%%%%%%%%%%%%% 
        \inferrule{
          \Gamma \vdash
          \pvec{u},
          C\Big[\;
          {\begin{alignedat}[c]{1}
              \myif \phihfresh{}{\mathsf{k}}{t}{\pvec{u},C,t}
              &\mythen \mathsf{n}\\[-0.5em] &\myelse h(t,\mathsf{k})
            \end{alignedat}}
          \;\Big]
          \sim
          \pvec{v}, s
        }{
          \Gamma \vdash
          \pvec{u}, C[h(t,\mathsf{k})] \sim \pvec{v}, s
        }
        &
        \inferrule{
          \Gamma \vdash_{\calp_1,\calp_2}
          \pvec{u},
          C\Big[\;
          {\begin{alignedat}[c]{1}
              \myif \phihfresh{\calp_1}{\mathsf{k}[\pvec{i}]}{t}{\pvec{u},C,t}
              &\mythen \mathsf{n}\\[-0.5em] &\myelse h(t,\mathsf{k}[\pvec{i}])
            \end{alignedat}}
          \;\Big]
          \sim
          \pvec{v}, s
        }{
          \Gamma \vdash_{\calp_1,\calp_2}
          \pvec{u}, C[h(t,\mathsf{k}[\pvec{i}])] \sim \pvec{v}, s
        }
        \\[2em]
        \text{when }
        \mathsf{n} \text{ fresh},
        \mathsf{k} \tpos_{h(\_,\cdot)} \st(\pvec{u},C,t)
        &
        \text{when }
        \mathsf{n} \text{ fresh}, 
        \mathsf{k} \tpos^{\calp_1}_{h(\_,\cdot)} \st(\pvec{u},C,t)
        \\[1em]
        \text{and }
        \begin{alignedat}[t]{2}
          &\phihfresh{}{\mathsf{k}}{t}{\pvec{u}}
          &\;\;\overset{def}{=}\;\;&
          \quad
          \bigwedge_{\mathclap{h(m,\mathsf{k}) \in \st(\pvec{u})}}
          \quad
          m \ne t
        \end{alignedat}
        &
        \text{and }
        \begin{alignedat}[t]{2}
          &\phihfresh{\calp}{\mathsf{k}[\pvec{i}]}{t}{\pvec{u}}
          &\;\;\overset{def}{=}\;\;&
          \quad
          \bigwedge_{\mathclap{(h(m,\mathsf{k}[\pvec{i_0}]),\pvec{j},c) \in \ost_{\calp}(\pvec{u})}}
          \quad
          \forall \pvec{j},
          (c \wedge \pvec{i} = \pvec{i_0})
          \rightarrow m \ne t
        \end{alignedat}
        \\\hline
      \end{array}
    \]}

  \textbf{XOR Rule:}
  {\small
    \[\arraycolsep=10pt
      \begin{array}{|c|c|}
        \hline
        \text{Base logic Rule} &
        \text{Meta-logic Rule}\\
        \hline
        %%%%%%%%%%%%%%%%%%%%%%%%%%%%%%%%%%%%%%%%%%%%%%%%%%%%%%%%%%%%%%%% 
        % XOR
        %%%%%%%%%%%%%%%%%%%%%%%%%%%%%%%%%%%%%%%%%%%%%%%%%%%%%%%%%%%%%%%% 
        \inferrule{
          \Gamma \vdash \mylen(t) = \mylen(\mathsf{n})\\\\
          \Gamma \vdash
          \pvec{u}, C[\mathsf{m}] \sim \pvec{v}, s
        }{
          \Gamma \vdash
          \pvec{u}, C[t \oplus \mathsf{n}[\pvec{j}]] \sim \pvec{v}, s
        }
        &
        \inferrule{
          \lreach{\Gamma} \vdash_{\calp_1}
          \mylen(t) = \mylen(\mathsf{n}[\pvec{j}])\\\\
          \Gamma \vdash_{\calp_1,\calp_2}
          \pvec{u},
          C\Big[\;
          {\begin{alignedat}[c]{1}
              \myif \phifresh{\calp_1}{\mathsf{n}[\pvec{j}]}{\pvec{u},C,t}
              &\mythen \mathsf{m}\\[-0.5em] &\myelse t \oplus \mathsf{n}[\pvec{j}]
            \end{alignedat}}
          \;\Big]
          \sim
          \pvec{v}, s
        }{
          \Gamma \vdash_{\calp_1,\calp_2}
          \pvec{u}, C[t \oplus \mathsf{n}[\pvec{j}]] \sim \pvec{v}, s
        }   
        \\[2em]
        \text{when }
        \mathsf{m} \text{ fresh and }
        \mathsf{n} \not \in \st(\pvec{u},C,t)
        &
        \text{when }
        \mathsf{m} \text{ fresh and }
        \begin{alignedat}[t]{2}
          &\phifresh{\calp}{\mathsf{n}[\pvec{i}]}{\pvec{u}}
          &\;\;\overset{def}{=}\;\;&
          \quad
          \bigwedge_{\mathclap{(\mathsf{n}[\pvec{j}],\pvec{j},c) \in \ost_{\calp}(\pvec{u})}}
          \quad
          \forall \pvec{j}, c \rightarrow \pvec{i} \ne \pvec{j}
        \end{alignedat}
        \\\hline
      \end{array}
    \]}

  \adrien{I use the function $\lreach{\Gamma}$ in the XOR rule, which extracts from $\Gamma$ the \emph{left} reachability statements. If we keep it, I will add the definition.}
  
  \caption{Rules of the base logic, and corresponding meta-logic rules.}
  \label{fig:rules-corresp-equiv}
\end{figure}


\begin{figure}[h]
  \begin{mathpar}
    \inferrule[FA-DUP]{
      \inferrule{
        \Gamma \vdash_{\calp_1,\calp_2} u,
        \mframe @ \pre(A(\pvec i)),
        \mexec @ \pre(A(\pvec i))
      }{
        \Gamma \vdash_{\calp_1,\calp_2} u,
        \mframe @ \pre(A(\pvec i)),
        \myif \mexec @ \pre(A(\pvec i)) \mythen \phi_{h} \myelse \bot
      }
    }{
      \Gamma \vdash_{\calp_1,\calp_2} u,
      \mframe @ \pre(A(\pvec i)),
      \mexec @ \pre(A(\pvec i)) \wedge \phi_{h}
    }
  \end{mathpar}

  We ask that $\phi_h \in H_{\{\pre(A(\pvec i))\}}$ where, for any set of
  timestamps $T$, $H_T$ is the least set of formulas and messages
  closed under function application, boolean connectives, and the
  following rules:
  $$ \inferrule{
    B(\pvec j) \in T
    \quad
    \phi \in H_{T\cup\{C(\pvec {k})\}}
  }{
    (\forall \pvec k.~ C(\pvec k)\leq B(\pvec j) \Rightarrow \phi) \in H_T
  }
  \quad\quad
  \inferrule{
    B(\pvec j) \in T
    \quad
    \phi \in H_{T\cup\{C(\pvec k)\}}
  }{
    (\exists \pvec k.~ C(\pvec k)\leq B(\pvec j) \wedge \phi) \in H_T
  }
  $$
  $$\inferrule{B(\pvec j) \in T}{
    \minp @ B(\pvec j) \in H_T
  }\quad\quad
  \inferrule{ }{
    \minp @ A(\pvec i) \in H_T
  }\quad\quad
  \inferrule{B(\pvec j) \in T}{
    \mout @ B(\pvec j) \in H_T
  }$$
  $$
  \inferrule{\phi \in H_T \quad B(\pvec j) \in T}{
    (\myif \mexec @ B(\pvec j) \mythen \phi \myelse \psi) \in H_T}
  $$
  \caption{FA-DUP rule.
  }
  \label{fig:fadup}
\end{figure}

\clearpage
\section{Archives}
\subsection{A proof technique}

Diff-equivalence is usually proved by induction and case analysis on
the timestamp. Even cases where the left and right actions are locally
identical are not trivial: it may be e.g.\ that the same name is outputted
by the action on both sides, but that each side has previously released
different information on that name.

To prove diff-equivalence, it can be interesting to prove that:
\[
  \phi^L_\tr, \phi^R_\tr
  \vdash_{\calp_1,\calp_2}
  \phi^L_\alpha \Leftrightarrow \phi^R_\alpha
\]

Then, in the induction step, we can directly replace the previous conditions by the same one.


This condition is only used to help for proving diff-equivalence, it is not necessary.
The gap comes from the fact that we are requiring conditions to be
synchronized for all random samplings.

\begin{example}
  With the bi-protocol of \cref{ex:negl} we would have to prove
  $\vdash_{\calp_1,\calp_2} g() = n \Leftrightarrow g() = m$ and
  $g()=n, g()=m \vdash_{\calp_1,\calp_2} \ok \sim \ok$, both of which hold.
  With the bi-protocol of \cref{ex:sync} we would have to prove
  $\vdash_{\calp_1,\calp_2} g() = (n)_0 \Leftrightarrow g() = (m)_0$ which does not hold.
\end{example}

\begin{example} \label{ex:indep}
  This proof technique does not work for \cref{ex:problem}.
  The same problem appears with the Basic-Hash protocol, even if we work around
  the problem described in \cref{sec:refined-diff}, we won't be able to show
  that conditionals are synchronized.  In the simple case of the trace
  $T(i,j).R(k,i,j)$ we have
  on the single-session side
  $$\pi_2(g_2(\pair{n_T(i,j),h(n_T(i,j),k'(i,j))})) =
  h(\pi_1(g_2(\ldots)),k'(i,j))$$
  and we would like this to imply (in the meta-logic)
  the same equality with $k(i)$ instead of $k'(i,j)$.
  This implication does not hold with overwhelming probability in all
  computational models: as in \cref{ex:problem}, $g_2(x)$ could be the second
  projection of $x$ with its first bit changed to $0$; if the hash is PRF,
  there should be a probability of roughly $1 \over 4$ that this leaves
  the hash unchanged with $k'(i,j)$ but not with $k(i)$.
\end{example}

\section{Outdated example : a signed DDH key exchange}
\charlie{abus de notations dans cette partie}

We briefly show how one can prove the security of a signed DDH key exchange. The protocol in pi-calculus is provided in Figure~\ref{fig:signed_ddh} and the run of an honest execution in Figure~\ref{fig:dh_ke}. This example is a simplified instance of classical key-exchange security. Notably, we assume that identities are already fixed.

\begin{figure}
  % \setlength{\belowcaptionskip}{-15pt}
  \setmsckeyword{} \drawframe{no}
  \setmscscale{0.9}
  \begin{center}
    \begin{msc}{}
      \setlength{\instwidth}{0\mscunit}
      \setlength{\instdist}{7cm}
      \setlength{\topheaddist}{0cm}
      \declinst{initiator}{
        \begin{tabular}[c]{c}
          \textsc{A} \\
          \colorbox{gray}{{\;\; $sk_A,a_i$\;\;}}
        \end{tabular}}{}

      \declinst{receiver}{
        \begin{tabular}[c]{c}
          \textsc{B} \\
          \colorbox{gray}{{\;\;  $sk_B,b_i$ \;\;}}
        \end{tabular}}{}

      \nextlevel[-1]
      \mess{$\mysign(g^{a_i},sk_A)$}{initiator}{receiver}
      \nextlevel[1.5]
      \mess{$\mysign(<g^{a_i},g^{b_i}>,sk_B)$}{receiver}{initiator}
      \nextlevel[1.5]
      \mess{$\mysign(<g^{a_i},g^{b_i}>,sk_A)$}{initiator}{receiver}




    \end{msc}
  \end{center}
  \caption{Diffie Hellman key exchange}\label{fig:dh_ke}
\end{figure}

\begin{figure}
  \[
    \begin{array}{cc}
      \begin{array}[t]{l@{~}l}
        A_i := & \aout{\mysign(g^{a_i},sk_A)} : \alpha_1; \\
        &\ain{x}; \\
        & \myif \mycsign(x,pk(sk_B)) \\
        & ~ \wedge \pi_1(\mygetmess(x))=g^{a_i}  \mythen \\
        & \quad \aout{\mysign(\mygetmess(x),sk_A) } : \alpha_2; \\
        & \quad \myfind j \mysuchthat g^{b_j} = \pi_2(\mygetmess(x)) \\
        & \qquad \aout{\diff{\pi_2(\mygetmess(x))^{a_i}}{k_{i,j}}} : \alpha_3\\
        & \quad \myelse \\
        & \qquad \aout{\diff{\pi_2(\mygetmess(x))^{a_i}}{\bot}}  : \alpha_4 \\
        & \myelse \\
        & \bot
      \end{array}
      &
      \begin{array}[t]{l@{~}l}
        B_i := &\ain{x}; \\
        & \myif \mycsign(x,pk(sk_A)) \mythen \\
        & \quad \aout{\mysign(<\mygetmess(x), g^{b_i}>,sk_B)} : \beta_1; \\
        & \quad \ain{y}; \\
        & \quad \myif \mycsign(y,pk(sk_A))\\
        & \quad ~ \wedge \mygetmess(y) = <\mygetmess(x), g^{b_i}> \mythen \\
        & \quad \quad \myfind j \mysuchthat g^{a_j} = \mygetmess(x) \\
        & \quad \qquad \aout{\diff{\mygetmess(x)^{b_i}}{k_{j,i}}} : \beta_2 \\
        & \quad \quad \myelse \\
        & \quad \qquad \aout{\diff{\mygetmess(x)^{b_i}}{\bot}} : \beta_3 \\
        & \quad \myelse \\
        & \quad \bot \\
        & \myelse \\
        & \bot
      \end{array}
    \end{array}
  \]
  \label{fig:signed_ddh}
  \caption{A signed DDH key exchange}
\end{figure}

We outline the proof of the fact that $!_i A_i \| B_i$ is diff-equivalent. There are four actions with choices in the output, thus, we have to show that, for all trace $\tr$, for all $i,j$:
\begin{enumerate}
\item $\begin{array}[t]{l}
    \phi_\tr,  \mycsign(x,pk(sk_B)), \mygetmess(x)=<g^{a_i}, g^{b_j}>, \mouts_\tr^L \sim \mouts_\tr^R,  \\
    \quad \vDash \mouts_\tr^L, \pi_2(\mygetmess(x))^{a_i} \sim \mouts_\tr^R, k_{i,j}
  \end{array}
  $ ($\alpha_3$)

\item $\begin{array}[t]{l}
    \phi_\tr,  \mycsign(x,pk(sk_B)),  \not \exists j. \pi_2(mygetmess(x))= g^{b_j},  \mouts_\tr^R \sim \mouts_\tr^L \\
    \quad \vDash \mouts_\tr^L, \pi_2(\mygetmess(x))^{a_i} \sim \mouts_\tr^R, \bot
  \end{array} $ ($\alpha_4$)
\item $\begin{array}[t]{l}
    \phi_\tr,  \mycsign(y,pk(sk_A)), \mycsign(x,pk(sk_A)), \mygetmess(y)=<g^{a_j}, g^{b_i}>,  \mouts_\tr^R \sim \mouts_\tr^L \\ \quad \vDash \mouts_\tr^L, \pi_2(\mygetmess(x))^{b_i} \sim \mouts_\tr^R, k_{j,i}
  \end{array}$ ($\beta_2$)
\item $\begin{array}[t]{l}
    \phi_\tr,  \mycsign(y,pk(sk_A)), \mycsign(x,pk(sk_A)),  \not \exists j. \pi_2(mygetmess(x))= g^{b_j},  \mouts_\tr^R \sim \mouts_\tr^L\\
    \vDash \mouts_\tr^L, \pi_2(\mygetmess(x))^{b_i} \sim \mouts_\tr^R, \bot
  \end{array}$ ($\beta_3$)
\end{enumerate}

Regarding goal $(2)$ and $(4)$, we remark that it is a case where $\Gamma \vdash \false$. Indeed, for $(2)$ applying EUFCMA yields that there exists $j$ such that $x = \mysign(<g^{a_i},g^{b_j}>,sk_B)$ which is in contradiction with  $\not \exists j. \pi_2(mygetmess(x))= g^{b_j}$.

Regarding goals $(1)$ and $(3)$, we mainly use DDH. To this end, we first use EUFMCA, to ensure that we have a matching conversation between the two sessions, and then use DDH. \charlie{je détail pas, ça prend du temps de formaliser proprement DDH vis à vis des actions, et je pense pas que ce soit l'objectif actuel}




%%% Local Variables:
%%% mode: latex
%%% TeX-master: "main"
%%% End:


\newpage

\section{Old stuff}



\paragraph{Facts}
Facts are schemas of $\bc$ formulas indexed by path constraints:
\[
  \phi \mdef
  \cforall_\theta\, \valpha.\;
  \psi
  \ra
  \bigvee_{i}
  \cexists_\gammai \vbetai. \psi_i
\]

\paragraph{Environments}
\[
  \env \mdef
  \aset \mid
  \pcnstr \mid
  \decls
  \qquad
  \text{ where }
  \qquad
  \decls \mdef
  \left\{
    \decl{t_\sfx}
    {t_\epsilon,(t_\sfy^\act)_{\act \in \aset}}
  \right\}
\]
where:
\begin{itemize}
\item $\aset$ is the set of symbolic actions.
\item $\pcnstr$ is a set of path constraints.
\item $\decls$ is a set of inductive term declarations.
\end{itemize}
\paragraph{Judgments}
\[
  \ejudge{\env}{\facts{\valpha}{\theta}{\Gamma}}{\phi}
\]
where:
\begin{itemize}
\item $\env$ is the environment.
\item $\pvtype{\valpha}{\theta}$ declares a set of constrained path variables.
\item $\Gamma$ is a set of facts.
\item $\phi$ is the goal.
\end{itemize}

\begin{remark}
  $\env$ will be invariant during derivations. We omit it when there is no ambiguity.
\end{remark}

\paragraph{Standard Introduction Rules}
\begin{mathpar}
  \inferrule[\rraintro]{
    \judge{\facts{\valpha}{\theta}{
        \Gamma \cup \{\psi_0\}}}
    {
      \cexists_\gammap \vdelta. \psi
    }
  }{
    \judge{\facts{\valpha}{\theta}{\Gamma}}
    {
      \psi_0
      \ra
      \cexists_\gammap \vdelta. \psi
    }
  }

  \inferrule[\andlintro]{
    \judge{\facts{\valpha}{\theta}{\Gamma \cup \{ \psi \} \cup \{\psi'\}}}
    {\phi}
  }{
    \judge{\facts{\valpha}{\theta}{\Gamma \cup \{ \psi \wedge \psi'\}}}
    {\phi}
  }

  \inferrule[\orlintro]{
    \judge{\facts{\valpha}{\theta}{\Gamma \cup \{ \psi \}}}
    {\phi}\\
    \judge{\facts{\valpha}{\theta}{\Gamma \cup \{\psi'\}}}
    {\phi}
  }{
    \judge{\facts{\valpha}{\theta}{\Gamma \cup \{ \psi \vee \psi'\}}}
    {\phi}
  }

  \inferrule[\andrintro]{
    \judge{\facts{\valpha}{\theta}{\Gamma}}{\psi}\\
    \judge{\facts{\valpha}{\theta}{\Gamma}}{\psi'}
  }{
    \judge{\facts{\valpha}{\theta}{\Gamma}}{\psi \wedge \psi'}
  }

  \inferrule[\orrintro-1]{
    \judge{\facts{\valpha}{\theta}{\Gamma}}{\psi}
  }{
    \judge{\facts{\valpha}{\theta}{\Gamma}}{\psi \vee \psi'}
  }

  \inferrule[\orrintro-2]{
    \judge{\facts{\valpha}{\theta}{\Gamma}}{\psi'}
  }{
    \judge{\facts{\valpha}{\theta}{\Gamma}}{\psi \vee \psi'}
  }

  \inferrule[\toprintro]{
  }{
    \judge{\facts{\valpha}{\theta}{\Gamma}}{\top}
  }
\end{mathpar}

\paragraph{Other Rules}
\begin{mathpar}
  \inferrule[\absurd]{
    \bot \in \Gamma
  }{
    \judge{\facts{\valpha}{\theta}{\Gamma}}{\phi}
  }

  \inferrule[\axiom]{
    \psi \in \Gamma
  }{
    \judge{\facts{\valpha}{\theta}{\Gamma}}{\psi}
  }

\end{mathpar}

\paragraph{Modal Introduction Rules}
\begin{mathpar}
  \inferrule[\grintro]{
    \judge{\facts{\valpha,\vbeta}{\theta\wedge\gamma}{\Gamma}}
    { \phi }\\
    \valpha \cap \vbeta = \emptyset
  }{
    \judge{\facts{\valpha}{\theta}{\Gamma}}
    {
      \cforall_\gamma\, \vbeta.\;
      \phi
    }
  }

  \inferrule[\printro]{
    \nu : \vbeta \mapsto \valpha\\
    \judge{\facts{\valpha}{\theta}{\Gamma}}{\psi\nu}\\
    \centail{\theta}{\gamma\nu}
  }{
    \judge{\facts{\valpha}{\theta}{\Gamma}}{\cexists_\gamma \vbeta. \psi}
  }
\end{mathpar}

\paragraph{Path Constraints Rules}
\begin{mathpar}
  \inferrule[\clweaken]{
    \judge
    {\facts{\valpha}{\gamma}{\Gamma}}{\phi}\\
    \centail{\theta}{\gamma}
  }{
    \judge{\facts{\valpha}{\theta}{\Gamma}}{\phi}
  }

  \inferrule[\crstrengthen]{
    \judge{\facts{\valpha}{\theta}{\Gamma}}{\cexists_\delta \vbeta. \phi}\\
    \centail{\delta}{\gamma}
  }{
    \judge{\facts{\valpha}{\theta}{\Gamma}}{\cexists_\gamma \vbeta. \phi}
  }

  \inferrule[\cempty]{
    \centail{\theta}{\bot}
  }{
    \judge{\facts{\valpha}{\theta}{\Gamma}}{
      \phi
    }
  }

  \inferrule[\cdisj]{
    \judge{\facts{\valpha}{\theta}{\Gamma}}{\phi}\\
    \judge{\facts{\valpha}{\gamma}{\Gamma}}{\phi}
  }{
    \judge{\facts{\valpha}{\theta \vee \gamma}{\Gamma}}{\phi}
  }
\end{mathpar}

\paragraph{Rules}
\begin{mathpar}
  \inferrule[\defunroll{t_\tau}]{
    \decl{t_\sfx}
    {t_\epsilon,(t_\sfy^\act)_{\act \in \aset}} \in \decls\\
    \tau \in \valpha\\
    \tauo \not \in \valpha\\\\
    \judge
    {\facts{\valpha}
      {\theta\wedge\tau = \epsilon}
      {\Gamma \wedge t_\tau = t_\epsilon}}
    {\phi}\\
    \left(
      \judge
      {\facts{\valpha,\tauo}
        {\theta\wedge\tauo = \tpred{\tau}\wedge \tauo = \act}
        {\Gamma \wedge t_\tau = t_\tauo^\act}}
      {\phi}
    \right)_{\act \in \aset}
  }{
    \judge{\facts{\valpha}{\theta}{\Gamma}}{\phi}
  }

  \inferrule[\requ]{
    \judge{\facts{\valpha}{\theta}{\Gamma}}{\psi[s]}\\
    \judge{\facts{\valpha}
      {\theta}{\Gamma}}
    {s = t}
  }{
    \judge{\facts{\valpha}{\theta}{\Gamma}}{
      \psi[t]
    }
  }

  \inferrule[\papply]{
    \cexists_\gamma \vbeta. \psi
    \in \Gamma\\
    \valpha \cap \vbeta = \emptyset\\
    \judge{\facts{\valpha, \vbeta}{\theta \wedge \gamma}
      {\Gamma \cup \{\psi\}}}{\phi}
  }{
    \judge{\facts{\valpha}{\theta}{\Gamma}}{\phi}
  }

  \inferrule[\apply]{
    (\cforall_\gamma\, \vbeta.\;
    \psi
    \ra
    \phi_0)
    \in \Gamma\\
    \nu : \vbeta \mapsto \valpha\\
    \centail{\theta}{\gamma\nu}\\
    \judge{\facts{\valpha}{\theta}{\Gamma}}{\psi\nu}\\
    \judge{\facts{\valpha}{\theta}
      {\Gamma \cup \{\phi_0\nu\}}}{\phi}
  }{\judge{\facts{\valpha}{\theta}{\Gamma}}{\phi}}

  \inferrule[\induc]{
    \phi' \equiv
    \cforall_{\gammao} \tauo.\phi\\
    \gammao \equiv
    \subst{\gamma}{\tau}{\tauo}\wedge\tauo < \tau\\
    \tau,\tauo \not \in \valpha\\
    \judge
    {\facts{\valpha,\tau}{\theta\wedge\gamma}{\Gamma\cup\{\phi'\}}}
    {\phi}
  }{
    \judge
    {\facts{\valpha}{\theta}{\Gamma}}
    {\cforall_\gamma \tau.\phi}
  }
\end{mathpar}

\paragraph{Superpositions of $\clweaken$ or $\crstrengthen$ with other rules}
\begin{itemize}
\item $\mrule{\clweaken}$ with $\mrule{\printro}$:
  \[
    \inferrule[\printrogen]{
      \nu : \vbeta \mapsto \valpha\\
      \judge{\facts{\valpha}{\theta \wedge (\gamma\nu)}{\Gamma}}{\psi\nu}\\
      \judge{\facts{\valpha}{\theta \wedge \neg(\gamma\nu)}{\Gamma}}{\bot}
    }{
      \judge{\facts{\valpha}{\theta}{\Gamma}}{\cexists_\gamma \vbeta. \psi}
    }
  \]

\item $\mrule{\crstrengthen}$ with $\mrule{\apply}$ and $\mrule{\absurd}$:
  \[
    \inferrule[\botapply]{
      (
      \cforall_\gamma\, \vbeta.\;
      \psi
      \ra
      \bot
      ) \in \Gamma\\
      \nu : \vbeta \mapsto \valpha\\
      \judge{\facts{\valpha}{\theta \wedge (\gamma\nu)}{\Gamma}}{\psi\nu}\\
      \judge{\facts{\valpha}{\theta \wedge \neg(\gamma\nu)}{\Gamma}}{\phi}
    }{\judge{\facts{\valpha}{\theta}{\Gamma}}{\phi}}
  \]

\item $\mrule{\crstrengthen}$ with $\mrule{\apply}$ (general case):
  \[
    \inferrule[\botapply]{
      (
      \cforall_\gamma\, \vbeta.\;
      \psi
      \ra
      \phio
      ) \in \Gamma\\
      \nu : \vbeta \mapsto \valpha\\
      \judge{\facts{\valpha}{\theta \wedge (\gamma\nu)}{\Gamma}}{\psi\nu}\\
      \judge{\facts{\valpha}{\theta \wedge (\gamma\nu)}
        {\Gamma \cup \{ \phio\nu\}}}
      {\phi}\\
      \judge{\facts{\valpha}{\theta \wedge \neg(\gamma\nu)}{\Gamma}}{\phi}
    }{\judge{\facts{\valpha}{\theta}{\Gamma}}{\phi}}
  \]

\end{itemize}


\end{document}

%%% Local Variables:
%%% mode: latex
%%% TeX-master: t
%%% End:
