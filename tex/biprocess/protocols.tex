\section{Protocols}

\begin{itemize}
  \item $\N$ names with index arity
  \item $\X$ variables
  \item $\I$ index variables
  \item $\F$ function symbols with usual arity
  \item $\A$ action symbols with index arity
  \item $\C$ channel names
\end{itemize}

\begin{figure}[h]
  \[
    \begin{array}{rcll}
      t & := & \sfn(i_1 \dots i_n) \mid x \mid \sff(t_1,\dots,t_n) & \text{term}
      \\
      tt & := & t \mid \choice(t_1,t_2) \mid \sff(tt_1,\dots,tt_n) & \text{bi-term}
    \end{array}
  \]
  \caption{Syntax for bi-processes}
  \label{fig:syntax-bi-process}
\end{figure}

\begin{definition}
A \emph{symbolic action} $\alpha = \sfa(i_1,\dots,i_n)$, where $\sfa \in \A$ and $i_1,\dots,i_n \in \I$, is defined by:
\begin{itemize}
  \item a bi-term $\mtest_{\alpha}$ (intuitively, the condition of this action)
  \item a bi-term $\mout_{\alpha}$ (intuitively, the output of this action)
  \item two channel names $\cin_\alpha$ and $\cout_\alpha$ (intuitively, the channels used for the input and the output of this action)
\end{itemize}
where the only variable appearing in $\mtest_{\alpha}$ and $\mout_{\alpha}$ is $x_{\alpha}$ (intuitively, the input of this action).\footnote{TODO - Traiter le cas des variables d'index libres dans $\mtest_{\alpha}$ (try find).}
Names appearing in $\mtest_{\alpha}$ and $\mout_{\alpha}$ are of the form $\sfn(i_1,\dots,i_k)$ with $k \leq n$.
\end{definition}

\begin{definition}
A \emph{concrete action} is an instanciation of a \emph{symbolic action} by giving integer values to index variables.
\end{definition}

\begin{definition}
  \label{def:proto}
  A \emph{protocol} is defined by:
  \begin{itemize}
    \item a set of index variables $i_1,\dots,i_n$,
    \item a finite set of \emph{symbolic actions} with distinct action symbols featuring only index variables from $i_1,\dots,i_n$,
    \item a transitive \emph{sequential dependency} relation $\before$ on \emph{concrete actions},
    \item a \emph{conflict} relation $\conflict$ on \emph{concrete actions}.
  \end{itemize}
  Intuitively, $\sfa_1(1)\before\sfa_2(1,2)$ means that $\sfa_1(1)$ must happen before $\sfa_2(1,2)$ in any execution
  and $\sfa_1(1,2)\conflict\sfa_2(1,2)$ means that $\sfa_1(1,2)$ and $\sfa_2(1,2)$ cannot both occur in any execution.
\end{definition}

\begin{example}[Basic Hash]
  \label{ex:basic-hash-bi-process}
  We introduce the protocol $P_{BH}$ defined by:
  \begin{itemize}
    \item the set of index variables $\{i,j,ii,jj\}$,
    \item the set of symbolic actions $\{\T(i,j),\R_1(ii,jj),\R_2(ii,jj)\}$ where,
    \begin{itemize}
      \item action $\alpha_1 := \T(i,j)$
        \begin{itemize}
          \item $\mtest_{\alpha_1} := \true$
          \item $\mout_{\alpha_1} := \langle n(i,j), \h(\choice(k(i),k'(i)),n(i,j) \rangle$
        \end{itemize}
      \item action $\alpha_2 := \R_1(ii,jj)$
        \begin{itemize}
          \item $\mtest_{\alpha_2} := \eq(\snd(\g(x_{\alpha_2})),  \h(\choice(k(ii),k'(ii)),\fst(\g(x_{\alpha_2})))$
          \item $\mout_{\alpha_2} := \ok$
        \end{itemize}
      \item action $\alpha_3 := \R_2(ii,jj)$
        \begin{itemize}
          \item $\mtest_{\alpha_3} := \noteq (\snd(\g(x_{\alpha_3})),  \h(\choice(k(ii),k'(ii)),\fst(\g(x_{\alpha_3}))))$
          \item $\mout_{\alpha_3} := \error$
        \end{itemize}
    \end{itemize}
    \item a conflict relation $\conflict$ such that $\forall p,q \in \mathbb{N}, \R_1(p,q)\conflict\R_2(p,q)$.
  \end{itemize}
\end{example}

\begin{definition}
  Given a protocol $P$ according to \Cref{def:proto}, a \emph{trace} is a sequence $\sfa_1(\vect {i_1}) \dots \sfa_n(\vect {i_n})$ made of concrete actions such that:
  \begin{itemize}
    \item for all $1 \leq i,j \leq n$, if $\sfa_i = \sfa_j$ then $\vect {i_i} \neq \vect {i_j}$;
    \item for all $1 \leq i < j \leq n$, it is not the case that $\sfa_i(\vect {i_i}) \conflict \sfa_j(\vect {i_j})$;
    \item for all $1 \leq i \leq n$, for all $\sfa_k(\vect {i_k}) \before \sfa_i(\vect {i_i})$, there exists $1 \leq j < i$ such that $\sfa_k(\vect {i_k}) = \sfa_j(\vect {i_j})$;
    \item we cannot have in the trace $\sfa_i(\vect {i_{i_1}}.\vect {i_{i_2}}.\vect {i_{i_3}})$ and $\sfa_j(\vect {i_{j_1}}.\vect {i_{j_2}}.\vect {i_{j_3}})$
    with $\vect {i_{i_1}} \neq \vect {i_{j_1}}$ and $\vect {i_{i_2}} = \vect {i_{j_2}}$.
  \end{itemize}
\end{definition}

\begin{example}
  \label{ex:basic-hash-trace}
  Some traces for the protocol defined in \Cref{ex:basic-hash-bi-process}.
  \begin{itemize}
    \item $\T(1,1),\R_1(1,1)$
    \item $\T(1,1).\R_1(1,1).\T(1,2).\R_1(1,2)$
    \item $\T(1,1).\R_1(1,1).\T(2,1).\R_1(2,1)$
  \end{itemize}
\end{example}

\begin{definition}
  Given a trace $\tr = \alpha_1 \dots \alpha_n$, we define recursively the \emph{frame} $\mouts(\tr)$ of this trace:
  \begin{itemize}
    \item $\mouts(\alpha_1) := \{ \mout_{\alpha_1}\{x_{\alpha_1} \mapsto \varnothing\} \}$
    \item $\forall i, 2 \leq i \leq n, \mouts(\alpha_1 \dots \alpha_i) := \mouts(\alpha_1 \dots \alpha_{i-1}) \cup \{ \mout_{\alpha_{i-1}}\{x_{\alpha_{i-1}} \mapsto \mouts(\alpha_1 \dots \alpha_{i-1})\} \}$
  \end{itemize}
\end{definition}

\begin{definition}
  Given a trace $\tr = \alpha_1 \dots \alpha_n$, we define:
  $$\mtests_\tr := \mand(\{ \mtest_{\alpha_i}\{x_{\alpha_i}\mapsto\mouts(\alpha_1 \dots \alpha_{i-1})\} \}_{1 \leq i \leq n})$$
\end{definition}

\bigskip
\noindent
We use the notation $(\_)^L$ and $(\_)^R$ to denote the left and right projections (defined as usual), for example:
\begin{itemize}
  \item $(\mtest_{\alpha})^L$ and $(\mtest_{\alpha})^R$
  \item $(\mout_{\alpha})^L$ and $(\mout_{\alpha})^R$
  \item $(\mtests_{\tr})^L$ and $(\mtests_{\tr})^R$
  \item $(\mouts_{\tr})^L$ and $(\mouts_{\tr})^R$
\end{itemize}
