\section{Semantics for protocols}

For a given trace, definition of its left/right frame and its left/right executability formula (conjunction of all conditions) expressed as elements in the logic

% \begin{definition}
%   Given a symbolic trace $\tr = \alpha_1 \dots \alpha_n$, we define recursively the \emph{frame} $\mouts(\tr)$ of this trace:
%   \begin{itemize}
%     \item $\mouts(\alpha_1) := \{ \mout_{\alpha_1}\{x_{\alpha_1} \mapsto \varnothing\} \}$
%     \item $\forall i, 2 \leq i \leq n, \mouts(\alpha_1 \dots \alpha_i) := \mouts(\alpha_1 \dots \alpha_{i-1}) \cup \{ \mout_{\alpha_{i-1}}\{x_{\alpha_{i-1}} \mapsto \mouts(\alpha_1 \dots \alpha_{i-1})\} \}$
%   \end{itemize}
% \end{definition}
%
% \begin{definition}
%   Given a symbolic trace $\tr = \alpha_1 \dots \alpha_n$, we define:
%   $$\mtests_\tr := \mand(\{ \mtest_{\alpha_i}\{x_{\alpha_i}\mapsto\mouts(\alpha_1 \dots \alpha_{i-1})\} \}_{1 \leq i \leq n})$$
% \end{definition}
%
% \bigskip
% \noindent
% We use the notation $(\_)^L$ and $(\_)^R$ to denote the left and right projections (defined as usual), for example:
% \begin{itemize}
%   \item $(\mtest_{\alpha})^L$ and $(\mtest_{\alpha})^R$
%   \item $(\mout_{\alpha})^L$ and $(\mout_{\alpha})^R$
%   \item $(\mtests_{\tr})^L$ and $(\mtests_{\tr})^R$
%   \item $(\mouts_{\tr})^L$ and $(\mouts_{\tr})^R$
% \end{itemize}

\paragraph{Reasoning on protocols}

Meaning of:
\begin{itemize}
  \item a formula of the meta-logic is valid for a protocol
  \item a formula of the logic is valid for a trace
\end{itemize}

\paragraph{Indistinguishability}

Define
\begin{itemize}
  \item indistinguishability for a protocol
  \item the associated (meta-)logic formula(s)
\end{itemize}

% \begin{definition}
%   A bi-process is \emph{diff-equivalent} when, for any trace $\tr$ of this bi-process and for any computational model $\M$, we have:
%   $$\M \models \myif (\mtests_{\tr})^L \mythen (\mouts_{\tr})^L \sim \myif (\mtests_{\tr})^R \mythen (\mouts_{\tr})^R$$
% \end{definition}
%
% \begin{example}
%   For example, for the bi-process introduced in \Cref{ex:basic-hash-bi-process} and the trace introduced in \Cref{ex:basic-hash-trace}, we would like to verify that the following equivalence holds (amongst many others):
%
%   $$\forall \M, \M \models \myif (\mtests_{\tr})^L \mythen (\mouts_{\tr})^L \sim \myif (\mtests_{\tr})^R \mythen (\mouts_{\tr})^R$$
%
%   with:
%   \begin{itemize}
%     \item $(\mtests_{\tr})^L := \mand(\true, \eq(\snd(\g(\langle n(i,j), \h(k(i),n(i,j)) \rangle)), \h( k(i), \fst( \g( \langle n(i,j), \h(k(i),n(i,j)) \rangle )))))$
%     \item $(\mouts_{\tr})^L := \{ \langle n(i,j), \h(k(i),n(i,j) \rangle, \ok \}$
%     \item $(\mtests_{\tr})^R := \mand(\true, \eq(\snd(\g(\langle n(i,j), \h(k'(i),n(i,j)) \rangle)), \h( k'(i), \fst( \g( \langle n(i,j), \h(k'(i),n(i,j)) \rangle )))))$
%     \item $(\mouts_{\tr})^R := \{ \langle n(i,j), \h(k'(i),n(i,j) \rangle, \ok \}$
%   \end{itemize}
% \end{example}
