\section{Logic}

The base logic is that of Bana and Comon (CCS'14). It is first-order logic
with a distinguished kind of constant for representing names (written
$\mathsf{n}$, $\mathsf{m}$, $\mathsf{k}$, etc.)
and featuring a single predicate noted $\sim$ (or rather, a family
of predicates $\sim_k$ of arity $2 k$ for all $k\in\mathbb{N}$).
The terms are all of the same sort: they are meant to represent
messages\footnote{
  \david{I hope/believe that the distinction between booleans and messages
  can be ignored for the theory.}
}.

In practice, the logic will be considered with function symbols for
representing cryptographic primitives as well as attacker computations, and
boolean constants and connectives.

The formulas of the base logic are intended to be interpreted in
\emph{computational models}, where:
\begin{itemize}
  \item terms are interpreted as PPT Turing machines which,
    given infinite random tapes and a security parameter output a bitstring;
  \item $\sim$ is interpreted as indistinguishability;
  \item names are independent random samplings;
  \item function symbols correspond to deterministic machines.
\end{itemize}

In particular, function symbols corresponding to cryptographic primitives
are interpreted as implementations of the primitives, subject to some
assumptions.

For example, if $\mathsf{n}$ and $\mathsf{m}$ are distinct names,
the (atomic) formulas $\mathsf{n}\sim\mathsf{m}$ and
$\mathsf{n}\stackrel{.}{=}\mathsf{m}\sim\mathsf{false}$
are valid.
