\section{Trace properties}

We want to verify that a formula $\phi$ (the security property)
holds for all executions of some protocol, in all computational models
satisfying some cryptographic assumptions.
This can be guaranteed if we have $Ax\models\phi$
where $Ax$ is a (recursive) set of axioms that are sound
wrt.\ the intended class of meta-interpretations and computational models.
We might be able to formulate such a set $Ax$ (which would consist of
two parts, first the usual BC axioms, second some meta-logic axioms
constraining the meta-interpretations to correspond to protocol executions).
However, we will go for an easier task: designing a set of inference
rules that allow to derive meta-logic formulas (rather, sequents)
that are valid in all intended models.


We can now formally define when a protocol satisfies some trace property
$\phi$:
we require that the property holds in all traces that are actually executable
--- more precisely, we require that $\phi_\tr \Rightarrow \phi$ is almost
always true, where $\phi_\tr$ expresses that the trace is executable.

\begin{definition}
  Given a trace $\tr$, its executability formula is
  $$\phi_\tr = \bigwedge_{\alpha\in\tr}
  \phi_\alpha\{x_\alpha\mapsto\minp@\alpha,x^s_\alpha\mapsto s@\alpha\}$$.
\end{definition}

\begin{definition}
  A meta-logic formula $\phi$ is valid wrt.\ our protocol when,
  for any symbolic trace $\tr$ of the protocol,
  we have $I_\tr \models \phi_\tr \Rightarrow \phi$.
\end{definition}

In other words, we require that, for any $\tr$ and
for any computational model,
we have $\M \models (\phi_\tr\Rightarrow\phi)^{I_\tr} \sim \true$.

In practice, we do not want to consider all computational models, but
restrict to the ones satisfying some cryptographic assumption: reformulate
what's above so that it's more natural.

\subsection{Sequent calculus}

In the tool, sequents come with an environment which explicitly
declares all the variables that might occur free in the sequent's formulas.
For simplicity, we do not include them here. We shall write $\vdash t:T$
when $t$ is a term of sort $\T$ (which might be $\idx$ or $\timestamp$).

\begin{definition}
  A sequent $\Gamma \vdash \phi$ is composed of a set of meta-logic formulas
  $\Gamma$ and a meta-logic formula $\phi$.
  It is valid when the meta-logic formula
  $(\wedge\Gamma) \Rightarrow \phi$ is valid.
\end{definition}

\begin{proposition}
  The rules of \cref{fig:lk,fig:names} are sound: if the premisses are valid,
  then so is the conclusion.
\end{proposition}

\begin{figure}
  \begin{mathpar}
    \inferrule[Axiom]{~}{\Gamma,\phi\vdash\phi}
    \quad\quad
    \inferrule[Cut]{
      \Gamma \vdash \phi
      \quad
      \Gamma,\phi \vdash \psi
    }{
      \Gamma \vdash \psi
    }
  \end{mathpar}
  \begin{mathpar}
  \inferrule[${\wedge}$-L]{
    \Gamma,\phi,\phi'\vdash\psi
  }{
    \Gamma,\phi\wedge\phi'\vdash\psi
  }
  \quad\quad
  \inferrule[${\wedge}$-R]{
    \Gamma \vdash \phi
    \quad
    \Gamma \vdash \phi'
  }{
    \Gamma \vdash \phi\wedge\phi'
  }
  \end{mathpar}
  \begin{mathpar}
    \inferrule[${\lnot}$-R]{
      \Gamma, \phi \vdash \bot
    }{
      \Gamma \vdash \lnot\phi
    }
    \quad\quad
    \inferrule[${\lnot}$-L]{
      \Gamma \vdash \phi
    }{
      \Gamma,\lnot\phi \vdash \psi
    }
    \quad\quad
    \inferrule[Raa]{
      \Gamma, \lnot\phi \vdash \bot
    }{
      \Gamma \vdash \phi
    }
  \end{mathpar}
  \begin{center}
  \emph{other propositional rules of classical sequent calculus}
  \end{center}
  \begin{mathpar}
      \inferrule[${=}$-R]{~}{\Gamma \vdash t=t} \quad\quad
      \inferrule[${=}$-L]{
        (\Gamma\vdash\phi)\{x\mapsto t',x'\mapsto t\}
      }{
        t=t', \Gamma\{x\mapsto t,x'\mapsto t'\} \vdash
        \phi\{x\mapsto t,x'\mapsto t'\}
      }
  \end{mathpar}
  \begin{mathpar}
     \inferrule[$\forall$-L]{
       \Gamma,\phi\{x\mapsto t\} \vdash \psi
       \quad
       \vdash t:\T
     }{
       \Gamma,\forall x:\T.\phi \vdash \psi}
     \quad\quad
     \inferrule[$\forall$-R]{
       \Gamma \vdash \phi}{\Gamma \vdash \forall x:\T:\phi}
  \end{mathpar}
   \caption{Generic inference rules. These are the rules of classical
   first-order sequent calculus. In \textsc{$\forall$-R} we require that
   $x$ does not appear free in $\Gamma$.}
   \label{fig:lk}
\end{figure}

\begin{figure}
  \begin{mathpar}
  \inferrule{
    \mathsf{n}\neq\mathsf{m}
  }{
    \Gamma,\mathsf{n}[\vec i]=\mathsf{m}[\vec j]\vdash \phi
  }
  \quad\quad
  \inferrule{
    \Gamma, i_1=j_1, \ldots, i_k=j_k \vdash \phi
  }{
    \Gamma,\mathsf{n}[i_1,\ldots,i_k]=\mathsf{n}[j_1,\ldots,j_k]\vdash \phi
  }
  \end{mathpar}
  \begin{mathpar}
    \inferrule{~}{
      \Gamma,t=\mathsf{n}[\vec i] \vdash \phi
    }
  \end{mathpar}
  \caption{Inference rules for equalities on names.
  The last rule only applies when, for any meta-interpretation $I$,
  the term $(t)^I$ does not contain any message variable and does
  not contain any occurrence of $\mathsf{n}_{\sigma_\I(\vec i)}$
  --- this condition will be over-approximated in our implementation.}
  \label{fig:names}
\end{figure}

\begin{proposition}
  The rules of \cref{fig:xor} are sound wrt.\ computational models
  where $\oplus$ is interpreted as exclusive or:
  if the premisses are valid in all such computational models,
  then so is the conclusion.
  \emph{Do we need to impose a condition on message lengths?}
\end{proposition}

\begin{figure}
  \begin{mathpar}
    \inferrule{~}{\Gamma \vdash t\oplus t = 0}
    \quad\quad
    \inferrule{~}{\Gamma \vdash t\oplus t' = t'\oplus t}
    \quad\quad
    \inferrule{~}{\Gamma \vdash t\oplus (t'\oplus t'') = (t\oplus t')\oplus t''}
  \end{mathpar}
  \begin{mathpar}
    \inferrule{
      (\Gamma \vdash \phi)\{x\mapsto \mathsf{m}[\vec j]\}
    }{
      (\Gamma \vdash \phi)\{x\mapsto t\oplus\mathsf{n}[\vec i]\}
    }
  \end{mathpar}
  \caption{Inference rules for exclusive or.
  The tool does not follow these rules closely, but uses a more high-level
  congruence closure algorithm modulo xor. The last rule is not implementated
  at all and is here mostly for illustration purposes, and is subject to
  the condition that, for any meta-interpretation $I$,
  $(\Gamma\vdash\phi)^I$ does not contain instances of the names
  $\mathsf{n}$ and $\mathsf{m}$ and does not contain message variables.
  This condition would be over-approximated in an implementation.}
  \label{fig:xor}
\end{figure}


\subsection{Trace axioms}

We now identify some valid formulas, which can thus be used as axioms
when trying to derive a trace property.

\begin{proposition}
  The following formulas are valid wrt.\ the protocol:
  \begin{itemize}
    \item for any $\alpha \before \beta$,
      $\happens(\beta) \Rightarrow \happens(\alpha) \wedge \alpha < \beta$;
    \item for any $\alpha \conflict \beta$,
      $\happens(\alpha)\wedge\happens(\beta) \Rightarrow \bot$;
    \item for any $\alpha$,
      $\happens(\alpha) \Rightarrow
      \phi_\alpha\{x_\beta\mapsto\minp@\beta\}_\beta$.
    \item
      $\forall \tau,\tau',\tau'', \happens(\tau)\wedge\happens(\tau'')\wedge
      \tau\leq\tau'\wedge\tau'\leq\tau''\Rightarrow\happens(\tau')$;
    \item
      $\forall \tau, \happens(\tau) \Rightarrow \pre(\tau)<\tau \wedge
      \forall \tau'.~ \pre(\tau)\leq\tau'<\tau \Rightarrow
      \tau'=\pre(\tau)$;
    \item for any $a\neq b\in\Actions$,
      $\forall \ov{i},\ov{j}, a(\ov{i})\neq b(\ov{j})$;
    \item for any $a\in\Actions$,
      $\forall \ov{i},\ov{j}, a(\ov{i})=a(\ov{j}) \Rightarrow \ov{i}=\ov{j}$.
  \end{itemize}
\end{proposition}

Note that obvious axioms such as the predecessor or injectivity axioms
would not hold for simpler constructions of $I_\tr$.
