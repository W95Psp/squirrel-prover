\documentclass[table,svgnames]{beamer}
\usetheme{metropolis}           % Use metropolis theme
%\usetheme[numbering=fraction,block=fill,background=dark]{metropolis}

\usepackage{graphicx}
\usepackage{listings}
\usepackage{fontawesome}
\usepackage{pifont}
\usepackage{multicol}
\definecolor{Background}{HTML}{FAFAFA}
\definecolor{Orange}{HTML}{ED923A}
\definecolor{Red}{HTML}{CB2C2C}
%\usepackage{pgfpages}
\newcommand<>{\red}[1]{\textcolor#2{Red}{#1}}
\newcommand<>{\orange}[1]{\textcolor#2{Orange}{#1}}
\newcommand<>{\blue}[1]{\textcolor#2{blue}{#1}}

\newcommand<>{\barre}[1]{
  \alt#2{\beameroriginal{\sout}{#1}}{#1}
}



\usepackage[T1]{fontenc}
\usepackage[utf8]{inputenc}
%\usepackage{a4wide}
%\usepackage{lmodern}
\PassOptionsToPackage{normalem}{ulem}
\usepackage{ulem}
\usepackage[english]{babel}
\usepackage{amsmath}
\usepackage{stmaryrd}
%\usepackage{msc5}
\usepackage{amssymb}
\usepackage{amsthm}
\usepackage{mathtools}
\usepackage{mathpartir}
%%\usepackage{proof}
\usepackage{thmtools, thm-restate}
\usepackage{todonotes}
\usepackage{mathtools}
\usepackage{listings}


\usepackage{color,soul}
\usepackage{hyperref}
\usepackage{url}
\usepackage{wasysym}
\usepackage{xspace}

\usepackage[backend=bibtex,natbib=true]{biblatex}
\usepackage{listings}% http://ctan.org/pkg/listings
\lstset{
  basicstyle=\ttfamily,
  mathescape
}
%% Comments
\newenvironment{comment}{\color{red}}{}
\newcommand{\margincomment}[1]{\marginpar{\scriptsize\textcolor{red}{#1}}}



%%Formatting stuff
\newcommand{\ov}[1]{\overline{#1}}

%%Tools/languages names
\newcommand{\Sapic}{\textsc{Sapic}\xspace}
\newcommand{\Slapic}{\textsc{Slapic}\xspace}
\newcommand{\Tamarin}{\textsc{Tamarin}\xspace}

%%Languages constructs
\DeclareMathOperator{\inp}{in}
\DeclareMathOperator{\outp}{out}


%%Function symbols and constants
\newcommand{\checkreport}{\mathsf{check}}
\newcommand{\report}{\mathsf{report}}
\newcommand{\sk}{\mathsf{sk}}
\newcommand{\sign}{\mathsf{sign}}
\newcommand{\Locs}{\mathcal{L}}
\newcommand{\HLoc}{\mathsf{HLoc}}
\newcommand{\Fq}{\mathbb{F}_q}
\newcommand{\K}{\mathbb{K}}

%%Operators on languages
\DeclareMathOperator{\tr}{tr}

%%Rewriting stuff
\newcommand{\nf}{\!\downarrow}



% Logic
\newcommand{\se}[2]{ \textsf{[}\;#1 : #2 \;\textsf{]}}
\newcommand{\ju}[3][p]{ \textsf{[}\;#2 \;\textsf{]}:_{#1} #3}
\newcommand{\rname}[1]{{\text{\ensuremath{#1}}}}
\newcommand{\rcond}[1]{{\text{\fbox{#1}}}}
\newcommand{\rcondbig}[3]{{\text{\fbox{\hspace{-1ex}\begin{minipage}{#1}%
   \vspace{#2} \hspace{-2ex}%
  \begin{displaymath}%
  \begin{array}{l}%
  #3%
  \end{array}%
  \end{displaymath}%
  \end{minipage}}}}}
\newcommand{\EAx}{\mathcal{E}}
\newcommand{\eqE}[2]{#1 =_\EAx #2}
\newcommand{\eqEA}[3]{#1 \models #2 =_{\EAx} #3}
\newcommand{\dedE}[2]{#1 \vdash_{\EAx} #2}
\newcommand{\dedEwith}[3]{#1 \vdash_{\EAx}^{#2} #3}
\newcommand{\dedEA}[3]{#1 \models #2 \vdash_{\EAx} #3}
\newcommand{\dedEAwith}[4]{#1 \models #2 \vdash_{\EAx}^{#3} #4}
\newcommand{\compE}[2]{#1 \mathbin{\simeq_{\EAx}} #2}
\newcommand{\compEA}[3]{ #1 \models #2 \mathbin{\simeq_{\EAx}} #3}
\newcommand{\spcomp}{\mathit{sp}}
\newcommand{\conseq}{\mathit{conseq}}
\newcommand{\noquant}{\mathit{nquant}}
\renewcommand{\sp}[2]{\mathit{sp}_{#2}(#1)}
\newcommand{\pleq}{\preceq}
\newcommand{\peq}{\doteq}
\newcommand{\peql}{\leqq} % peq or pleq

\newcommand{\Pre}[3]{\mathop{\textsf{[}\;#2 : #3 \;\textsf{]}}\nolimits_{#1}}
\newcommand{\Pres}[2]{\mathop{\textsf{[}\;#2\;\textsf{]}}\nolimits_{#1}}
\newcommand{\PreSym}[1]{\mathop{\Pr\nolimits_{#1}}}
\newcommand{\tsucc}{\textsf{succ}}
\newcommand{\tadv}{\textsf{adv}}
\newcommand{\PrSucc}[2]{\Pre{\tsucc}{#1}{#2}}
\newcommand{\PrAdv}[2]{\Pre{\tadv}{#1}{#2}}
\newcommand{\PrDist}[4]{\Pre{\se{#1\;}{\;#2}}{#3}{#4}}


% rules
\newcommand{\ruleM}[1]{\textsc{#1}}
\newcommand{\Unfold}{\ruleM{Unfold}}
\newcommand{\Insert}{\ruleM{Add}}
\newcommand{\Subst}{\ruleM{Subst}}
\newcommand{\Swap}{\ruleM{Swap}}
\newcommand{\Oswap}{\ruleM{OSwap}}
\newcommand{\EvCtx}{\ruleM{EvCtx}}
\newcommand{\Assert}{\ruleM{Assert}}

\newcommand{\DistSym}{\ruleM{Sym}}
\newcommand{\DistRefl}{\ruleM{Refl}}
\newcommand{\Trans}{\ruleM{Trans}}
\newcommand{\DistEq}{\ruleM{Deq}}

\newcommand{\Rnd}{\ruleM{Rnd}}
\newcommand{\RndStar}{\ruleM{Rnd}^{\star}}
\newcommand{\Simp}{\ruleM{Simp}}
%\newcommand{\IfEq}{\ruleM{IfEq}}
\newcommand{\Except}{\ruleM{Except}}

\newcommand{\CaseEv}{\ruleM{CaseEv}}
\newcommand{\FalseEv}{\ruleM{FalseEv}}
\newcommand{\LeqEq}{\ruleM{LeqEq}}

\newcommand{\AbsorbAdv}{\ruleM{AbsorbAdv}}
\newcommand{\AbsorbAdvO}{\ruleM{AbsorbAdv}_O}
\newcommand{\CallAdv}{\ruleM{CallAdv}}

\newcommand{\Upto}{\ruleM{Upto}}
\newcommand{\Hybrid}{\ruleM{Hybrid}}

\newcommand{\Abstract}{\ruleM{Abstract}}
\newcommand{\Guess}{\ruleM{Guess}}
\newcommand{\Find}{\ruleM{Find}}
\newcommand{\Indep}{\ruleM{Indep}}

\newcommand{\ifte}[3]{#1 ? #2 \mathop{:} #3}


%% logical operations
\newcommand{\cchoose}[3]{#1\,?\,#2:#3}
\newcommand{\cfalse}{\textsf{false}}
\newcommand{\ctrue}{\textsf{true}}


% commands in games
\newcommand{\samp}[2]{#1 \leftarrow\!\!\!\!\!^{\$}\  #2} %FIXME
\newcommand{\ass}[2]{#1 \leftarrow #2}
\newcommand{\Aasym}{\ensuremath{\textsf{A}}}
\newcommand{\Basym}{\ensuremath{\textsf{B}}}
\newcommand{\osym}{\ensuremath{\textsf{o}}}
\newcommand{\boundfun}{\delta}
\newcommand{\boundsym}[1]{\ensuremath{\textsf{n}_{#1}}}

\newcommand{\Adv}{\ensuremath{\mathcal{A}}}
\newcommand{\Bdv}{\ensuremath{\mathcal{B}}}
\newcommand{\Advs}{\mathbf{A}}
\newcommand{\Ora}{\mathcal{O}}
\newcommand{\clet}[2]{\textsf{let } #1 \mathop{\textsf{=}} #2}
\newcommand{\clets}[2]{#1 = #2}
\newcommand{\cif}{\textsf{if }}
\newcommand{\celif}{\textsf{elif }}
\newcommand{\oless}{\textsf{bif}^{<}}
\newcommand{\oeq}{\textsf{bif}^{=}}
\newcommand{\ogreater}{\textsf{bif}^{>}}
\newcommand{\cwith}{\textsf{with }}
\newcommand{\cthen}{\textsf{then }}
\newcommand{\celse}{\textsf{else }}
\newcommand{\creturn}{\textsf{return }}
\newcommand{\cskip}{\textsf{skip }}
\newcommand{\cguard}{\textsf{guard}}
\newcommand{\cassert}{\textsf{assert}}
\newcommand{\cvar}[2]{\textsf{var }#1\;:\;#2}
\newcommand{\crvar}[2]{\textsf{rvar }#1\;:\;#2}
\newcommand{\chash}[2]{\textsf{hash }#1\;:\;#2}
\newcommand{\chashpriv}[2]{\textsf{phash }#1\;:\;#2}
\newcommand{\cperm}[2]{\textsf{perm }#1\;:\;#2}
\newcommand{\csync}[3]{\textsf{sync}^{#1}_{#2}(#3)}
\newcommand{\Prob}[2]{\Pr_{#1}[\;#2\;]}
%\newcommand{\cdo}{\textsf{do }}
%\newcommand{\cguard}{\textsf{guard}}
\newcommand{\lcomp}{[\;}
\newcommand{\rcomp}{\;]}
\newcommand{\compmid}{\mid}
\newcommand{\csep}{;\;}
\newcommand{\lsep}{,\;}
\newcommand{\bound}[1]{\delta_{#1}}
\newcommand{\counter}[1]{c_{#1}}
\newcommand{\queries}[1]{Q_{#1}}


% contexts and positions
\newcommand{\hole}{\Box}
\newcommand{\subst}[2]{#1 \{ #2 \}}
% \newcommand{\plug}[2]{\subst{#1}{\hole \mapsto #2}}
\newcommand{\plug}[2]{#1\{#2\}}
\newcommand{\setpos}[3]{#1\{ #2 \}_{#3}}
\newcommand{\getpos}[2]{#1 |_{#2}}

\newcommand{\pwhile}{\textsc{PWhile}}

% math
\newcommand{\dom}{\textsf{dom}}
\newcommand{\var}{\mathit{var}}
%\newcommand{\st}[1][]{\mathit{st}_{#1}}
\newcommand{\dunion}{\uplus}
\newcommand{\Dunion}{\biguplus}
% functor application
\newcommand{\fapp}[2]{#1[#2]}


% functions
\newcommand{\concl}{\mathit{conc}}
\newcommand{\probsem}[1]{\mathit{prob}_{#1}}
\newcommand{\rng}{\mathit{rng}}
\newcommand{\card}{\mathit{card}}
\newcommand{\asyms}{\mathit{asym}}
\newcommand{\osyms}{\mathit{osym}}
\newcommand{\vars}{\mathit{var}}
\newcommand{\poly}{\mathit{poly}}
\newcommand{\valid}{\mathit{valid}}
\newcommand{\splitExists}{\mathit{splitExists}}

% unsorted
\newcommand{\emap}{\hat{e}}
\newcommand{\autogp}{\textsf{AutoG\&P}\xspace}


\newcommand{\mal}{\mathcal{M}}
\newcommand{\ro}{\mathcal{RO}}
\newcommand{\rw}{\mathcal{RW}}
\newcommand{\chan}[1]{{\sf #1}}
\newcommand{\mali}[2]{$\mal^{\chan{#1}}_{{\sf in:}#2}$}
\newcommand{\malo}[2]{$\mal^{\chan{#1}}_{{\sf out:}#2}$}
\newcommand{\malio}[3]{$\mal^{\chan{#1}}_{{\sf in:}#2,out:#3}$}
\newcommand{\maliom}[2]{$\mal^{\chan{#1}}_{{\sf io:}#2}$}
%Scenario display
\newcommand{\PH}{\textsc{PH}}
\newcommand{\FS}{\textsc{FS}}
\newcommand{\NC}{\textsc{NC}}


%%Languages constructs
\newcommand{\myif}{\textbf{if}\;}
\newcommand{\mythen}{\;\textbf{then}\;}
\newcommand{\myelse}{\;\textbf{else}\;}
\newcommand{\myevent}{\textbf{event}\;}
\newcommand{\new}{\textbf{new}}
\newcommand{\cin}{\textbf{in}}
\newcommand{\cout}{\textbf{out}}
\newcommand{\mylet}{\textbf{let}\;}
\newcommand{\myin}{\textbf{in}\;}

\newcommand{\gStep}{\textit{Google 2-step}\xspace}
\newcommand{\fido}{\textit{FIDO's U2F}\xspace}
\newcommand{\su}{\textit{Silent-U2F}\xspace}

   %% short version
\newcommand{\prot}[1]{\textsc{#1}\xspace}


\newcommand{\gV}{\prot{g2V}}
\newcommand{\gVD}{\prot{g2V$^{\sf dis}$}}
\newcommand{\gVFPR}{\prot{g2V$^{\sf fpr}$}}

\newcommand{\gST}{\prot{g2OT}}
\newcommand{\gSTFPR}{\prot{g2OT$^{\sf fpr}$}}
\newcommand{\gSTD}{\prot{g2OT$^{\sf dis}$}}


\newcommand{\gDTFPR}{\prot{g2DT$^{\sf fpr}$}}
\newcommand{\gDTD}{\prot{g2DT$^{\sf dis}$}}

\newcommand{\uF}{\prot{U2F}}
\newcommand{\tb}{\prot{TokenBinding}}

\newcommand{\tbu}{\prot{U2F$_{\sf tb}$}}
\newcommand{\tbg}{\prot{g2DT$^{\sf dis}_{\sf tb}$}}


% \newcommand{\gV}{\prot{g2V}}
% \newcommand{\gVD}{\prot{g2V$^{\sf dis}$}}
% \newcommand{\gST}{\prot{g2ST}}
% \newcommand{\gSTD}{\prot{g2ST$^{\sf dis}$}}
% \newcommand{\gDTD}{\prot{g2DT$^{\sf dis}$}}
% \newcommand{\uF}{\prot{U2F}}
% \newcommand{\SuF}{\prot{SoftU2F}}

\newcommand{\sku}{\prot{U2F$_{\sf su}$}}
\newcommand{\skg}{\prot{g2DT$^{\sf dis}_{\sf su}$}}

%%Results display
\newcommand{\cmark}{\textcolor{Lime}{\ding{51}}}
\newcommand{\bigcmark}{\textcolor{Lime}{\ding{52}}}
\newcommand{\greycmark}{\textcolor{Grey}{\ding{51}}}
\newcommand{\bigbluecmark}{\textcolor{Blue}{\ding{52}}}
\newcommand{\biggreycmark}{\textcolor{Grey}{\ding{52}}}
\newcommand{\bluecmark}{\textcolor{Blue}{\ding{51}}}
\newcommand{\qmark}{\textcolor{Grey}{\ding{51}}}
\newcommand{\xmark}{\textcolor{Red}{\ding{55}}}
\newcommand{\bigxmark}{\textcolor{Red}{\ding{54}}}
\newcommand{\greyxmark}{\textcolor{Grey}{\ding{55}}}
\newcommand{\biggreyxmark}{\textcolor{Grey}{\ding{54}}}
\newcommand{\cannot}{?}


\newcommand{\mysig}{\;\texttt{sign}}
\newcommand{\myenc}{\;\texttt{enc}}
\newcommand{\checksign}{\;\texttt{checksign}}
\newcommand{\getmess}{\;\texttt{getmess}}
\newcommand{\St}{\;\texttt{St}}
\newcommand{\mypref}{\;\texttt{pref}}
\newcommand{\Or}{\mathcal{O}}
\newcommand{\ind}[1]{\cong_{#1}}
%%% Local Variables: 
%%% mode: latex
%%% TeX-master: "symbolic-iee"
%%% End: 

\usetikzlibrary{tikzmark,positioning}

\newcommand*{\yellowemph}[1]{%
\tikz[baseline]\node[rectangle, fill=yellow, rounded corners, inner sep=0.3mm,anchor=base]{#1};%
}

\usepackage{tkz-graph}
\GraphInit[vstyle = Empty]
\tikzset{
  LabelStyle/.style = {
                        minimum width = 2em, fill = Background,
                        text = Orange, },
  VertexStyle/.append style = { fill=Background,   text=black,
                                font = \fontsize{1.5cm}{5cm}},
  EdgeStyle/.append style = {->, bend left=15} }


\usepackage[absolute,overlay]{textpos}
%\usepackage{msc5}
\addtobeamertemplate{footnote}{}{\vspace{2ex}}




\title{Squirrel Prover}


\begin{document}

%\begin{frame}
%   \tikz [remember picture,overlay]
%    \node at
%        ([yshift=-2cm]current page.north)
%        %or: (current page.center)
%        {\includegraphics[width=\textwidth]{bck.png}};
%
%        \phantom{$$ toto $$}
%   \titlepage
%\end{frame}



 \maketitle

 \begin{frame}{The BC model in 30 secondes}
   \begin{block}{BC}
     A logic built over terms and a predicate $\sim$, where
\begin{itemize}
  \item terms are interpreted as PPT Turing machines which, \\
    given infinite random tapes and a security parameter, \\
    output a bitstring;
  \item $\sim$ is interpreted as computational indistinguishability;
  \item names are independent random samplings;
  \item function symbols correspond to deterministic machines.
\end{itemize}


\end{block}
\end{frame}
 \begin{frame}{The BC model in 30 seconds}
\begin{block}{Examples}
  Let $n,n'$ be names, then
  \begin{itemize}
    \pause
  \item     $n \sim n'$
    \pause
  \item $ (n \stackrel{.}{=}n') \sim \textsf{false}$
  \end{itemize}
\pause
With a function symbol $h$, and any terms $t_1,t_2$,
\[(t_1 \sim t_2 \Rightarrow h(t_1) \sim h(t_2))\]
\end{block}

\end{frame}


\begin{frame}{The tool}

  \begin{alertblock}{Squirrel Prover}
    Design a tool, allowing to prove properties (reachability and equivalence) about protocols that may contain an arbitrary number of replications.
  \end{alertblock}
\pause
\centering
 What does a protocol look like ? (cf. demo)

\end{frame}


\begin{frame}{The meta logic}
  We have a classical syntax, extended with
  \begin{itemize}
  \item names $n(i_1,....i_k)$ parameterized by indices inside a set $\mathcal{I}$;
  \item timestamps variables $\tau$, to quantify over all possible instants of a trace;
  \item macros $\textsf{cond}@\tau,\textsf{input}@\tau,\textsf{ouput}@\tau$ to talk about the condition, input and ouput of the action at instant $\tau$.
  \end{itemize}

  The logic allows for quantifications over timestamps and indices.
  \pause
\centering
 What kind of formula do we want to prove ?(cf. demo)

\end{frame}


\begin{frame}{Trace models}
  \begin{block}{}

  To interpret formulas that contain quantifications over timestamps and
  indices, we define trace models based on a protocol, that extends the BC computational models by providing:
  % DAVID je ne dirais pas "extends", c'est une notion supplémentaire
  % dans notre théorie
  \begin{itemize}
  \item a finite index domain $\mathcal{D}_{\mathcal{I}} \subseteq \mathbb{N}$;
    \item an interpretation of index variables $\sigma_{\mathcal{I}} :\mathcal{I} \mapsto \mathcal{D}_{\mathcal{I}} $;
    \item a domain for timestamps $\mathcal{D}_{\mathcal{T}}$, inside concrete actions of the protocol, and with a total ordering on the domain;
      % DAVID "inside" ?
    \item an interpretation of index variables $\sigma_{\mathcal{T}} :\mathcal{T} \mapsto \mathcal{D}_{\mathcal{T}} $.
  \end{itemize}

    \end{block}
\end{frame}


\begin{frame}{Going to BC}
  \begin{block}{}
Then, given a trace model $T$, we can define an interpretation $()^T$going
    from a formula of the meta logic to term of BC, with for instance
\begin{itemize}
\item $(\forall \tau. \phi)^T := (\bigwedge_{t\in \mathcal{D}_{\mathcal{T}}} \phi)^{T[\tau \mapsto t]}$
\item $(\exists \tau. \phi)^T := (\bigvee_{t\in \mathcal{D}_{\mathcal{T}}} \phi)^{T[\tau \mapsto t]}$
  \item \ldots
\end{itemize}
\end{block}

\pause
A formula $\phi$ is true for a trace model $T$, denoted $T \models \phi$, if \[M \models \phi^I \sim true\]\\ \pause
% DAVID qui est M ?
A formula $\phi$ is true for a given protocol, if it holds for all trace models of the protocol.
\end{frame}

\begin{frame}{What's left}
  \begin{block}{}
We design a sequent calculus allowing to reason about the validity of a formula w.r.t. a protocol. \\
\pause
Each rule of the sequent can become a tactic of the tool.
\pause
And that's all folks !
\pause
Let's try to prove things...
\end{block}
\end{frame}

\begin{frame}{Details}
  \begin{block}{Names}
We assume that the set of names $\mathcal{N}$ is such that for each name $n \in \mathcal{N}$, we also have an infinite set of names ${n_i}_{i \in \mathbb{N}} \subset \mathcal{N}$.
\[(n_i)^I = n_{i\sigma_{\mathcal{I}}}\]

\end{block}
\end{frame}
\begin{frame}{Details}
  \begin{block}{Macros}
We assume that the set of names $\mathcal{N}$ is such that for each name $n \in \mathcal{N}$, we also have an infinite set of names ${n_i}_{i \in \mathbb{N}} \subset \mathcal{N}$.

  \begin{eqnarray*}
  \mexec^\TM(\tinit) &=& \true \\
  \mexec^\TM(\alpha'_{k+1}) &=&
    \mcond^\TM(\alpha'_{k+1})
    \stackrel{.}{\wedge}
    \mexec^\TM(\alpha'_k)
\end{eqnarray*}


\[\begin{array}{rll}
    \mframe^\TM(\alpha'_{k+1}) &=&
                                   \begin{array}{l@{~}l}
    \langle &\mframe^\TM(\alpha'_k),\\
   & \myif \mexec^\TM(\alpha'_{k+1}) \mythen \mout^\TM(\alpha'_{k+1}) \myelse 0
                                     \rangle
                                     \end{array}
    \\
    \minp^\TM(\alpha'_{k+1}) &=&
    g_{k}(\mframe^\TM(\alpha'_k))
  \\

\end{array}\]

\end{block}

\end{frame}

\end{document}


%%% Local Variables:
%%% TeX-master: "slides"
%%% End:
