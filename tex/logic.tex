We call \emph{base logic} the system of Bana and Comon (CCS'14).
It is first-order logic with a distinguished kind of constant for representing
names (written $\mathsf{n}$, $\mathsf{m}$, $\mathsf{k}$, etc.)
and featuring a single predicate noted $\sim$ (or rather, a family
of predicates $\sim_k$ of arity $2 k$ for all $k\in\mathbb{N}$).
The terms are all of the same sort: they are meant to represent
messages.

In practice, the logic will be considered with function symbols for
representing cryptographic primitives as well as attacker computations.
We will also assume some standard boolean constants and operations, with
a fixed semantics (for instance, $\_ \stackrel{.}{=} \_$ or
$\mathsf{if \_ then \_ else \_}$).

The formulas of the base logic are intended to be interpreted in
\emph{computational models}, where:
\begin{itemize}
  \item terms are interpreted as PPT Turing machines which,
    given infinite random tapes and a security parameter output a bitstring;
  \item $\sim$ is interpreted as indistinguishability;
  \item names are independent random samplings;
  \item function symbols correspond to deterministic machines.
\end{itemize}

In particular, function symbols corresponding to cryptographic primitives
are interpreted as implementations of the primitives, subject to some
assumptions.

For example, if $\mathsf{n}$ and $\mathsf{m}$ are distinct names,
the (atomic) formulas $\mathsf{n}\sim\mathsf{m}$ and
$\mathsf{n}\stackrel{.}{=}\mathsf{m}\sim\mathsf{false}$
are valid.

\section{Syntax of terms and formulas}

We recall next the syntax and semantics of the base logic, which deals with
messages and indistinguishability.
We then define the syntax of our meta-logic, which will allow to reason about
protocols. The semantics of the meta-logic will only be defined relative to a
protocol, a notion that is introduced in the next section.

Terms of the meta-logic are of three possible sorts: index, timestamps
or message. Formulas of the meta-logic may feature predicates over each sort.
Given some interpretation $I$ of indices and timestamp variables in some finite
sets, a meta-logic formula $\phi$ will be translated to a term $(\phi)^I$
of the base logic.
The validity of the meta-logic formula will be defined as the validity of the
base logic formula $(\phi)^I \sim \mathsf{true}$.

\medskip

We assume three infinite sets of variables:
$\X$ (whose elements are noted $x$, $y$, $z$) for message variables;
$\I$ (whose elements are noted $i$, $j$, $k$) for index variables;
$\XT$ (whose elements are noted $\tau$) for timestamp variables.

We assume a set $\F$ of indexed function symbols
(used to model encryptions, pairs,\dots).
Each of these symbols comes with an index arity as well as a message arity:
if $f\in\F$ has index arity $k$ and message arity $n$,
then for all index variables $i_1,\ldots,i_k$ and terms $t_1,\ldots,t_n$,
$f[i_1,\ldots,i_k](t_1,\ldots,t_n)$ is a term.

\begin{example}
  Function symbols representing cryptographic primitives will have index
  arity $0$, and a message arity depending on the kind of primitive:
  two for a keyed hash, three for a randomized encryption, etc.
  Functions symbols representing identities (for example, a constant value
  associated to each tag) will have $0$ as message arity and $1$ as index arity.
\end{example}

We assume a set $\N$ of indexed names symbols
(used to model random samplings)
and a set of indexed action symbols $\Actions$
(used to model specific timestamps).
These indexed symbols only have an index arity: they
cannot be applied to terms.

We finally assume a set $\M$ of indexed macro symbols.
An indexed macro symbol will be applied to the specified number of
indices and also to a single timestamp.
This timestamp gives the instant at which the macro is evaluated.
In practice, macro symbols will be used to describe the inputs and outputs
of the protocol, but also the execution conditions as well as the value of
mutable memory cells.

We shall write $\A_k$ (resp.\ $\N_k$ or $\M_k$) for the set of action
symbols (resp.\ name or macro symbols) of index arity $k$.
We write $\F_k^n$ for function symbols of index arity $k$ and message
arity $n$.

\begin{figure}[t]
\[
  \begin{array}{rclll}
    i &:=& i & \text{index variable} & \\
    \\
    T &:=& \tau & \text{timestamp variable} & \\
      &\mid& \tinit & \text{initial timestamp} & \\
      &\mid& a[i_1,\ldots,i_k] & \text{constant} & (a\in\Actions_k) \\
      &\mid & \pre(T) & \text{predecessor} & \\
    \\
    t & := & \mathsf{n}[i_1,\ldots,i_n] &\text{name} & (\mathsf{n}\in\N_k) \\
      & \mid & x  & \text{message variable} & (x\in\X) \\
      & \mid & f[i_1,\ldots,i_k](t_1,\dots,t_n) &\text{function application} &
      (f\in\F_k^n)\\
      & \mid & m[i_1,\ldots,i_k]@T &\text{macro application} & (m\in\M_k)\\
      & \mid & \myif \phi \mythen t \myelse t'
        &\text{conditional} \\
      & \mid & \myfind \vec{i} \mysuchthat \phi \myin t \myelse t'
        &\text{lookup}
  \end{array}
\]
\caption{Syntax of meta-logic terms}\label{fig:terms}
\end{figure}

\begin{figure}[t]
  \[
  \begin{array}{c}
   \begin{array}{rcll}
    \T &  := & \timestamp \mid \idx & \text{(meta logic sorts)} \\
    \\[2ex]
   \atom & := & t=t'
 & \text{atomic proposition over messages } \\
  &\mid & T=T' \mid T \leq T' \mid \mcond@T \mid \mexec@T &  \text{atomic proposition
  over timestamps } \\
  &\mid & i=i'  &  \text{atomic proposition
    over indices } \\
    \end{array}
\\
\\
     \begin{array}{rcll}
    \phi & ::= &  \true \mid \false \mid \phi \wedge \phi' \mid  \phi
    \vee \phi' \mid   \phi \Rightarrow \phi'\mid \neg \phi \mid
    \forall x : \T.\ \phi \mid \exists x:\T.\ \phi \mid \atom &
    \end{array}

\end{array}
    \]
    \caption{Syntax of meta-logic formulas}\label{fig:formulas}
\end{figure}

\begin{definition}
  Given a (meta logic) signature $\Sigma = (\F,\N,\M,\Actions)$
  and some sets of variables
  $\X$, $\I$ and $\XT$,
  \cref{fig:terms} defines the syntax of (meta-logic) terms
  of sort message (noted $t$) and timestamp (noted $T$),
  and \cref{fig:formulas} defines the syntax of formulas (noted $\phi$).
  Note that terms and formula are mutually inductive, due to the presence
  of conditionals and lookups in terms.
  The only terms of sort index are index variables.
  The set of message terms of the meta-logic is noted $\Msg_\Sigma$.
\end{definition}
\charlie{ We have $\XT$, $T$, $t$ and $\T$. This will probably be a bit confusing... }
%%% Local Variables:
%%% mode: latex
%%% TeX-master: "main"
%%% End:
