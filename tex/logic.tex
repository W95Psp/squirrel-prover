\section{Abstract meta-logic}

As a first step we define a meta-logic that is abstract, in that it does
not explicitly refer to the protocol under study. It has all the ingredients
for reasoning about protocols (mainly indexed symbols and macro symbols)
but their interpretation is, for now, arbitrary.

\subsection{Base logic}

The base logic is that of Bana and Comon (CCS'14). It is first-order logic
with a distinguished kind of constant for representing names (written
$\mathsf{n}$, $\mathsf{m}$, $\mathsf{k}$, etc.)
and featuring a single predicate noted $\sim$ (or rather, a family
of predicates $\sim_k$ of arity $2 k$ for all $k\in\mathbb{N}$).
The terms are all of the same sort: they are meant to represent
messages\footnote{
  David: I hope/believe that the distinction between booleans and messages
  can be ignored for the theory.
}.

In practice, the logic will be considered with function symbols for
representing cryptographic primitives as well as attacker computations, and
boolean constants and connectives.

The formulas of the base logic are intended to be interpreted in
\emph{computational models}, where:
\begin{itemize}
  \item terms are interpreted as PPT Turing machines which,
    given infinite random tapes and a security parameter output a bitstring;
  \item $\sim$ is interpreted as indistinguishability;
  \item names are independent random samplings;
  \item function symbols correspond to deterministic machines.
\end{itemize}

In particular, function symbols corresponding to cryptographic primitives
are interpreted as implementations of the primitives, subject to some
assumptions.

For example, if $\mathsf{n}$ and $\mathsf{m}$ are distinct names,
the (atomic) formulas $\mathsf{n}\sim\mathsf{m}$ and
$\mathsf{n}\stackrel{.}{=}\mathsf{m}\sim\mathsf{false}$
are valid.

\subsection{Terms}

Terms of the meta-logic are of three possible sorts: index, timestamps
or message. Formulas of the meta-logic may feature predicates over each sort.
Given interpretation $I$ and $T$ of index and timestamp variables
into two finite sets, a formula $\phi$ of the meta-logic will be translated
into a term $(\phi)^{I,T}$ of the base logic in which index and action terms,
and predicates over them, will have disappeared. The validity of the
meta-logic formula $\phi$ will be defined as the validity of the base logic
$(\phi)^{I,T}\sim\mathsf{true}$.

\medskip

We assume three infinite sets of variables:
$\X$ (whose elements are noted $x$, $y$, $z$) for message variables;
$\I$ (whose elements are noted $i$, $j$, $k$) for index variables;
$\XT$ (whose elements are noted $\tau$) for timestamp variables.

We assume a set $\F$ of indexed function symbols
(used to model encryptions, pairs,\dots).
Each of these symbols comes with an index arity as well as an usual arity:
if $f\in\F$ has index arity $k$ and arity $n$,
then for all index variables $i_1,\ldots,i_k$ and terms $t_1,\ldots,t_n$,
$f[i_1,\ldots,i_k](t_1,\ldots,t_n)$ is a term.

We assume a set $\N$ of indexed names symbols,
(used to model random samplings)
and a sets of indexed constants $\Actions$
(used to model specific timestamps).
These indexed symbols only have an index arity: they
cannot be applied to terms.

We finally assume a set $\M$ of indexed macro symbols
equipped with an arity and index arity, like function symbols.
An indexed macro symbol will be applied to the specified number of
indices, variables, and also to a single timestamp.
This timestamp gives the instant at which the macro is evaluated.
We will typically assume macro symbols $\mout$ and $\minp$
(with index arity $0$) for representing messages inputted and outputted
at a particular point of an execution.

We shall write $\F_k$ (resp.\ $\N_k$ or $\M_k$) for the set of function
symbols (resp.\ name or macro symbols) of index arity $k$.

\begin{figure}
\[
  \begin{array}{rcll}
    n &:=& \mathsf{n}[i_1,\ldots,i_n] & (\mathsf{n}\in\N_k)
    \\
    F &:=& f[i_1,\ldots,i_k] & (f\in\F_k)
    \\
    M &:=& m[i_1,\ldots,i_k] & (m\in\M_k)
  \end{array}
\]
\[
  \begin{array}{rcll}
    T &:=& \tau & \text{timestamp variable} \\
      &\mid& a[i_1,\ldots,i_k] & \text{constant (for $a\in\mathbb{A}$)} \\
      &\mid & \pre(T) & \text{predecessor}
\end{array}
   \]
\[
     \begin{array}{rcll}
    t & := & n &\text{name} \\
    & \mid & x  & \text{message variable $x\in\X$} \\
    & \mid & F(t_1,\dots,t_n) &\text{function application}\\
    & \mid & M(t_1,\ldots,t_n)@T &\text{macro application}\\
       \end{array}
     \]
     \caption{Meta-logic terms}\label{fig:terms}
\end{figure}

\begin{definition}
  Given a (meta logic) signature $\Sigma = (\F,\N,\M,\Actions)$
  and some sets of variables
  $\X$, $\I$ and $\XT$,
  \cref{fig:terms} defines the syntax of (meta-logic) terms
  of sort message (noted $t$) and timestamp (noted $T$).
  The only terms of sort index are index variables.
  The set of message terms of the meta-logic is noted $\Msg_\Sigma$.
\end{definition}

We can now define a meta-interpretation as the structure needed to
give a meaning to index and timestamp terms. A meta-interpretation
also induces a translation from a meta-logic signature $\Sigma$
to some base logic signature $\Sigma^I$. For example, if
$\Sigma$ contains a name $\mathsf{n}$ with index arity $1$, and
the meta-interpretation interprets indices in a domain $D_\I = \{
  17, 22 \}$, the base logic signature will feature
two names $\mathsf{n}_{17}$ and $\mathsf{n}_{22}$.

\begin{definition}
  A meta-interpretation $I$\footnote{\adrien{I don't like the name meta-interpretation. Meta-model would be better. Or something like trace-model. }}
  for $\Sigma = (\F,\M,\N,\Actions)$ consists in:
  \begin{itemize}
    \item two finite sets $D_\I$ and $D_\XT$ called the index and timestamp
      domains of the interpretation;
    \item mappings $\sigma_\I : \I \to D_\I$
      and $\sigma_\XT : \XT \to D_\XT$ that interpret index and
      timestamp variables as elements of their respective domains;
    \item a total ordering $\leq$ over $D_\XT$,
      a function $p : D_\XT \to D_\XT$ (for interpreting the
      predecessor\footnote{
        It does not matter that the predecessor means anything
        wrt.\ the ordering. We will impose later that it behaves
        well enough.
      })
      and a subset $H_\XT\subseteq D_\XT$ (for identifying timestamps
      that actuallly happen in an execution\footnote{
        We cannot identify timestamps freely in the meta-interpretation
        because we want to be able to use axioms such as
        $\forall i\neq j.~ a[i] \neq a[j]$ without restricting
        them to timestamps that really happen.
        \adrien{I think there are more fundamental reason for that. Considering the way for formulate things, for any $a \in \bbA$ (of arity 1) and $i \in D_\I$, $a[i]$ has an interpretation. But the protocol may not allow for it (because of restrictions). By consequence we need to be able to interpret $a[i]$ without it being valid.}
      });
    \item for each constant $a \in \Actions$ of index arity $k$,
      an interpretation $\hat{a} : D_\I^k \to D_\XT$;
    \item for each macro symbol $m \in \M$ of index
      arity $k$ and arity $n$, an interpretation
      $\hat{m} : D_\I^k \times \Msg_\Sigma^n \times D_\XT \to \Msg^I_\Sigma$,
      where $\Msg^I_\Sigma$ is the set of terms of the base
      logic for the (base logic) signature
      $(\F^I,\N^I)$ with
      \[
        \F^I =
        \{ f_{e_1,\ldots,e_k} : f \in \F_k, e_1,\ldots,e_k \in D_\I \}
        \text{ and }
        \N^I = \{ \mathsf{n}_{e_1,\ldots,e_k} : \mathsf{n}\in\N_k,
        e_1,\ldots,e_k\in D_\I \}.
      \]
      \adrien{The interpretation of macros can depend on the precise element of $\D_\T$ it is interpreted at. In the tool, this is more restricted. E.g.\ $\hat{m}@a[i]$ can only depend on $a$, not on $i$  (except to re-use $i$ in the term). Instead, we could interpret macro symbol as terms of the meta-logic (with restrictions to have termination).}
  \end{itemize}
\end{definition}

Note that, for each name $\mathsf{n}\in \N$ and indices $n_1,\ldots,n_k \in
D_\I$, $\mathsf{n}_{n_1,\ldots,n_k}$ refers to a distinct name in $\N^I$.
\adrien{The same is true for function symbols. The thing to remark here is that in a BC computational model, different names will always be interpreted by i.i.d.\ uniform random samplings. This is not true for function symbols ($\ne$ function symbols may have the same interpretation).}

Given a meta-interpratation $I$, we define the interpretation in $I$ of terms of the meta-logic has terms of the base logic.
\begin{definition}
  Given a meta-interpretation $I$
  we define $(T)^{I} \in D_\XT$ and $(t)^{I} \in \Msg^I_\Sigma$ as follows:
  \begin{eqnarray*}
    (\tau)^{I} &=& \sigma_\XT(\tau) \\
    (\pre(T))^{I} &=& p((T)^{I}) \\
    (a[i_1,\ldots,i_k])^{I} &=& \hat{a}(\sigma_\I(i_1),\ldots,\sigma_\I(i_k))
  \end{eqnarray*}
  \begin{eqnarray*}
    (\mathsf{n}[i_1,\ldots,i_k])^{I} &=& \mathsf{n}_{\sigma_\I(i_1),\ldots,\sigma_\I(i_k)}
    \\
    (x)^{I} &=& x
    \\
    (f[i_1,\ldots,i_k](t_1,\ldots,t_n))^{I} &=&
    f_{\sigma_\I(i_1),\ldots,\sigma_\I(i_k)}\bigl(
      (t_1)^{I},\ldots,(t_n)^{I}
    \bigr)
    \\
    (m[i_1,\ldots,i_k](t_1,\ldots,t_n)@T)^{I} &=&
    \hat{m}(\sigma_\I(i_1),\ldots,\sigma_\I(i_k),
      (t_1)^{I},\ldots,(t_n)^{I},
      (T)^{I})
  \end{eqnarray*}
\end{definition}

\begin{example}
  Consider the meta-logic term $t := \mathsf{h}(\mout@a[i],\mathsf{k}[i])$
  and an interpretation $I$ with $D_T = [1;10]$ and $D_I = [1;3]$
  (which might be relevant if we are considering traces of ten actions
  with three agents) such that $\sigma_I(i)=2$ and $\hat{a}(2)=10$
  and $\hat{\mout}(10)=\mathsf{ok}$ (the message outputted at step 10
  is $\mathsf{ok}$ \adrien{I don't like this. It seems that macro can be interperted as the term we want at any timestamp.}).
  We then have $(t)^I = \mathsf{h}(\mathsf{ok},\mathsf{k}_2)$.
\end{example}

The reason why macros take only one timestamp argument is purely practical:
we have no use for more. At this point one might wonder why we separate
indices and actions given that they are interpreted similarly: it is again
purely practical, we will use them for different purposes, and we will need
less structure on indices than on actions, making reasoning easier on them
(one can simply compare indices, there is no ordering and no predecessor
operation on them).

\subsection{Formulas}

The syntax of the meta-logic formulas is given in \cref{fig:syntax}.
We could have included more generally a notion of predicate macro that
would have to be expanded when translating a formula from the meta-logic
to the base logic. Instead we opted to list the few atoms that we
will use in practice.

\begin{figure}
  \[
  \begin{array}{c}
   \begin{array}{rcll}
    \T &  := & \timestamp \mid \idx & \text{(meta logic sorts)} \\
    \\[2ex]
   \atom & := & t=t'
 & \text{atomic proposition over messages } \\
  &\mid & T=T' \mid T \leq T' \mid \happens(T) &  \text{atomic proposition
  over timestamps } \\
  &\mid & i=i'  &  \text{atomic proposition
    over indices } \\
    \end{array}
\\
\\
     \begin{array}{rcll}
    \phi & ::= &  \true \mid \false \mid \phi \wedge \phi' \mid  \phi
    \vee \phi' \mid   \phi \Rightarrow \phi'\mid \neg \phi \mid
    \forall x : \T.\ \phi \mid \exists x:\T.\ \phi \mid \atom &
    \end{array}

\end{array}
    \]
    \caption{Syntax of first order formulas.}\label{fig:syntax}
\end{figure}

\begin{definition}
  If $I$ is a meta-interpretation and $e\in D_\I$,
  $I[i\mapsto e]$ is the interpretation where $\sigma_\I$ is
  modified so that $\sigma_{\I}(i)=e$.
  We define the translation of a meta-logic formula $\phi$
  into the base logic \emph{term} $(\phi)^I$
  as follows:
  \begin{eqnarray*}
    (\phi\wedge\phi')^I &=& (\phi)^I \stackrel{.}{\wedge} (\phi')^I
    \quad \text{and similarly for other boolean connectives} \\
    (\forall i:\idx. \phi)^I &=&
    \stackrel{.}{\wedge}_{e\in D_\I} (\phi)^{I[i\mapsto e]} \\
    (\forall \tau:\timestamp. \phi)^I &=&
    \stackrel{.}{\wedge}_{e\in D_\XT} (\phi)^{I[\tau\mapsto e]} \\
    (\exists i:\idx. \phi)^I &=&
    \stackrel{.}{\vee}_{e\in D_\I} (\phi)^{I[i\mapsto e]} \\
    (\exists \tau:\timestamp. \phi)^I &=&
    \stackrel{.}{\vee}_{e\in D_\XT} (\phi)^{I[\tau\mapsto e]} \\
    (i=i')^I &=&
    \left\{\begin{array}{ll}
      \mathsf{true} & \text{if } \sigma_\I(i)=\sigma_\I(i') \\
      \mathsf{false} & \text{otherwise}
    \end{array}\right. \\
    (T = T')^I &=&
    \left\{\begin{array}{ll}
      \mathsf{true} & \text{if } (T)^I = (T')^I \\
      \mathsf{false} & \text{otherwise}
    \end{array}\right.
    \quad\text{ and similarly for $\leq$} \\
    (\happens(T))^I &=& \mathsf{true}
    \text{ if } (T)^I \in H_\XT
    \text{ and } \mathsf{false} \text{ otherwise}
    \\
    (t=t')^I &=& (t)^I \stackrel{.}{=} (t')^I
  \end{eqnarray*}
\end{definition}

\begin{definition}
  A formula $\phi$ of the meta-logic is said to be valid when,
  for any meta-interpretation $I$, the base logic formula
  $(\phi)^I \sim \mathsf{true}$ is valid.
  (In other words, we have $\M,\sigma\models(\phi)^I\sim\mathsf{true}$
  i.e. the boolean term $(\phi)^I$ is true with overwhelming
  probability in any computational model $\M$ and for any interpretation
  $\sigma$ of the free message variables.\footnote{
    In the tool, we only allow universal quantification over messages, and
  only allow it at toplevel. The validity of such formulas is the same as
  when the variables are left free.})
\end{definition}

\begin{definition}
  A meta-logic formula $\phi$ is a
  logical consequence of a set $S$ of meta-logic formulas
  (noted $S \models \phi$)
  when
  $\M\models(\phi)^I\sim\mathsf{true}$ holds for any $\M$ and $I$ such that
  $\M\models(\psi)^I\sim\mathsf{true}$ holds for all $\psi\in S$.
  \adrien{This definition seems not necessary. We interpret meta-formula as (schemas of) first-order formulas. Hence entailment is exactly first-order logic entailment.}
\end{definition}

We want to verify that a formula $\phi$ (the security property)
holds for all executions of some protocol, in all computational models
satisfying some cryptographic assumptions.
This can be guaranteed if we have $Ax\models\phi$
where $Ax$ is a (recursive) set of axioms that are sound
wrt.\ the intended class of meta-interpretations and computational models.
We might be able to formulate such a set $Ax$ (which would consist of
two parts, first the usual BC axioms, second some meta-logic axioms
constraining the meta-interpretations to correspond to protocol executions).
However, we will go for an easier task: designing a set of inference
rules that allow to derive meta-logic formulas (rather, sequents)
that are valid in all intended models.

\subsection{Sequent calculus}

In the tool, sequents come with an environment which explicitly
declares all the variables that might occur free in the sequent's formulas.
For simplicity, we do not include them here. We shall write $\vdash t:T$
when $t$ is a term of sort $\T$ (which might be $\idx$ or $\timestamp$).

\begin{definition}
  A sequent $\Gamma \vdash \phi$ is composed of a set of meta-logic formulas
  $\Gamma$ and a meta-logic formula $\phi$.
  It is valid when the meta-logic formula
  $(\wedge\Gamma) \Rightarrow \phi$ is valid.
\end{definition}

\begin{proposition}
  The rules of \cref{fig:lk,fig:names} are sound: if the premisses are valid,
  then so is the conclusion.
\end{proposition}

\begin{figure}
  \begin{mathpar}
    \inferrule[Axiom]{~}{\Gamma,\phi\vdash\phi}
    \quad\quad
    \inferrule[Cut]{
      \Gamma \vdash \phi
      \quad
      \Gamma,\phi \vdash \psi
    }{
      \Gamma \vdash \psi
    }
  \end{mathpar}
  \begin{mathpar}
  \inferrule[${\wedge}$-L]{
    \Gamma,\phi,\phi'\vdash\psi
  }{
    \Gamma,\phi\wedge\phi'\vdash\psi
  }
  \quad\quad
  \inferrule[${\wedge}$-R]{
    \Gamma \vdash \phi
    \quad
    \Gamma \vdash \phi'
  }{
    \Gamma \vdash \phi\wedge\phi'
  }
  \end{mathpar}
  \begin{mathpar}
    \inferrule[${\lnot}$-R]{
      \Gamma, \phi \vdash \bot
    }{
      \Gamma \vdash \lnot\phi
    }
    \quad\quad
    \inferrule[${\lnot}$-L]{
      \Gamma \vdash \phi
    }{
      \Gamma,\lnot\phi \vdash \psi
    }
    \quad\quad
    \inferrule[Raa]{
      \Gamma, \lnot\phi \vdash \bot
    }{
      \Gamma \vdash \phi
    }
  \end{mathpar}
  \begin{center}
  \emph{other propositional rules of classical sequent calculus}
  \end{center}
  \begin{mathpar}
      \inferrule[${=}$-R]{~}{\Gamma \vdash t=t} \quad\quad
      \inferrule[${=}$-L]{
        (\Gamma\vdash\phi)\{x\mapsto t',x'\mapsto t\}
      }{
        t=t', \Gamma\{x\mapsto t,x'\mapsto t'\} \vdash
        \phi\{x\mapsto t,x'\mapsto t'\}
      }
  \end{mathpar}
  \begin{mathpar}
     \inferrule[$\forall$-L]{
       \Gamma,\phi\{x\mapsto t\} \vdash \psi
       \quad
       \vdash t:\T
     }{
       \Gamma,\forall x:\T.\phi \vdash \psi}
     \quad\quad
     \inferrule[$\forall$-R]{
       \Gamma \vdash \phi}{\Gamma \vdash \forall x:\T:\phi}
  \end{mathpar}
   \caption{Generic inference rules. These are the rules of classical
   first-order sequent calculus. In \textsc{$\forall$-R} we require that
   $x$ does not appear free in $\Gamma$.}
   \label{fig:lk}
\end{figure}

\begin{figure}
  \begin{mathpar}
  \inferrule{
    \mathsf{n}\neq\mathsf{m}
  }{
    \Gamma,\mathsf{n}[\vec i]=\mathsf{m}[\vec j]\vdash \phi
  }
  \quad\quad
  \inferrule{
    \Gamma, i_1=j_1, \ldots, i_k=j_k \vdash \phi
  }{
    \Gamma,\mathsf{n}[i_1,\ldots,i_k]=\mathsf{n}[j_1,\ldots,j_k]\vdash \phi
  }
  \end{mathpar}
  \begin{mathpar}
    \inferrule{~}{
      \Gamma,t=\mathsf{n}[\vec i] \vdash \phi
    }
  \end{mathpar}
  \caption{Inference rules for equalities on names.
  The last rule only applies when, for any meta-interpretation $I$,
  the term $(t)^I$ does not contain any message variable and does
  not contain any occurrence of $\mathsf{n}_{\sigma_\I(\vec i)}$
  --- this condition will be over-approximated in our implementation.}
  \label{fig:names}
\end{figure}

\begin{proposition}
  The rules of \cref{fig:xor} are sound wrt.\ computational models
  where $\oplus$ is interpreted as exclusive or:
  if the premisses are valid in all such computational models,
  then so is the conclusion.
  \emph{Do we need to impose a condition on message lengths?}
\end{proposition}

\begin{figure}
  \begin{mathpar}
    \inferrule{~}{\Gamma \vdash t\oplus t = 0}
    \quad\quad
    \inferrule{~}{\Gamma \vdash t\oplus t' = t'\oplus t}
    \quad\quad
    \inferrule{~}{\Gamma \vdash t\oplus (t'\oplus t'') = (t\oplus t')\oplus t''}
  \end{mathpar}
  \begin{mathpar}
    \inferrule{
      (\Gamma \vdash \phi)\{x\mapsto \mathsf{m}[\vec j]\}
    }{
      (\Gamma \vdash \phi)\{x\mapsto t\oplus\mathsf{n}[\vec i]\}
    }
  \end{mathpar}
  \caption{Inference rules for exclusive or.
  The tool does not follow these rules closely, but uses a more high-level
  congruence closure algorithm modulo xor. The last rule is not implementated
  at all and is here mostly for illustration purposes, and is subject to
  the condition that, for any meta-interpretation $I$,
  $(\Gamma\vdash\phi)^I$ does not contain instances of the names
  $\mathsf{n}$ and $\mathsf{m}$ and does not contain message variables.
  This condition would be over-approximated in an implementation.}
  \label{fig:xor}
\end{figure}

%%% Local Variables:
%%% mode: latex
%%% TeX-master: "main"
%%% End:
