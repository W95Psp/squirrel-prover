\section{Indistinguishability}

\newcommand{\pair}[1]{\langle #1 \rangle}

We now define indistinguishability of processes, and how we could verify it.
We make a distinction between symbolic and observable traces:
symbolic ones are sequences of action names as in \cref{def:trace},
while observable traces are sequences of pairs $\pair{c_i,c_o}$.
Intuitively, such a pair describes the input and output channels of an 
action; several actions might have the same input and output channels.
We write $A : \pair{c_i,c_o}$ when action $A$ inputs on channel $c_i$ and
outputs on channel $c_o$.

For convenience, we talk of processes, though in reality the definitions
deal with abstract systems described by sets of actions.

\newcommand{\fold}{\mathsf{fold}}

\begin{definition}
  Given an observable trace $t$ and a process $P$,
  we define $\fold(P,t)$ as the frame describing all possible
  executions of $P$ along $t$:
  \begin{eqnarray*}
    \fold(P,\epsilon) &=& \epsilon \\
    \fold(P,(c_i,c_o).t') &=&
      \mathsf{if}_{A:(c_i,c_o)}
      \ldots
  \end{eqnarray*}
\end{definition}

\begin{definition}
  Two processes $P$ and $Q$ are indistinguishable when,
  for all observable traces $t$,
  for all computational model $\M$,
  we have
  $\M \models \fold(P,t) \sim \fold(Q,t)$.
  This itself means that no attacker can distinguish,
  with non-negligible probability, between the left- and right-hand sides.
\end{definition}

This notion of equivalence should coincide with the usual computational
indistinguishability for bounded processes. In the unbounded case, it
is restrictive to quantify on traces in this way
and ask for indistinguishability only for each trace.

\subsection{Straightforward diff-equivalence}

\newcommand{\diff}[2]{\mathsf{choice}(#1,#2)}

Two processes may be given as a bi-process, i.e. a system with a single
set of symbolic actions, each symbolic action describing an execution step
of both the left and right processes, using $\diff{\_}{\_}$
terms as usual. In that case, we may define a stronger equivalence,
relying on a simpler notion of frame which does not contain conditionals,
because we reason directly on symbolic traces.
We use $(\_)^L$ and $(\_)^R$ to denote the left and right projections,
e.g. $\phi_\tr^L$ is the conjunction of all the left-hand side
conditions for the left side of the bi-process.

\begin{definition}
  Given a symbolic trace $\tr = \alpha_1\ldots\alpha_n$, we define its frame 
  $\mouts_\tr = \pair{\hat{\mout}(\alpha_1),\dots,\hat{\mout}(\alpha_n)}$
  as the sequence of outputs produced by the trace.
\end{definition}

Note that this frame is not a sequence of meta-level terms,
but a sequence of base-level terms featuring
$\diff{\_}{\_}$ operators.
We now define the equivalence,
writing $\myif \phi \mythen t$ for $\myif \phi \mythen t \myelse 
\mathsf{error}$, with an error constant that is unused elsewhere.

\begin{definition}
  A bi-process is diff-equivalent when,
  for any symbolic trace $\tr$ of the protocol and
  for any computational model $\M$,
  we have:
  $$\M \models
  \myif (\phi_{\tr^L})^{I_{\tr}} \mythen (\mouts_{\tr})^L
  \sim   
  \myif (\phi_{\tr^R})^{I_{\tr}} \mythen (\mouts_{\tr})^R$$
\end{definition}

Note that this equivalence cannot hold if there exists a $\tr$
whose probability of execution significantly differs between the two
projections of the bi-process.
Hence this imposes a form of synchronization on the execution of
conditionals on the two sides of bi-processes.
The interest of imposing this artificial constraint is that we
can stop considering executions where one process goes to $\mythen$
branch while the other goes to its $\myelse$ branch: this is implicit
in the fact that we consider here symbolic and not observable traces.

\newcommand{\In}{\mathsf{in}}
\newcommand{\Out}{\mathsf{out}}

\begin{example}
  Consider the bi-process
  $\In(c,x).\myif x=\diff{0}{1} \mythen \Out(c,n) \myelse \Out(c,m)$
  where $n$ and $m$ are arbitrary, possibly equal names.
  Its two projections are indistinguishable, but the
  bi-process is not diff-equivalent.
  Indeed we have 
  $\myif g()=0 \mythen \Out(c,n) \not\sim
   \myif g()=1 \mythen \Out(c,m)$: the attacker can simply choose
  $g()=0$ to distinguish the two sides.
  Our bi-process can however easily be
  rewritten into a diff-equivalent process, e.g. by pushing the conditional
  inside the output.

  If we modify our bi-process into
  $\In(c,x).\myif x=\diff{0}{1} \mythen \Out(c,n) \myelse \Out(c,0)$
  then the two projections become distinguishable.
  The attack is obtained with an execution
  where one process outputs a name while the other outputs $0$. Such
  desynchronized executions are not taken into account with diff-equivalence,
  but diff-equivalence still fails due to the desynchronized condition,
  as before.
\end{example}

\newcommand{\ok}{\mathsf{ok}}

In the next example, we omit the $\myelse$ branch when it consists of a null 
process.

\begin{example} \label{ex:negl}
  Consider the bi-process
  $\In(c,x).\myif x=\diff{n}{m} \mythen \Out(c,\ok)$.
  It is diff-equivalent, and its projections are
  indistinguishable as expected.
  In the bi-process, the condition $x=\diff{n}{m}$ does not pass
  with the same inputs on the left and right, but it passes with
  the same negligible probability.
\end{example}

\begin{example} \label{ex:sync}
  Consider
  $\In(c,x).\myif x=(\diff{n}{m})_0 \mythen \Out(c,n)$
  where $(t)_0$ denotes the first bit of $t$.
  This bi-process is not diff-equivalent because
  $\myif x=(n)_0 \mythen n \not\sim \myif x=(m)_0 \mythen n$:
  there is a non-negligible probably that one conditional only passes.
  The two projections are distinguishable: the attacker sends $0$;
  on the left he receives with probability $1\over 2$ a bitstring whose
  first bit is $0$;
  on the right process he receives with probably only $1\over 4$
  a bitstring whose last bit is $0$.

  If we change $\Out(c,n)$ into $\Out(c,\ok)$,
  we have indistinguishable processes but diff-equivalence still fails.
\end{example}

\begin{example}
  Consider $\Out(c,\diff{n}{m}).
  \In(c,x).
  \myif x=\diff{n}{m} \mythen \Out(c,\ok)$.
  The projections are observationally equivalent -- in fact they are
  $\alpha$-equivalent.
  The bi-process is not diff-equivalent for the same reason as in the
  previous example.
  \emph{TODO: does it change anything if we replace all occurrences
  of $n$ and $m$, respectively, by $h(n,k)$ and $h(m,k)$ for a fixed $k$?}
  \emph{TODO: make sure that this failure of diff-equivalence will not
  appear in our case studies.}
\end{example}

\subsubsection{Formal verification}

Ultimately we will want to use the meta-logic to avoid enumerating all
traces, but reason abstractly over them. However, as a first step we
keep the enumeration.

Given a bi-process $B$, we should be able to give a meaning to its inputs
and outputs along a trace using macros $\minp^L$, $\minp^R$, $\mout^L$
and $\mout^R$. For instance, $\minp^L@A$ represents the input term for
the left at timestamp $A$ in a trace --- the meta-interpretation will
be such that
$\hat{\minp}^L(A_n) = g_n(\hat{\mout}^L(A_1),\ldots,\hat{\mout}^L(A_{n-1}))$.

Then, for any symbolic trace $\tr.\alpha$ of $B$, prove the following goals:
$$ \phi^L_\tr, \phi^R_\tr \vdash \phi^L_\alpha \Leftrightarrow \phi^R_\alpha $$
$$ \phi^L_{\tr.\alpha}, \phi^R_{\tr.\alpha} \vdash 
   \mouts_{\tr.\alpha}^L \sim \mouts_{\tr.\alpha}^R $$
The first kind of judgment has been defined above: these are the sequents
of the meta-logic for reachability. The second kind of judgment is defined
below: $\Gamma \vdash u \sim v$ stands for $(\myif \Gamma \mythen u) \sim
(\myif \Gamma \mythen v)$. For these goals we are thus imposing a left-hand
side condition on right-hand side outputs, but it should not matter because
the left- and right-hand side conditions have been shown to hold 
simultaneously (with overwhelming probability).

\emph{TODO: Check that it implies diff-equivalence as previously defined.}

\begin{example}
  With the bi-process of \cref{ex:negl} we would have to prove
  $\vdash g() = n \Leftrightarrow g() = m$ and
  $g()=n, g()=m \vdash \ok \sim \ok$, both of which hold.
  With the bi-process of \cref{ex:sync} we would have to prove
  $\vdash g() = (n)_0 \Leftrightarrow g() = (m)_0$ which does not hold.
\end{example}

\subsection{Refined diff-equivalence}

The Basic-Hash protocol is unlinkable, wrt.\ Solène's definition.
This has been verified in the symbolic model and we expect that it holds
in the computational one too.

Formally, modify \texttt{examples/basic-hash.mbc} to consider the
following bi-process:
$$(!_k R(j)) \mid (!_i !_j T(\diff{k(i)}{k'(i,j)}))$$

On the left side, we have unboundedly many
identities represented by keys $k(i)$, and each $k(i)$ is used for
many sessions $j$.
On the right, each identity $k'(i,j)$ is used for a single session.
Note that we cannot simply write $k'(j)$ or $k(j)$, because then the
same identity $k'(j)$ would appear in the different copies of the
first $!_i$.
On either sides, the tag uses a session nonce $n_T(i,j)$.

On both sides we have unboundedly many sessions of the reader,
with a conditional as follows ($x$ is the first input message):
$$\mathsf{find} \ i,j \ \mathsf{suchthat} \ \pi_2(x) = 
h(\pi_1(x),\diff{k(i)}{k'(i,j)})$$

To sum up, we have the following actions:
\begin{itemize}
  \item $T(i,j)$ where the tag sends
    $\pair{n_T(i,j),h(n_T(i,j),\diff{k(i)}{k'(i,j)})}$;
  \item $R(k,i,j)$ is the only action with a conditional,
    where reader session $k$ accepts its first message,
    having found out that it corresponds to identity $k(i)$ (if on
    the left) or $k'(i,j)$ (on the right).
\end{itemize}

We cannot hope to have a synchronized execution of $R(k,i,j)$ on the
left and right: index $j$ is unused on the left!
However, we can hope to prove that, if $R(k,i,j)$ happens on the left,
then $R(k,i,j')$ happens on the right for some $j'$: this intuitively
follows from well-authentication.
Note that this $j'$ is necessarily unique by definition of the semantics:
the $\mathsf{tryfind}$ selects the first $j$ for which the condition holds.

Conversely, it is only easier to prove that if $R(k,i,j)$ happens
on the right, then it happens on the left, for any $j'$.

Having this form of synchronization for the conditionals is intuitively
enough to conclude the the two processes are indistinguishable,
provided we have also verified that the produced frames are indistinguishable:
indeed, for any observable trace, if the trace is executed on one side
using a symbolic trace $\tr$, then it can be executed on the other side
(modulo the changes of indices as above) to yield an indistinguishable
frame. Of course, all this has to be done more carefully, taking into
account probabilities, etc.

\emph{TODO: Develop this idea into a refined notion of diff-equivalence
which could be used to prove this kind of unlinkabilities. It would have
a specific treatment of actions when it is detected that an index is unused
on one side of the bi-process. We might adapt the verification conditions
of the previous section as follows, for all symbolic traces $\tr$:}
$$ \phi^L_\tr, \phi^R_\tr \vdash
   \phi^L_\alpha \Leftrightarrow (\exists j.~ \phi^R_\alpha) $$
$$ \phi^L_{\tr.\alpha}, \phi^R_{\tr.\alpha} \vdash 
   \mouts_{\tr.\alpha}^L \sim \mouts_{\tr.\alpha}^R $$
\emph{In the first kind of condition, the existential is here to account
for the possible change of some index. There is no change in the second
kind of condition: there we only consider executions where the conditions
have passed on both sides, restricting indices $j$.}

\subsection{Reasoning about equivalences}

Now equipped with a logic to reason about reachability properties, we turn 
ourself to proving general indistinguishability. We consider the case of 
bi-processes, where the set of actions available on both side is the same, but 
the actions may differ inside their terms. Thus, a bi-process is simply a 
protocol, where each action $\alpha$ has two versions, on for the left and on 
for the right process, which we denote $\alpha^L$ and $\alpha^R$. For each 
possible trace $\tr$ of the protocol, there exists then the two versions 
$\tr^L$ and $\tr^R$.

To reason about those properties, we once again try to define a well-suited logic.
\begin{definition}
  A sequent $\Delta, \Gamma \vDash t^L \sim t^R $ is composed of a set of meta-logic formulas
  $\Gamma$, a set of indistinguishability $\Delta$, and sequences of terms $t^L,t^R$.
  It is valid when, for any meta-interpretation $I$ and computational model $\M$, $\M, \Delta \models \myif \bigwedge_{\phi \in \Gamma} (\phi)^I \mythen t^L \sim  \myif \bigwedge_{\phi \in \Gamma} (\phi)^I \mythen t^R $
\end{definition}


We provide in \cref{fig:lk-ind} a set of sound rules for this second sequent calculus.

\begin{figure}
  \begin{mathpar}
    \inferrule[Expand]{\Gamma \vDash t^L \sim t^R
\quad
    }{\Gamma \cup \phi \vDash t^L \sim t^R}
    \quad\quad
    \inferrule[Cut]{
      \Gamma \vDash t^L \sim t^R
      \quad
      \Gamma \vDash t^R \sim t^S
    }{\Gamma \vDash t^L \sim t^S}
  \end{mathpar}
  \begin{mathpar}
  \inferrule[\myif-R]{
\Gamma \vDash \myif \phi \mythen t^L \sim \myif \phi \mythen t^R
  }{
 \Gamma \cup \phi \vDash t^L \sim t^R
  }
  \quad\quad
  \inferrule[\myif-L]{
 \Gamma \cup \phi \vDash t^L \sim t^R
  }{
\Gamma \vDash \myif \phi \mythen t^L \sim \myif \phi \mythen t^R
  }
  \end{mathpar}
  \begin{mathpar}
    \inferrule[${\lnot}$-R]{
      \Gamma \vdash \bot
    }{
      \Gamma \vDash t^L \sim t^R
    }
    \quad\quad
    \inferrule[fresh]{
n,m \not \in \Gamma
    }{
      \Gamma \vDash n \sim m
    }
    \quad\quad
    \inferrule[Subst]{
      \Gamma \cup \{\EQ(t_1,t_2) \}  \vDash t^L \sim t^R
    }{
      \Gamma \cup \{\EQ(t_1,t_2) \} \vDash t^L[ t_1 / t_2] \sim  t^R[ t_1 / t_2]
    }
  \end{mathpar}
  \begin{mathpar}
  \inferrule[F-Cut]{
    \Gamma \vdash \phi
    \quad
    \Gamma \cup \phi \vDash t^L \sim t^R
  }{
 \Gamma \vDash t^L \sim t^R
  }
  \quad\quad
  \inferrule[Dup]{
 \Gamma \vDash t^L,u \sim t^R,v
  }{
 \Gamma \vDash t^L,u,u \sim t^R,v,v
  }
  \end{mathpar}


   \caption{Generic inference rules for indistinguishability}
   \label{fig:lk-ind}
\end{figure}
\begin{lemma}
The rules presented in


\end{lemma}


We also give a small Lemma, which allows to reason on the traces in an incremental way.

\begin{lemma}
  A bi-process is diff-equivalent if for any symbolic trace $\tr$ of the protocol, and any available action $\alpha$,
\[\phi_{\tr^L} \vDash
\begin{array}[t]{l}
    \myif (\phi_{\alpha^L})^{I_{\tr^L \cdot \alpha^L}} \mythen (\mouts_{\tr^L\cdot \alpha^L})^{I_{\tr^L\cdot \alpha^L}}
    \\ \qquad
    \sim  \myif (\phi_{\alpha^R})^{I_{\tr^L \cdot \alpha^R}} \mythen (\mouts_{\tr^L\cdot \alpha^R})^{I_{\tr^L\cdot \alpha^R}} \end{array}\]

\end{lemma}

%%% Local Variables:
%%% mode: latex
%%% TeX-master: "main"
%%% End:
