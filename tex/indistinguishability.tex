\section{Indistinguishability}

Now equipped with a logic to reason about reachability properties, we turn ourself to proving general indistinguishability. We consider the case of bi-processes, where the set of actions available on both side is the same, but the actions may differ inside their terms. Thus, a bi-process is simply a protocol, where each action $\alpha$ has two versions, on for the left and on for the right process, which we denote $\alpha^L$ and $\alpha^R$. For each possible trace $\tr$ of the protocol, there exists then the two versions $\tr^L$ and $\tr^R$.


We first define the frame corresponding to a trace.
\begin{definition}
  Given a trace $\tr = \alpha_1,\dots,\alpha_n$, we define its frame $\mouts_\tr = \hat{\mout}(\alpha_1),\dots,\hat{\mout}(\alpha_n)$ as  the sequence of outputs produced by the trace.
\end{definition}

We can now define diff-equivalence for bi-processes. Essentialy, for each trace such that it can happen with non negligible probabilities, the left and right frames must be indistinguishable.

\begin{definition}
  A bi-process is diff-equivalent if for any symbolic trace $\tr$ of the protocol,
  we have $\M \models \myif (\phi_{\tr^L})^{I_{\tr^L}} \mythen (\mouts_{\tr^L})^{I_{\tr^L}} \sim   \myif (\phi_{\tr^R})^{I_{\tr^R}} \mythen (\mouts_{\tr^R})^{I_{\tr^R}} $.
\end{definition}

To reason about those properties, we once again try to define a well-suited logic.
\begin{definition}
  A sequent $\Delta, \Gamma \vDash t^L \sim t^R $ is composed of a set of meta-logic formulas
  $\Gamma$, a set of indistinguishability $\Delta$, and sequences of terms $t^L,t^R$.
  It is valid when, for any meta-interpretation $I$ and computational model $\M$, $\M, \Delta \models \myif \bigwedge_{\phi \in \Gamma} (\phi)^I \mythen t^L \sim  \myif \bigwedge_{\phi \in \Gamma} (\phi)^I \mythen t^R $
\end{definition}


We provide in \cref{fig:lk-ind} a set of sound rules for this second sequent calculus.

\begin{figure}
  \begin{mathpar}
    \inferrule[Expand]{\Gamma \vDash t^L \sim t^R
\quad
    }{\Gamma \cup \phi \vDash t^L \sim t^R}
    \quad\quad
    \inferrule[Cut]{
      \Gamma \vDash t^L \sim t^R
      \quad
      \Gamma \vDash t^R \sim t^S
    }{\Gamma \vDash t^L \sim t^S}
  \end{mathpar}
  \begin{mathpar}
  \inferrule[\myif-R]{
\Gamma \vDash \myif \phi \mythen t^L \sim \myif \phi \mythen t^R
  }{
 \Gamma \cup \phi \vDash t^L \sim t^R
  }
  \quad\quad
  \inferrule[\myif-L]{
 \Gamma \cup \phi \vDash t^L \sim t^R
  }{
\Gamma \vDash \myif \phi \mythen t^L \sim \myif \phi \mythen t^R
  }
  \end{mathpar}
  \begin{mathpar}
    \inferrule[${\lnot}$-R]{
      \Gamma \vdash \bot
    }{
      \Gamma \vDash t^L \sim t^R
    }
    \quad\quad
    \inferrule[fresh]{
n,m \not \in \Gamma
    }{
      \Gamma \vDash n \sim m
    }
    \quad\quad
    \inferrule[Subst]{
      \Gamma \cup \{\EQ(t_1,t_2) \}  \vDash t^L \sim t^R
    }{
      \Gamma \cup \{\EQ(t_1,t_2) \} \vDash t^L[ t_1 / t_2] \sim  t^R[ t_1 / t_2]
    }
  \end{mathpar}
  \begin{mathpar}
  \inferrule[F-Cut]{
    \Gamma \vdash \phi
    \quad
    \Gamma \cup \phi \vDash t^L \sim t^R
  }{
 \Gamma \vDash t^L \sim t^R
  }
  \quad\quad
  \inferrule[Dup]{
 \Gamma \vDash t^L,u \sim t^R,v
  }{
 \Gamma \vDash t^L,u,u \sim t^R,v,v
  }
  \end{mathpar}


   \caption{Generic inference rules for indistinguishability}
   \label{fig:lk-ind}
\end{figure}
\begin{lemma}
The rules presented in


\end{lemma}


We also give a small Lemma, which allows to reason on the traces in an incremental way.

\begin{lemma}
  A bi-process is diff-equivalent if for any symbolic trace $\tr$ of the protocol, and any available action $\alpha$,
\[\phi_{\tr^L} \vDash
\begin{array}[t]{l}
    \myif (\phi_{\alpha^L})^{I_{\tr^L \cdot \alpha^L}} \mythen (\mouts_{\tr^L\cdot \alpha^L})^{I_{\tr^L\cdot \alpha^L}}
    \\ \qquad
    \sim  \myif (\phi_{\alpha^R})^{I_{\tr^L \cdot \alpha^R}} \mythen (\mouts_{\tr^L\cdot \alpha^R})^{I_{\tr^L\cdot \alpha^R}} \end{array}\]

\end{lemma}

%%% Local Variables:
%%% mode: latex
%%% TeX-master: "main"
%%% End:
